\chapter{Introducción}\label{Cap:Introduccion}
 
\subsection{Motivación}
\linenumbers


Debido a las condiciones climáticas especificas del territorio uruguayo. Se produce una atmósfera inestable provocada por el choque de masas de aire calientes, originadas en los trópicos, y corrientes de aires fríos que migran desde el polo. Esta eminente peligro produce vientos extremos no sinópticos sumamente violentos y destructivos. Un trágico evento se sucedió el 10 de marzo de 2002 cuando una tormenta convectiva afecto un área de alderredor 6500 km$^2$ en el sur del país \cite{tormenta2002}. En el norte de Montevideo los anemómetros capturaron velocidades de ráfaga de 34 $m/s$ y de acuerdo con el nivel daño causado se, en determiandos puntos podría haber superado los 56 $m/s$. Fue tal el nivel de devastación que causó el colapso de 19 torres de trasmisión eléctrica de 500 kV y 48 de 150kV, además de unos 700 edificios y 1250 techos de hogares que fueron destruidos \citep{duranona2015significance}. No solo afecto a las construcciones sino también muchos productores rurales y sus estancias productivas derribando invernaderos, montes y plantaciones. El costo de reparación de las torres es estimo en 2 millones de dolares y en simultaneo se gastaron unos 10 millones de dolares destinados para suplir la red con energía geotérmica proveniente de combustibles fósiles . En total los daños fueron costeados con un presupuesto de unos 27 millones de dolares \cite{duranona2019first}.

Las líneas de trasmisión eléctrica son frecuentemente afectadas por eventos climáticos severos como corrientes descendentes o tornados. Estos eventos pueden provocar su desconexión, con consecuencias potencialmente graves. En el periodo 2000-2007 se registraron más de veinte eventos de salida en servicio por esta causa en una de las principales líneas de Uruguay (Palmar-Montevideo). Este tipo de fenómenos inducen fuertes movimientos en los cables, provocando un balanceo excesivo de los mismos. Estas amplitudes desmesuradas implican vulneraciones en la aislación del sistema, al aproximar sus cadenas de aisladores a las torres. Produciéndose descargas a tierra e indeseables interrupciones del suministro que han afectado a la capital durante varias horas. El modelado estructural de vientos severos sobre líneas de transmisión eléctrica ha sido abordado por la comunidad científica internacional desde diversas ópticas a lo largo de las últimas cuatro décadas. Se han presentado modelos semi-analíticos, análisis experimentales en túneles de viento y en campo más recientemente en modelos numéricos.
 
Esto plantea la necesidad de contar con herramientas complementarias que sean capaces de emular la respuesta de estos sistemas ante perfiles de viento no sinópticos. Este es el principal objetivo de este trabajo, profundizar en la bibliografía para el modelado estructural de conductores y crear un modelo robusto, consistente capaz de simular lineas de trasmisión eléctrica ante vientos los medidos experimentalmente en \cite{stengel2017measurements}. 
  

\subsection{Enfoque}

Numerosos autores de la literatura han acuñado sus investigaciones en elementos multinodales de barras como ser: \cite{desai1995finite}, \cite{yan2009numerical} y los trabajos \cite{gani2010dynamic} \cite{yang2016nonlinear}. No obstante, debido a la inherente rigidez a flexión en el comportamiento estructural del cable deben considerarse vigas tridimensionales. Por otra parte, los grandes desplazamientos y rotaciones que se presentan durante las trayectorias en tormentas, conducen a implementar un modelo de vigas apto para este tipo de solicitaciones. El abordaje corrotacional es propicio para este tipo de aplicaciones, pues desde su base matemática, se construye desacoplando la deformación local con deformaciones cinemáticas de cuerpo rígido para grandes amplitudes. Estas es el atractivo fundamental de la propuesta corrotacional, su versatilidad ante diferentes formulaciones locales. Permitiendo incorporar distintos tipos de elementos, fácilmente.

El campo de la metodología corrotacional es muy amplio, pero debido a la claridad y contemporaneidad en el desarrollo de sus publicaciones, se eligió un grupo de investigadores específicos. En \citep{Le2011} se publicó una formulación para vigas 2D, en conjunción con la parte estática desarrollada por el Dr. Jean Marc Battini  en \citep{Battini2002}. La extensión dinámica de este último, devino en el artículo \citep{Le2014} que se implementó en esta tesis. Lo innovador y atractivo se centra en el desarrollo analítico consistente no solo para los términos estáticos sino también dinámicos. Además en comparación con otras formulaciones se obtienen resultados certeros y confiables con un menor numero de elementos, ventaja principal para modelar grandes dominios como en el caso de lineas de alta tensión.  

Debido a las ventajas mencionas, esta metodología es implementada en diversos campos de aplicación ingenieril. La robustez, solidez y versatilidad del modelo es un atractivo importante que la hace aplicable en vastos campos de la ingeniería entre otras: aeronaves, turbinas propulsoras, molinos. En particular la formulación \citep{Le2014}  ha sido aplicado en trabajos recientes en el área de ingeniera marina, robótica y civil en \citep{albino2018co}, \cite{asadi2019multibody} y \cite{viana2020formulation}. Esto evidencia que la metodología es potente para diversos campos de estudio. No obstante, según el conocimiento del autor, ningún software comercial hasta la fecha utiliza formulaciones corrotacionales para la solución de problemas dinámicos. Asimismo esta no ha sido aplicada conductores sometidos a vientos extremos donde se desarrollan grandes amplitudes en distancias de centenas de metros.

En la temática específica de conductores, la tesis del autor \citet{foti2013corotational} destaca por su publicación detallada utilizando elementos corrotacionales de vigas 3D. Estudios experimentales mostraban discordancias respecto al modelo, debido a dos factores, las actualizaciones angulares mediante aproximaciones incrementares y el comportamiento inmanente del sistema. En trabajos posteriores del mismo autor, se corrigen las limitaciones y modelan los deslizamientos internos de las hebras y su histéresis sobre el fenómeno \citet{foti2018finite}. La aplicación de estos modelos sometidos ante tormentas conectivas aun es una interrogante. Y también así el perjuicio de las mismas sobre la continuidad e integridad del servicio. 

\subsection{Estructura}

 Este documento consta de cinco capítulos: Introducción, Estado del arte, Preliminares, Resultados Numéricos y Conclusiones. Inicialmente en el Capítulo \ref{Cap:EstadoDelArte} se realiza un recorrido histórico por la bibliografía consultada en materia de simulaciones aplicadas a conductores eléctricos, con un enfoque computacional y semi analítico. También se narran los diferentes estudios locales e internacionales sobre vientos extremos para concluir en un tour dentro del abordaje corrotacional. Continuamente en el Capítulo \ref{Cap:Preliminares}, con el objetivo de acercar la metodología corrotacional al lector, se presenta someramente una descripción simplificada, según lo propuesto por la bibliografía principal \citet{Le2014}. Una vez presentada dicha formulación, se despliega la metodología utilizada para esta investigación en el Capítulo \ref{Cap:Metodologia}. Aquí se detallan las hipótesis fundamentales del modelado estructural y de viento, explicándose las condiciones de borde impuestas y un análisis sobre el amortiguamiento aerodinámico considerado. En este mismo capítulo, se despliega la implementación del algoritmo numérico utilizado y las estructuras de pseudocódigo referentes a los principales scripts de la implementación computacional en el software \footnote{https://github.com/ONSAS/ONSAS/}\href{https://github.com/ONSAS/ONSAS/}{ONSAS}. 
 
 Posteriormente, se resuelven tres aplicaciones numéricas en el Capítulo \ref{Cap:ResultadosNumericos}. La primera de ellas persigue el objetivo de validar numericamente la implantación. Este ejemplo es un modelo clásico en la literatura corrotacional donde se observan resultados acordes en contraste con los presentados en \cite{Le2014}. De manera subsiguiente, se modela un ejemplo presentado por el autor Luca Foti en \cite{Foti2016}. Este consiste en un conductor eléctrico sometido a una carga artificial, extraída de un viento tipo capa límite atmosférica. Por último, se presenta un problema complejo de dos vanos consecutivos, compuesto por tres torres modeladas con elementos de barra tipo Green y seis conductores por elementos de viga corrotacional. El sistema de trasmisión eléctrica, con geometrías y propiedades reales, es atacado por un perfil de viento capturado durante una corriente descendente en el norte de Alemania por \cite{stengel2017measurements}. Finalmente en  \ref{Cap:Conlcusiones} se sintetizan los principales resultados enriquecedores de este trabajo, además de plasmarse eventuales trabajos a futuro, con lineamientos para profunidzar en la temática y sus posibles aplicaciones en el mercado de distribución eléctrica.