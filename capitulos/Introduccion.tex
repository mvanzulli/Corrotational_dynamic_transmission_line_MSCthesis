\chapter{Introducción}\label{Cap:Introduccion}
 
\section{Motivación}
\linenumbers
Los eventos climáticos extremos representan un desafío para el diseño de todo tipo de estructuras en todo el mundo. En Uruguay el choque de masas de aire caliente, originado en el trópico con corrientes de aires fríos que migran desde el polo produce una atmósfera inestable. Estos fenómenos atmosféricos son peligrosos, ya que producen vientos extremos no sinópticos sumamente violentos y destructivos. Un registro trágico de este tipo de eventos, sucedió el 10 de marzo del 2002, cuando una tormenta convectiva afectó un área de alrededor de $6500$ km$^2$ en el sur del país \cite{tormenta2002}. En el norte de Montevideo los anemómetros capturaron velocidades de ráfaga de $34$ m/s y de acuerdo con el nivel de daño causado, se estimaron que en ciertos puntos podría haber superado los $56$ m/s. Este valor es mayor a la velocidad de diseño establecida por la norma UNIT 50-84. Fue tal el nivel de devastación, que 19 torres de transmisión eléctrica de $500$ kV y 48 de $150$ kV colapsaron, además de unos 700 edificios y 1250 techos de hogares que fueron destruidos \citep{duranona2015significance}. Este tornado no solo afectó a las construcciones, sino también muchos productores rurales y sus estancias productivas, derribando invernaderos, montes y plantaciones. El costo de reparación asociado con las torres se estimó en 2 millones de dólares y en simultaneo se destinaron unos 10 millones de dólares a suplir la red con energía termoeléctrica, proveniente de combustibles fósiles. El presupuesto estimado de los daños en total ascendió a la suma de 27 millones de dólares según \cite{duranona2019first}. 

Las líneas de transmisión eléctrica son frecuentemente afectadas por eventos climáticos severos como \gls{CD} o tornados. Estos eventos pueden provocar la desconexión de las líneas, con consecuencias potencialmente graves. En el periodo 2000-2007 se registraron más de veinte eventos de salida en servicio por esta causa en una de las principales líneas de Uruguay (Palmar-Montevideo). Este tipo de fenómenos inducen fuertes movimientos en los cables, provocando un balanceo excesivo de los mismos. Estas amplitudes desmesuradas implican vulneraciones en la aislación del sistema, al aproximar sus cadenas aisladoras a las torres. Produciéndose descargas a tierra e indeseables interrupciones del suministro que han afectado a la capital durante varias horas. Una ilustración del fenómeno se muestra en la Figura \ref{fig:INTRO:IlusExcesiveBalance} para una torre de alta tensión ubicada en la ruta 5 del departamento de Canelones, Uruguay. 

\begin{figure}[htbp]
	\centering
	\def\svgwidth{80mm}
	\input{./imagenes/Introduccion/Torre.pdf_tex}
	\caption{Ilustración de balanceos excesivos torre Ruta 5.}
	\label{fig:INTRO:IlusExcesiveBalance}
\end{figure}  

%El modelado estructural de vientos severos sobre líneas de transmisión eléctrica, ha sido abordado por la comunidad científica internacional desde diversas ópticas, principalmente a lo largo de las últimas cuatro décadas. Se han presentado modelos semi-analíticos, análisis experimentales en túneles de viento y de campo, más recientemente utilizando métodos computacionales.

Esto plantea la necesidad de desarrollar más conocimiento a nivel local para el modelado computacional de este tipo de problemas de gran escala y complejidad. Este es el principal objetivo de este trabajo, profundizar en la bibliografía para el modelado estructural de conductores y crear un modelo robusto, consistente y capaz de simular líneas de transmisión eléctrica sometidas bajo la acción de vientos extremos.



\section{Enfoque}

Los autores de la literatura han acuñado sus investigaciones en diversos tipos de elementos. Utilizando elementos de barras se destacan los trabajos de: \cite{desai1995finite}, \cite{yan2009numerical}, \cite{gani2010dynamic}, \cite{yang2016nonlinear}. A pesar de la gran esbeltez de las líneas de transmisión eléctrica, las mismas cuentan con rigidez a flexión. Los elementos de barra no son capaces de representarla, por ende, es necesario incorporar elementos de vigas tridimensionales. Debido a los grandes desplazamientos y rotaciones que se presentan durante las trayectorias en tormentas, se consideró importante implementar una formulación corrotacional considerando la dinámica del problema.

El campo de la metodología corrotacional es muy amplio, pero debido a la claridad y contemporaneidad en el desarrollo de sus publicaciones, se tomó como principal referencia de la formulación a \cite{Le2014}. A e´sta se le agregaron componentes no lineales debido a la interacción del sólido en un fluido que ejerce determinadas fuerzas. Esta formulación se implementó computacionalmente a la herramienta \href{https://github.com/ONSAS/ONSAS.m}{ONSAS} presentado inicialmente por \cite{bruno2017introduccion} \footnote{https://github.com/ONSAS/ONSAS.m}. Este software de código abierto viene siendo desarrollado por el grupo de investigación: Modelado e Identificación en Sólidos y Estructuras integrado por docentes del Instituto de Mecánica y Producción Industrial (IIMPI) y por docentes del Instituto de Estructuras y Transporte (IET) de la Facultad de Ingeniería UdelaR e investigadores internacionales. Se desarrollaron tres modelos computacionales con grado de complejidad progresivo. El primero de ellos permitió validar los códigos implementados, el segundo acercar el modelo a la aplicación central de esta tesis y el último generar resultados sobre la respuesta de sistemas de transmisión eléctrica ante las fuerzas de vientos extremos.


\section{Estructura de la tesis}

Este documento consta de seis capítulos: Introducción, Estado del arte, Preliminares, Metodología, Resultados Numéricos y Conclusiones. Inicialmente en el Capítulo \ref{Cap:EstadoDelArte} se realiza un recorrido histórico en materia de simulaciones aplicadas a conductores eléctricos, con un enfoque computacional y semi-analítico. También se narran los diferentes estudios locales e internacionales sobre vientos extremos, para concluir en un recorrido dentro del abordaje corrotacional. Posteriormente en el Capítulo \ref{Cap:Preliminares}, con el objetivo de acercar la metodología corrotacional al lector, se presenta una descripción con foco conceptual, según lo propuesto por la bibliografía principal de \citet{Le2014}. Una vez presentada dicha formulación, se despliega la metodología utilizada para esta investigación en el Capítulo \ref{Cap:Metodologia}. Aquí se detallan las hipótesis fundamentales del modelado estructural y de viento, explicándose las condiciones de borde impuestas y un análisis sobre el amortiguamiento aerodinámico. En este mismo capítulo, se desarrolla la implementación del algoritmo numérico utilizado con la extensión de fuerzas viscosas y las estructuras de pseudocódigo referentes a los principales \textit{scripts} de la implementación computacional en el software \href{https://github.com/ONSAS/ONSAS/}{ONSAS}. 


Posteriormente, se resuelven tres aplicaciones numéricas en el Capítulo \ref{Cap:ResultadosNumericos}. La primera de ellas persigue el objetivo de validar numéricamente la implantación. De manera subsiguiente, se modela un ejemplo de un conductor eléctrico sometido a una carga artificial, extraída de un viento tipo \gls{CLA}. Por último, se presenta un problema realista de un sistema de transmisión eléctrica, con geometrías y propiedades reales, sometido por un perfil de viento capturado durante una CD en el norte de Alemania. Finalmente en el Capítulo \ref{Cap:Conlcusiones} se sintetizan los principales resultados enriquecedores de esta investigación, además de plasmarse eventuales trabajos a futuro, con lineamientos para profundizar en la temática y sus posibles aplicaciones en el mercado de distribución eléctrica.  
