\chapter{Introducción}
 
\subsection{Motivación}
\linenumbers


Debido a las condiciones climáticas especificas del territorio uruguayo. Se produce una atmósfera inestable provocada por el choque de masas de aire calientes, originadas en los trópicos, y corrientes de aires fríos que migran desde el polo. Esta eminente peligro produce vientos extremos no sinópticos sumamente violentos y destructivos. Un trágico evento se sucedió el 10 de marzo de 2002 cuando una tormenta convectiva afecto un área de alderredor 6500 km$^2$ en el sur del país\cite{DNM2002}. En el norte de Montevideo los anemómetros capturaron velocidades de ráfaga de 34 $m/s$ y de acuerdo con el nivel daño causado se, en determiandos puntos podría haber superado los 56 $m/s$. Fue tal el nivel de devastación que causó el colapso de 19 torres de trasmisión eléctrica de 500 kV y 48 de 150kV, además de unos 700 edificios y 1250 techos de hogares que fueron destruidos \citep{duranona2015significance}. No solo afecto a las construcciones sino también muchos productores rurales y sus estancias productivas derribando invernaderos, montes y plantaciones. El costo de reparación de las torres es estimo en 2 millones de dolares y en simultaneo se gastaron unos 10 millones de dolares destinados para suplir la red con energía geotérmica proveniente de combustibles fósiles . En total los daños fueron costeados con un presupuesto de unos 27 millones de dolares \cite{duranona2019first}.

Las líneas de trasmisión eléctrica son frecuentemente afectadas por eventos climáticos severos como corrientes descendentes o tornados. Estos eventos pueden provocar su desconexión, con consecuencias potencialmente graves. En el periodo 2000-2007 se registraron más de veinte eventos de salida en servicio por esta causa en una de las principales líneas de Uruguay (Palmar-Montevideo). Este tipo de fenómenos inducen fuertes movimientos en los cables, provocando un balanceo excesivo de los mismos. Estas amplitudes desmesuradas implican vulneraciones en la aislación del sistema, al aproximar sus cadenas de aisladores a las torres. Produciéndose descargas a tierra e indeseables interrupciones del suministro que han afectado a la capital durante varias horas. El modelado estructural de vientos severos sobre líneas de transmisión eléctrica ha sido abordado por la comunidad científica internacional desde diversas ópticas a lo largo de las últimas cuatro décadas. Se han presentado modelos semi-analíticos, análisis experimentales en túneles de viento y en campo más recientemente en modelos numéricos.
 
Esto plantea la necesidad de contar con herramientas complementarias que sean capaces de emular la respuesta de estos sistemas ante perfiles de viento no sinópticos. Este es el principal objetivo de este trabajo, profundizar en la bibliografía para el modelado estructural de conductores y crear un modelo robusto, consistente capaz de simular lineas de trasmisión eléctrica ante vientos los medidos experimentalmente en \cite{stengel2017measurements}. 
  

\subsection{Enfoque}
Numerosos autores del literatura han acuñado sus investigaciones en elementos multinodales de barras \cite{desai1995finite}Para el modelado estructural del cable se condideran modelas de vigas flexibles se utilizan en vastos campos de la ingeniería, entre otras: aeronaves, turbinas propulsoras, molinos eólicos marítimos y terrestres. Su comportamiento es usualmente modelado utilizando no linealidad geométrica para grandes desplazamientos. A pesar de las formulaciones `` Updated " (UL) y ``Total Lagrangian" (TL) clásicas, son capaces de representar correctamente movimientos de gran amplitud. Estudios contemporáneos \citet{albino2018co}\citet{asadi2019multibody}  muestran que la metodología es idónea para la aplicación en diversas áreas. No obstante, según el conocimiento del autor, ningún software comercial hasta la fecha utiliza formulaciones corrotacionales para la solución de problemas dinámicos. 
\subsection{Estructura}