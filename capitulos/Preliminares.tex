\chapter{Preliminares}\label{Cap:Preliminares}
\linenumbers

A continuación se presenta una descripción cualitativa y cuantitativa de la formulación corrotacional según lo propuesto en \citep{Le2014}. La temática se abordara progresivamente según la naturaleza de las variables. En primera instancia se describen la caracterización de magnitudes cinemáticas globales y locales en las Secciones \ref{Subsec:PRE:CienmaticCorrot} y \ref{Sec:PRE:LocalFormul}. Una vez ahondadas las variables asociadas al movimiento se expone como, a partir de estas, se deducen las variables estáticas y dinámicas en la Sección \ref{Subsec:PRE:DinamicCorrot}.


\section{Cinemática corrotacional}\label{Subsec:PRE:CienmaticCorrot}

El planteo corrotacional para elementos de viga 3D binodales, se basa en escindir la cinemática del movimiento en dos componentes. La primera de ellas representa grandes rotaciones y desplazamientos dados por la dinámica de un elemento rígido. La segunda componente tiene en cuenta los desplazamientos locales asociados a la flexibilidad del material. Este enfoque suele aplicarse al analizar deformaciones estáticas. Resulta intuitivo imaginar en un inicio como se deformaría la estructura de manera rígida para luego aplicarle la componente no rígida. Ahora bien, en este tipo de formulaciones, hace falta introducir una serie de sistemas de coordenadas que permiten representar los desplazamientos de cada una de las componentes.

Para el abordaje de este análisis debe comprenderse una serie de rotaciones consecutivas ilustradas en la Figura \ref{fig:PRE:IlusCorrotRot}. Para un elemento formado por los nodos 1 y 2 en sus extremos, se distinguen tres configuraciones. La primera de ellas en color azul representa el elemento en su estado indeformado o de referencia. El color naranja identifica a la componente de deformación no rígida mientras que en gris se ilustra la configuración de deformación rígida del elemento.

Para realizar traspasos de una componente a otra se definen una serie de transformaciones. La primera de ellas nominada \gls{R0} lleva al elemento desde su estado paramétrico a su estado de referencia. A partir de esa configuración podemos hallar la geometría deformada aplicando las transformaciones \gls{R1g} o \gls{R2g}, dependiendo el nodo de interés. Esta no es la única forma de hallar el estado deformado del elemento a partir de su configuración de referencia. Una alternativa consiste dado un nodo $i$ al interior del elemento, aplicar consecutivamente las transformaciones \gls{Rr} y \gls{Rroof} encontrando así el estado deformado partiendo desde su configuración de referencia.

\begin{figure}[htbp]
	\centering
	\def\svgwidth{100mm}
	\input{./imagenes/Preliminares/Corrotacional/IlusCorrotacional2.pdf_tex}
	\caption{Rotaciones a cada configuración.}
	\label{fig:PRE:IlusCorrotRot}
\end{figure}

A partir de las definiciones descritas anteriormente e ilustradas en la Figura \ref{fig:PRE:IlusCorrotRot}, resulta clarificante destacar los argumentos sobre la nomenclatura seleccionada. En primer lugar, la notación con supra- indice ``g'' refiere a la palabra globales. Es ilustrativo referirse de esta forma a dicha transformación, ya que permite encontrar de forma ``macro'' cuales es la configuración deformada partiendo del sistema de coordenadas isoparamétrico. Asimismo en la Figura \ref{fig:PRE:IlusCorrotRot}, tanto las rotaciones locales \gls{Rroof1}, \gls{Rroof2} como globales $\bf{R}_i^g$ se utiliza el sub-indice $i$ mientras que para la rotación de deformación rígida no hace falta esta distinción. Este detalle resulta clave para comprender la metodología corrotacional. Dado que componente de deformación rígida es rectilínea, la orientación de cada nodo es idéntica por lo que es posible prescindir del sub-indice $i$.

Naturalmente para encontrar la curva deformada que describe el elemento, hace falta la orientación y traslación de un sistema de coordenadas solidario a cada punto. Estas transformaciones se pueden representar matemáticamente con la artillería del álgebra matricial para rotaciones. Una presentación de la temática puede hallarse en la publicación \citep{kovzar1995finite}.

En los párrafos que prosiguen se desarrollan los sistemas solidarios a los nodos ubicados en los extremos del elemento. El estudio de deformaciones locales para los puntos interiores a la viga se detalla en la Sección \ref{Sec:PRE:LocalFormul}.

Para deducir las matrices asociadas a cada transformación resulta imprescindible definir un conjunto de bases que permitan seguir al elemento en cada configuración. Estas tríadas de versores se muestran gráficamente a continuación en la Figura \ref{fig:PRE:IlusCorrot}.

\begin{figure}[htbp]
	\centering
	\def\svgwidth{100mm}
	\input{./imagenes/Preliminares/Corrotacional/IlusCorrotacional.pdf_tex}
	\caption{Descripción de las bases corrotacionales.}
	\label{fig:PRE:IlusCorrot}
\end{figure}


Primeramente se define un sistema de referencia auxiliar integrado por la base ortogonal (\gls{E1},\gls{E2},\gls{E3}). Una vez ubicado el elemento en su estado inicial, las coordenadas se hallan en relación a tres vectores (\gls{e1},\gls{e2},\gls{e3}). Al aplicarle la traslación y rotación de cuerpo rígido la base (\gls{r1},\gls{r2},\gls{r3}) se anida al elemento y funciona como sistema de coordenadas en la configuración de deformación rígida. Por último, la base (\gls{t1i},\gls{t2i},\gls{t3i}) permite identificar la orientación y posición del nodo $i$ en la configuración deformada. Se hace énfasis en el hecho de que tanto la configuración inicial como la de deformación rígida requieren un único sistema de coordenadas. Por el contrario, la configuración deformada debido a la flexibilidad del elemento, requiere dos sistemas, denotados con la letra $\bf{t}_j^i$ donde el supra-indice $i$ identifica el nodo y el sub-indice $j$ la dirección.


La definición de las bases mencionadas en el párrafo anterior no es arbitraria. Una vez definidas las matrices de rotación resulta intuitivo y oportuno escribirlas a partir de los vectores solidarios a cada configuración. Esa relación intrínseca entre matrices y los versores se establece en la Tabla \ref{Table:PRE:RelacionVM} a continuación:

\begin{table}[htbp]
	\begin{center}
		\begin{tabular}{|c|c|}
			\hline
			Matriz & Vínculo de bases \\
			\hline \hline
			$\bf{R}_0$ &$(\bf{E_1},\bf{E_2},\bf{E_3})$ $\rightarrow$
			$(\bf{e_1},\bf{e_2},\bf{e_3})$   \\ \hline
			$\bf{R}_i^g$ & $(\bf{e_1},\bf{e_2},\bf{e_3})$ $\rightarrow$
			$(\bf{t_1^i},\bf{t_2^i},\bf{t_3^i})$ \\ \hline
			$\bf{\overline{R}}_i$ &
			$(\bf{r_1},\bf{r_2},\bf{r_3})$$\rightarrow$$(\bf{t_1^i},\bf{t_2^i},\bf{t_3^i})$
			\\ \hline
			$\bf{R}_r$ &
			$(\bf{E_1},\bf{E_2},\bf{E_3})$$\rightarrow$$(\bf{r_1},\bf{r_2},\bf{r_3})$ \\
			\hline
		\end{tabular}
		\caption{Caracterización de matrices en términos de la base.}
		\label{Table:PRE:RelacionVM}
	\end{center}
\end{table}


Los vínculos descritos en la tabla anterior se desprenden de las definiciones para cada matriz. Los vectores a la izquierda y derecha hacen referencia a la y a su respectiva imagen. A modo de ejemplo para la primer fila se tiene: $\bf{R}_0$. $(\bf{E_1},\bf{E_2},\bf{E_3})^T$ = $(\bf{e_1},\bf{e_2},\bf{e_3})$. Al plantear este tipo de vínculos entre vectores y haciendo uso de la propiedad para matrices ortonnormales de la Ecuación \ref{eq:PRE:PropOrto} es posible deducir las Expresiones \eqref{eq:PRE:Vincul1} y \eqref{eq:PRE:Vincul2}.
%
%
\begin{eqnarray}
		\label{eq:PRE:PropOrto}
		\bf{R}^T&=&\bf{R}^{-1}\\
		\label{eq:PRE:Vincul1}
		\bar{\bf{R_i}}&= &(\bf{R_r^g})^T\bf{R_i^g}\bf{R_o}\\
		\label{eq:PRE:Vincul2}
		\bf{R_i^g}\bf{R_o} &=& \bf{R_r^g}\overline{\bf{R_i}}
	\end{eqnarray}


El propósito de la descripción anterior, algo intrincada y engorrosa responde a la necesidad de crear herramientas analíticas que permitan vincular los
desplazamientos lineales y angulares, para las distintas configuraciones. Dado un punto arbitrario P, es posible ubicarlo en coordenadas locales y
globales tal cual se muestra en la Figura \ref{fig:PRE:IlusCorrotDisps}. En coordenadas locales sus grados de libertad son: el desplazamiento axial, etiquetado con la letra \gls{uP}, y sus desplazamientos angulares con el nombre \gls{thetaP}. Los siete grados de libertad se compactan en el vector \gls{DispLocal}$=(\bf{u_P},\bf{\overline{\theta_i^P}})$. Ahora bien, es posible desglosar el desplazamiento axial \gls{uP} en tres componentes según los vectores $\bf{r_i}$. Al vector desplazamientos de P en función de la base  $\bf{r_i}$ se le denomina \gls{dispAxialLocalRigid}.

Los desplazamientos de la viga en el punto P también se pueden expresar en coordenadas globales.  Para esto se utilizan las 6 magnitudes clásicas
\gls{GlobalDisp}$=($\gls{GlobalDispU},\gls{GlobalDispW}$)$. Esta tienen origen en la configuración de referencia o material hasta la deformada como se muestra en la Figura
\ref{fig:PRE:IlusCorrotDisps}.

\begin{figure}[htbp]
	\centering
	\def\svgwidth{100mm}
	\input{./imagenes/Preliminares/Corrotacional/IlusDisp.pdf_tex}
	\caption{Desplazamientos locales y globales del nodo P.}
	\label{fig:PRE:IlusCorrotDisps}
\end{figure}

Acorde con los desplazamientos presentados anteriormente, es propicio calcular sus diferenciales asociados. Estos emplearan un rol esencial para el cálculo de matrices tangentes y fuerzas internas. A continuación las Ecuaciones \eqref{eq:PRE:DifDisps1} y \eqref{eq:PRE:DifDisps2}
definen las variaciones de los desplazamientos locales y globales respectivamente.

\begin{eqnarray}\label{eq:PRE:DifDisps1}
		\bf{\delta d_l} &=& [\delta\bar{u}, \bf{\delta\overline{ \theta _1}^T},	\bf{\delta \overline{ \theta _2}^T}]^T\\
	\label{eq:PRE:DifDisps2}
	\bf{\delta d_g} &=& [\bf{\delta u_1^g}^T, \bf{\delta u_2^g}^T, \bf{{w_1^g}^T}, \bf{{w_2^g}^T}]^T
\end{eqnarray}

Consecuente con los desplazamientos infinitesimales, se desarrollan los diferenciales asociados a las transformaciones de giro $\bf{R_r^g}$, $\bf{R_i^g}$, $\bf{R_0}$ y $\bf{\overline{R}}_i$.
Para esto, primeramente deben obtenerse las matrices según lo explicitado en la Tabla \ref{Table:PRE:RelacionVM}. Las entradas de $\bf{R}_r$ y  $\bf{R}_i^g$ se hallan siguiendo las Ecuaciones \eqref{eq:PRE:Rr} y \eqref{eq:PRE:Rg} a continuación:

\begin{eqnarray}
	\label{eq:PRE:Rr}
	\bf{R_r}&=&[\bf{r_1} ~ \bf{r_2} ~ \bf{r_3}]\\
	\label{eq:PRE:Rg}
	\bf{R}_i^g&=&[\bf{t_1} ~ \bf{t_2} ~ \bf{t_3}]
\end{eqnarray}

 Los versores $\bf{r_i}$  se hallan a partir del vector director $\bf{r_1}$ que apunta del nodo 1 al 2. Es por esto que es preciso definirlo en función de las posiciones iniciales de los nodos en coordenadas globales \gls{CoordX1} y \gls{CoordX2}, sus desplazamientos $\bf{u}_1^g$ y $\bf{u}_2^g$ y el largo \gls{LargoLn} una vez deformado.

\begin{eqnarray}
	l_n&=& ||\bf{X}_2+\bf{u}_2-\bf{X}_1-\bf{u}_1||\\
	\bf{r_1}&=&\frac{\bf{x}_2+\bf{u}_2-\bf{x}_1-\bf{u}_1}{l_n}
\end{eqnarray}

El vector auxiliar $\bf{p}$ surge se define para hallar primeramente los vectores $\bf{r}_i$ y partir de estos la base $\bf{t}_i$. Estos versores son dinámicos y solidarios al movimiento. Están unidas a la configuración de deformación rígida y local respectivamente. El constante cambio de estas configuraciones en cada iteración, conduce a la necesidad de expresarlos en función de vectores asistentes.  Para esto se definen
$\bf{p}$, $\bf{p_1}$ y $\bf{p_2}$ en la Ecuación \eqref{Eqn:Corrot:DefAuxp}:

\begin{equation}\label{Eqn:Corrot:DefAuxp}
	\bf{p}=\frac{1}{2}(\bf{p}_1+\bf{p}_2),~~~~~~\bf{p_i}=\bf{R}_i^g\bf{R}_0[0~1~0]^T
\end{equation}

En la expresión anterior la matriz $\bf{R}_0$ se obtiene colgando los vectores $\bf{e}_i$ escritos como combinación lineal de la base $\bf{E_i}$. Una vez calculada esta matriz y evaluado las expresiones de la Ecuación \eqref{Eqn:Corrot:DefAuxp} se obtienen los restantes versores directores
de la componente de deformación rígida. Esto es:


\begin{equation}\label{Eqn:Corrot:VectorsR}
	\bf{r}_3=\frac{\bf{r_1}~x~\bf{p}}{||\bf{r_1}~x~\bf{p}||},~~~~~~\bf{r_2}=\bf{r_3}~x~\bf{r_1}
\end{equation}


Habiendo definido las matrices de rotación  es útil calcular las variaciones de las mismas. Estos cálculos son fundamentales para la transformación de variables y sus respectivos diferenciales.

\begin{equation}\label{eq:PRE:DifMatrix}
	\delta \overline{\bf{R_i}}=\delta\bf{R_r}^T\bf{R_i^g}\bf{R_0}+\bf{R_r}^T\delta \bf{R_i^g}\bf{R_0}
\end{equation}

En la Ecuación \eqref{eq:PRE:DifMatrix} se aplica la regla de la cadena para el cálculo de diferenciales matriciales. Dado que transformación $\bf{R_0}$ comunica la configuración indeformada y ambas configuraciones son fijas, su matriz es constante. Por lo tanto, su variación es nula. A diferencia de las matrices de giro $\overline{\bf{R_i}}$ y $ \bf{R_i^g}$ sus variaciones pueden hallarse según las Ecuaciones \eqref{eq:PRE:DifMatrix2} y \eqref{eq:PRE:DifMatrix3} respectivamente.


\begin{eqnarray}
	\label{eq:PRE:DifMatrix2}
	\delta \bf{R_i^g} &=& \widetilde{\delta\bf{w}_i^g}~\bf{R}_i^g\\
	\label{eq:PRE:DifMatrix3}
	\delta \bf{R_r^g} &=& \widetilde{\delta\bf{w}_r^g}~\bf{R}_r
\end{eqnarray}

En la ecuación \eqref{eq:PRE:DifMatrix3} el término $\widetilde{\delta\bf{w}_r^g}$ refiere a la operación skew del vector de ángulos de la componente de deformación rígida. Esta operación simplifica el producto vectorial de forma matricial y es sumamente útil para el cálculo de diferenciales asociados a matrices de rotación. La función \gls{Skew} aplicada al vector $\bf{\Omega}=(\Omega_1,\Omega_2,\Omega_3)$ toma la siguiente forma:

\begin{equation}\label{eq:PRE:Skew}
	\text{Skew}(\bf{\Omega})=\widetilde{\bf{\Omega}}
	=
	\begin{bmatrix}
		0 &-\Omega_3  &\Omega_2   \\
		\Omega_3&0  & -\Omega_1  \\
		-\Omega_2  & \Omega_1 & 0
	\end{bmatrix}
\end{equation}

En función de lo descrito anteriormente resta vincular los diferenciales de ángulos locales en términos de las variaciones globales. Para esto se definen las matrices $\bf{E} $y $\bf{G}$ según las Ecuaciones \eqref{Eqn:PRE:Corrot:DefE} \eqref{Eqn:PRE:Corrot:DefG}.

\begin{equation}\label{Eqn:PRE:Corrot:DefE}
	\bf{E}=\begin{bmatrix}
		\bf{R_r}& \bf{0}   & \bf{0}   & \bf{0} \\
		\bf{0}  & \bf{R_r} & \bf{0}   & \bf{0}\\
		\bf{0}  & \bf{0}   & \bf{R_r} & \bf{0} \\
		\bf{0}  & \bf{0}   & \bf{0}   & \bf{R_r}
	\end{bmatrix}\rightarrow \delta \bf{d_g}=E^T \bf{d_g}
\end{equation}

Notoese que las matrices $\bf{R}_r$ tiene dimensión 3x3. Para respetar dichas dimensiones, $\bf{0}$ es una matriz nula de 3x3 e $\bf{I}$ una matriz identidad del mismo número de filas y columnas. De forma subsiguiente $\bf{E}$ posee 12 entrada en filas y columnas asociadas a los 12 grados de libertad por elemento.

\begin{equation}\label{Eqn:PRE:Corrot:DefG}
	\begin{array}{r@{}l}
		\bf{G}&{}=\frac{\partial \bf{w_r^g}}{\partial \bf{d}^g}\\
		\bf{G}(1:6)&{}=\begin{bmatrix}
			0 &  0      &  \eta/l_n &  \eta_{12}/2  &-\eta_{11}/2  &  0  \\
			0 &  0      &   1/l_n   &       0       &      0       &  0   \\
			0 & -1/l_n  &      0    &       0       &      0       &  0
		\end{bmatrix}\\
		\bf{G}(7:12)&{}=\begin{bmatrix}
			0  &  0    &   -1/l_n   &      0      &     0        &    0 \\
			0  &  0     &-\eta/l_n  & \eta_{22}/2 &-\eta_{21}/2  &    0 \\
			0  &  1/l_n &       0   &      0      &     0        &    0
		\end{bmatrix}
	\end{array}
\end{equation}


En la columna 1 y 12 de la matriz $\bf{G}$ las entradas son nulas ya que los desplazamiento angulares globales no dependen de los estiramientos axiales de los nodos. Además, los parámetros $\eta$ se calculan realizando
los cocientes entre las componentes de los vectores $\bf{p}_j$ y $\bf{p_{ij}}$ según la Ecuación \eqref{Eqn:RPE:VectoresP}. Siendo el vector $p_j$ el producto $\bf{R_r}^T\bf{p}$ y $\bf{p_{ij}}$ la multiplicación de $\bf{R_r}^T\bf{p}_i$.

\begin{equation}\label{Eqn:RPE:VectoresP}
	\eta = \frac{p_1}{p_2}, ~~\eta_{11} = \frac{p_{11}}{p_2}, ~~~~\eta_{12} = \frac{p_{12}}{p_2}, ~~~~\eta_{21} = \frac{p_{21}}{p_2}, ~~~~\eta_{22} = \frac{p_{2}}{p_2},
\end{equation}


La relación entre los diferenciales anteriores, se pueden combinar de manera matricial, logrando así expresar los incrementos de ángulos locales en términos globales. Tal cual se expresa en la Ecuaciones \eqref{Eqn:RPE:IncrementosAngulos} donde la matriz $\bf{P}$ queda definida. Esto es de sumo interés ya que para el cálculo de fuerzas internas las variables causa y efecto de su generación son los desplazamientos locales. Por ende resulta imprescindible calcular su variación en términos globales.


\begin{equation}\label{Eqn:RPE:IncrementosAngulos}
	\begin{bmatrix}
		\bf{\delta\overline{\theta_1}}\\
		\bf{\delta\overline{\theta_2}}
	\end{bmatrix}=\left ( \begin{bmatrix}
		\bf{0} &\bf{I}  & \bf{0} &\bf{0} \\
		\bf{0}&\bf{0}  &\bf{0}  & \bf{I}
	\end{bmatrix}-\begin{bmatrix}
		\bf{G}^T\\
		\bf{G}^T
	\end{bmatrix} \right )\bf{E}^T \delta \bf{d_g}=\bf{P}\bf{E}^T \delta \bf{d_g}
\end{equation}

Análogamente se debe transcribir la fuerza axial en función de las coordenadas globales. Con este objetivo se define un versor auxiliar  $ \bf{r}$ que vincula los incrementos del desplazamiento axial $\delta \overline{u}$ con los globales. Esto permite escribir la Ecuación \eqref{eq:PRE:DifDisps1} en relación a  \eqref{eq:PRE:DifDisps2} haciendo uso de la expresión que prosigue \eqref{eq:PRE:DefincionR}

\begin{equation}\label{eq:PRE:DefincionR}
\delta \overline{u} = \bf{r}~ d_g ~~~~~~\bf{r} = [ -\bf{r}_1^T~ \bf{0}_{1,3}~ \bf{r}_1^T~ \bf{0}_{1,3}  ]
\end{equation}

\section{Formulación local}\label{Sec:PRE:LocalFormul}
La fundamental ventaja y atractivo de la formulación corrotacional es su versatilidad ante diferentes tipos de elementos. Esto se debe al desacoplamiento analítico en la caracterización de los desplazamientos locales y globales. En este apartado. se detallan las magnitudes cinemáticas en la configuración local para el cálculo de los vectores y matrices dinámicas de la Sección \ref{Subsec:PRE:DinamicCorrot}.

El movimiento local de una sección ubicada a una distancia \gls{CentroideX} de la viga, desde su configuración inicial, se define a partir de la rotación y traslación de la sección correspondiente a su centroide \gls{CentroideG}. Una ilustración de esto se muestra en la Figura \ref{fig:PRE:IlusLocalDisp}, donde la configuración de deformación rígida se identifica en punteado y la deformada en color naranja.



\begin{figure}[htbp]
	\centering
	\def\svgwidth{100mm}
	\input{./imagenes/Preliminares/Corrotacional/IlusLocalDisp.pdf_tex}
	\caption{Esquema de desplazamientos locales.}
	\label{fig:PRE:IlusLocalDisp}
\end{figure}

\begin{figure}[htbp]
	\centering
	\def\svgwidth{100mm}
 	\input{./imagenes/Preliminares/Corrotacional/IlusLocalAng.pdf_tex}
	\caption{Ilustración grados de libertad locales.}
	\label{fig:PRE:IlusLocalAng}
\end{figure}



%\begingroup
%\begin{figure}[htbp]
%	\centering
%	\subfigure[Esquema de desplazamientos locales ]{	\def\svgwidth{70mm}
%		\input{./imagenes/Preliminares/Corrotacional/IlusLocalDisp.pdf_tex}}\label{fig:PRE:IlusLocalDisp}
%	\subfigure[Esquema de angulos locales ]{	\def\svgwidth{70mm}
%		\input{./imagenes/Preliminares/Corrotacional/IlusLocalAng.pdf_tex}}\label{fig:PRE:IlusLocalAng}
%	\caption{Ilustración grados de libertad locales} 	\label{fig:PRE:IlusLocal}
%\end{figure}
%\endgroup

El movimiento de la base $\bf{t_i}$ en respecto del sistema $\bf{r_i^G}$ esta dado por los desplazamientos  $\bar{u}_3$ según el versor  $\bf{r_3^G}$ y análogamente para los vectores $\bar{u}_2$ y $\bar{u}_1$. Esto determina la ubicación del baricientro G. Su orientación se define a partir del plano punteado en color negro. La rotación de este respecto de tres ejes esta dada por el plano en naranja. Este se define por dos vectores $\bf{t_3^G}$ y $\bf{t_2^G}$ dentro del plano y un versor perpendicular $\bf{t_1^G}$. La transformación $\bf{\overline{R}}_G$ permite encontrar  los transformados de la base $\bf{r_i^G}$ etiquetados con las letras $\bf{t_i^G}$. Por último se observa el desplazamiento axial de la barra $\bar{u}$ correspondiente al del nodo 2 en la dirección $\bf{r_1}$.

Las interpolaciones para los puntos interiores al elemento se basan en las hipótesis de Bernoulli. Consecuentemente las interpolaciones son lineales para los desplazamientos axiales $\bar{u}_1$ y para los ángulo de torsión $\bf{\theta_1}$. Por la contraria, tanto para los desplazamientos transversales $\bar{u}_2$ y $\bar{u}_3$  como para los ángulos de flexión, las interpolaciones es través de polinomios cúbicos. Estas funciones interpolantes se detallan en las Ecuaciones \eqref{Eqn:PRE:FuncInterpol1}, \eqref{Eqn:PRE:FuncInterpol2} y \eqref{Eqn:PRE:FuncInterpol3}.

\begin{eqnarray}
		\label{Eqn:PRE:FuncInterpol1}
 		N_1 = 1 - \frac{x}{l_0},   		&~~~~& 	N_2= \frac{x}{l_0}\\
 		\label{Eqn:PRE:FuncInterpol2}
 		N_3 = x\left(1 - \frac{x}{l_0}\right)^2 	&~~~~&  N_4 - \left( 1 - \frac{x}{l_0} \right ) \frac{x^2}{l_0} \\
 		\label{Eqn:PRE:FuncInterpol3}
 		N_5 = \left(1 - \frac{3x}{l_0}\right) \left(1 - \frac{x}{l_0}\right) 	&~~~~&  N_6 =\left( \frac{3x}{l_0}-2\right) \left(\frac{x}{l_0}\right)
\end{eqnarray}

Para un punto ubicado a una distancia $x$ del nodo 1 según el vector $\bf{r_1}$ es posible calcular los desplazamientos locales en la base $\bf{r_i}$. Dado el punto arbitrario G que se desplazo en el sistemas de coordenadas locales según el vector $\bf{d_l}^G$. Los valores en términos de la componente de deformación rígida $\bf{r_i}$ se calculan aplicando la Ecuación \ref{Eqn:PRE:LocalDispsTransform}.

\begin{equation}\label{Eqn:PRE:LocalDispsTransform}
	\begin{bmatrix}
		\bar{u}_1^G\\
		\bar{u}_2^G\\
		\bar{u}_3^G\\
		 \bar{\theta}_1^G\\
		 \bar{\theta}_2^G\\
		 \bar{\theta}_3^G\\
	\end{bmatrix}
=
\begin{bmatrix}
	N_2 &0  & 0 & 0 & 0 & 0 & 0   \\
	0 & 0 & 0 & N_3 & 0 &0  &N_4  \\
	0& 0 &-N_3  & 0 &0  & -N_4 &0 \\
	0 & N_1 & 0 & 0 &N_2  & 0 &0   \\
	0&0  & N_5 & 0 & 0 & N_6 &0   \\
	0 & 0 & 0 & N_5 & 0 & 0 & N_6   \\
\end{bmatrix} \bf{d_l^G}
\end{equation}

Debido a que la matriz anterior presenta una gran cantidad de entradas nulas es útil agrupar las funciones de interpolaciones en matrices más pequeñas. De esta forma se construyen las matrices $\bf{P_1}$ y $\bf{P_2}$.  Estas expresan los desplazamientos transversales  $\bar{u}_2, \bar{u}_3 $  como también los ángulos $\bar{\theta}_1^G$  y $\bar{\theta}_2^G$ y $\bar{\theta}_3^G$ según los desplazamientos lineales del baricentro y los ángulos locales  $\overline{\bf{\theta_1}}$ y  $\overline{\bf{\theta_2}}$ para el nodo 1 y 2 respectivamente. Esta artimaña analítica se expresa a continuación en las Ecuaciones \eqref{Eqn:PRE:CompactTransveralDisp} y \eqref{Eqn:PRE:CompactTransversalAngs}:


 \begin{eqnarray}
 	\label{Eqn:PRE:CompactTransveralDisp}
		\begin{bmatrix}
			0     \\
		\bar{u}_2^G \\
		\bar{u}_3^G
	\end{bmatrix}=\bf{u_l}=\bf{P}_1
	\begin{bmatrix}
			\overline{\bf{\theta_1}}\\
			\overline{\bf{\theta_2}}
	\end{bmatrix} &~~~~~ &
	\bf{P}_1 = \begin{bmatrix}
		 0 &   0  & 0 & 0  & 0 & 0 \\
		 0  &  0 & N_3 & 0 & 0 & N_4 \\
		 0  &  -N_3  & 0 & 0 & -N_4 & 0
	\end{bmatrix} \\
	\label{Eqn:PRE:CompactTransversalAngs}
		\begin{bmatrix}
			\bar{\theta}_1^G\\
			\bar{\theta}_2^G \\
			\bar{\theta}_3^G
		\end{bmatrix}=\bf{\theta_l}=\bf{P}_2
		\begin{bmatrix}
			\overline{\bf{\theta_1}}\\
			\overline{\bf{\theta_2}}
		\end{bmatrix}& ~~~~~ &\bf{P}_2=
	\begin{bmatrix}
	N_1 & 0 & 0 & N_2 & 0 & 0 \\
	0  & N_5 & 0 & 0 &N_6 &0 \\
	0  &  0  & N_5 & 0 & 0 &N_6
	\end{bmatrix}
 \end{eqnarray}

Las hipótesis de Bernoulli desprecian las deformaciones por fuerzas cortantes, esto se refleja en sus polinomios de interpolación. Esta premisa no tiene perjucios sobre la aplicación con la que se modelará el elemento. La estructura de cables es extremadamente esbelta, con relaciones de diámetro respecto a largo ínfimas. Por la tanto, las deformaciones por cortante son efectivamente despreciables respecto a las inducidas por los momentos flectores.

\subsection{Variaciones en desplazamientos} \label{Sec:PRE:VariacionesDesplazamientos}
Ya se ha remarcado en reiteradas ocasiones la importancia de los desplazamientos diferenciales para el desarrollo de matrices tangentes y fuerzas. Antes de introducir al lector en la siguiente Sección, es preciso realizar una descripción previa para el cálculo de variaciones. En función de la Figura \ref{fig:PRE:IlusLocalDisp} queda definida la ubicación del baricentro OG partiendo desde el nodo 1. Esto se expresa en según la siguiente ecuación con notación simplificada:

\begin{equation}\label{Eqn:PRE:DispsOG}
 	\text{OG} = \bf{x}_1^g+\bf{u}_1^g + (x+\text{$\bar{u}_1$}) \bf{r_1} + (\text{$\bar{u}_2$})\bf{r_2}+ (\text{$\bar{u}_3$})\bf{r_3}
\end{equation}

Sustituyendo los polinomios interpolantes anteriormente definidos en \eqref{Eqn:PRE:DispsOG} y haciendo uso la matriz auxiliar $\bf{N}$ es posible escribir los desplazamientos del baricentro y su diferencial asociado.

\begin{eqnarray}
	\bf{N}&=&[\text{$N_1$}~~ \bf{I}~~ \bf{0}~~ \text{$N_2$}~~ \bf{I}~~ \bf{0}]\\
	OG    &=& \text{$N_1$}(\bf{x}_1^g+\bf{u}_1^g)+\text{$N_2$}(\bf{x}_2^g+\bf{u}_2^g)+ \bf{R}_r \bf{u}_l \\ \label{Eq:PRE:DifOG}
	\delta OG = \delta \bf{u} &= &\bf{N}\delta\bf{d}_g+\bf{R}_r \delta\bf{u}_l + \delta \bf{R}_r \bf{u}_l
\end{eqnarray}

 La expresión presentada \eqref{Eq:PRE:DifOG} depende de los desplazamientos locales. Esto dificulta el cálculo de su magnitud, ya que esos grados de libertad se encuentran solidarios a sistemas de coordenadas móviles. Para solucionar este problema, se sustituyen las Ecuaciones \eqref{Eqn:PRE:Corrot:DefE}, \eqref{Eqn:PRE:Corrot:DefG}, \eqref{Eqn:RPE:IncrementosAngulos} y \eqref{eq:PRE:DifMatrix2} lográndose de este modo, escribir a $ \delta \bf{u}$ en coordenadas globales. Además se compacta la notación definiendo la matriz $\bf{H}_1$ según la Ecuación \eqref{Eqn:PRE:DifUDefH}.

 \begin{equation}\label{Eqn:PRE:DifUDefH}
	\delta \bf{u} = \bf{R}_r(\bf{N}+\bf{P}_1\bf{P}-\widetilde{\bf{u}_l}\bf{G}^T)\bf{E}^T \delta \bf{d}_g = \bf{R}_r \bf{H}_1 \bf{E}^T \delta \bf{d}_g
 \end{equation}

 Para deducir la igualdad anterior se asumió que los incrementos angulares de las componentes locales, definidas en la Ecuación \eqref{eq:PRE:DifDisps1}, son despreciables frente a los de la componente de deformación rígida. Para el autor \cite{Le2014}, degbido a sus cambios de magnitud entre miseraciones, no hay diferencias asociadas a los incrementos de ángulos locales y rígidos. Esto es: ($\delta \overline{\bf{\theta}_{ri}}= \overline{\delta \bf{w}_i}$ ).

 Un procedimiento similar se aplicará en los siguientes párrafos a las magnitudes angulares. Consecuentemente el diferencial rotación del centro de masa se puede calcular en función de los desplazamientos nodales globales según se establece en la Ecuación

 \begin{equation}\label{Eqn:PRE:AngularBaricentro}
 	\delta \bf{w^g} (OG) =\delta \bf{w} =\bf{R}_r (\bf{P}_2\bf{P}+\bf{G}^T)\bf{E}^T\delta \bf{d}_g = \bf{R}_r\bf{H}_2\bf{E}^T \delta \bf{d}_g
 \end{equation}



\subsection{Velocidades y aceleraciones}\label{Sec:PRE:VelAc}

Las magnitudes dinámicas despeñan un papel primordial en el análisis implementado. Tanto velocidades como aceleraciones deben ser calculadas en términos globales. De igual modo, que en la Sección \ref{Sec:PRE:VariacionesDesplazamientos}, se obtienen sus diferenciales asociados. Derivando respecto al tiempo la Ecuación \eqref{Eqn:PRE:DifUDefH} se deducen las velocidades lineal \gls{VeloicdadLineal} según la Expresión \eqref{Eqn:PRE:Udot}. Al aplicar la regla del producto en \eqref{Eqn:PRE:Udot} se halla la aceleración lineal \gls{AceleracionLineal} del centro de masa del elemento en \eqref{Eqn:PRE:Udotdot}.

 \begin{eqnarray}
 	\label{Eqn:PRE:Udot}
	\dot{\bf{u}} &=&\bf{R}_r \bf{H}_1 \bf{E}^T \delta \dot{ \bf{d}_g }\\
	\label{Eqn:PRE:Udotdot}
	\ddot{\bf{u}}& = &\bf{R}_r \bf{H}_1 \bf{E}^T \delta \dot{ \bf{d}_g }+ (\dot{\bf{R}_r} \bf{H}_1 \bf{E}^T+\bf{R}_r \dot{\bf{H}_1} \bf{E}^T+\bf{R}_r \bf{H}_1 \dot{\bf{E}^T})\delta \dot{ \bf{d}_g }
\end{eqnarray}

Para calcular las igualdades anteriores hace falta evaluar las derivadas temporales de las matrices $\bf{E}$ y $\bf{R}_r$. Esta operatoria matricial, se traduce en derivar cada una de las entradas que integran la matriz. Dado que variable $\bf{E}$ depende de $\bf{R}_r$ se calculan inicialmente sus derivadas, para luego sustituirlas en $\dot{\bf{E}}$. Esto se realiza mediante la expresión en variaciones \eqref{eq:PRE:DifMatrix3} y resulta $\bf{R}_r=\bf{R_r}\widetilde{\dot{\bf{w}}_r}$. Al sustituir esta expresión en la derivada de $\dot{\bf{E}}$ se deduce la ecuación que prosigue:

\begin{equation}
\dot{\bf{E}}=\begin{bmatrix}
	\dot{\bf{R}_r} & \bf{0} & \bf{0} & \bf{0} \\
	\bf{0}&\dot{\bf{R}_r}  & \bf{0} & \bf{0}\\
	\bf{0}& \bf{0} & \dot{\bf{R}_r} & \bf{0}\\
	\bf{0}& \bf{0} & \bf{0} & \dot{\bf{R}_r}
\end{bmatrix}\begin{bmatrix}
	\widetilde{\dot{\bf{w}}_r} & \bf{0} & \bf{0} & \bf{0} \\
	\bf{0}& \widetilde{\dot{\bf{w}}_r}  & \bf{0} & \bf{0}\\
	\bf{0}& \bf{0} & \widetilde{\dot{\bf{w}}_r} & \bf{0}\\
	\bf{0}& \bf{0} & \bf{0} & \widetilde{\dot{\bf{w}}_r}
\end{bmatrix}=\bf{E}\bf{E}_t
\end{equation}

El valor skew de los desplazamientos globales sobre la componente de deformación rígida $\widetilde{\dot{\bf{w}}_r}$ se obtiene a partir del operador definido en la Ecuación\eqref{eq:PRE:Skew}, aplicado al vector $\dot{\bf{w}}_r=\bf{G}^T\bf{E}^T\dot{\bf{d}_g}$. Además para simplificar la notación a futuro, se condensa la Expresión \eqref{Eqn:PRE:Udotdot} definiendo la matriz $\bf{C}_1$ como se enseña a continuación:

\begin{eqnarray}
	\label{Eqn:PRE:DefC1}
    \bf{C}_1 &=& \widetilde{\dot{\bf{w}}_r}\bf{H_1}+\dot{\bf{H}_1}-\bf{H}_1\bf{E}_t\\
    \label{Eqn:PRE:UdotdotC1}
    \ddot{\bf{u}} &=&\bf{R}_r\bf{H}_1\bf{E}^T\ddot{{d}_g}+\bf{R}_r\bf{C}_1\bf{E}^T\dot{{d}_g}
\end{eqnarray}

Al igual que para las velocidades de traslación, por practicidad se simplificó la nomenclatura para evitar el abuso de notación. Derivando la Ecuación \eqref{Eqn:PRE:AngularBaricentro} respecto a la variable temporal, se deduce la velocidad angular \gls{VelocidadAngular} expresada en la Ecuación \eqref{Eqn:PRE:VelAngular}. Utilizando la regla del producto la aceleración angular  \gls{AceleracionAngular} según la Ecuación \eqref{Eqn:PRE:AcelAngular}:

\begin{eqnarray}
	\label{Eqn:PRE:VelAngular}
	\dot{\bf{w}}&=&\bf{R}_r\bf{H}_2\bf{E}^T\dot{{\bf{d_g}}}\\
	\label{Eqn:PRE:DefC}
	\bf{C}_2&=&\widetilde{\bf{w}_r}\bf{H}_2+\bf{\dot{H}_2}-\bf{H}_2\bf{E}_t\\
	\label{Eqn:PRE:AcelAngular}
	\ddot{\bf{w}}&=&\bf{R}_r\bf{H}_2\bf{E}^T\ddot{{\bf{d_g}}}+ \dot{\bf{R}_r} \bf{C}_2 \dot{\bf{E}^T} \dot{ \bf{d}_g }
\end{eqnarray}

Una descripción detallada puede encontrarse en \cite{Le2014}. Dentro del apéndice de este trabajo, se desglosa las operaciones para calcular las derivadas temporales de las matrices $\bf{H}_1$ y $\bf{H}_2$. También es posible escudriñar la deducción de las matrices $\bf{C_1}$, $\bf{C_2}$, $\bf{C_3}$ y $\bf{C_4}$.

\section{Dinámica corrotacional}\label{Subsec:PRE:DinamicCorrot}

Una vez descritas las magnitudes cinemáticas de la Sección \label{Subsec:PRE:CinematicCorrot} resulta plausible calcular los efectos dinámicos que generan sus variaciones. A continuación se presentan brevemente las variables más relevantes y una explicación concisa de su obtención. Estas
variables son el vector de fuerzas internas, inerciales y sus respectivas matrices tangentes según las referencias \citep{Le2014} y \citep{Battini2002}. Acompasando con el avance histórico de la materia, resulta natural analizar primeramente los vectores de fuerza interna y su matriz de rigidez asociada, para luego ahondar en la incorporación de términos dinámicos.


\subsection{Fuerza interna y matriz tangente}\label{Sec:PRE:Interna}

En este apartado se buscan obtener las expresiones de fuerza interna del elemento y su matriz tangente estática. El vector de fuerza interna \gls{FuerzaIntLocal} para el nodo $i$ se compone, de acuerdo a la nomenclatura desplazamiento-ángulo, por la fuerza axial \gls{FuerzaAxial}, dos momentos flectores \gls{MomentoFlector1}, \gls{MomentoFlector2} y un momento torsor \gls{MomentoTorsor} para cada nodo en su configuración deformada. Esta elección análoga a los desplazamientos locales para las fuerzas internas, se presenta en la Ecuación \eqref{Eqn:PRE:FuerzElem}.

\begin{equation}\label{Eqn:PRE:FuerzElem}
bf{ f_l^{int}} =[~fl_1 ~M^1_1~ M^1_2~ M^1_3~ M^2_1~ M^2_2~ M^2_3~] = [~fl_1~\boldsymbol{m}]
\end{equation}

Tanto las magnitudes de fuerza interna como inercial se calcularán inicialmente para coordenadas locales \gls{FuerzaIntLocal}, donde su cálculo es relativamente sencillo, para luego transcribir estos resultados en términos globales \gls{FuerzaIntGlobal}. Con este cometido se define la matriz $\bf{B}$ según se expresa en la Ecuación \eqref{Eqn:PRE:CambioCoord}.

\begin{equation}\label{Eqn:PRE:CambioCoord}
	\bf{\delta d_l}=\bf{B}~\bf{\delta d_g} ~~~~~~ 	\bf{f_g^{int}}=\bf{B}^T~\bf{ f_l^{int}}.
\end{equation}


Haciendo uso de la descomposición corrotacional el cambio de variables se realiza en dos etapas sucesivas. El primer cambio de coordenadas permite expresar los grados de libertad locales referenciados a la configuración de deformación rígida. Para clarificar, se ejemplificarán estos cambios de base para los desplazamientos, siendo análogo para el resto de las magnitudes. Esta primer transformación en la Figura \ref{fig:PRE:IlusCorrotDisps}, refiere a escribir los desplazamientos locales en términos de los rígidos ($\bf{t_i}$ $\rightarrow$ $\bf{r_i}$). Consecutivamente, el segundo cambio de variables, transforma los desplazamientos desde la configuración de deformación rígida a la indeformada ($\delta \bf{d_l}$ $\rightarrow$ $\delta \bf{d_g}$). De esta manera se logra expresar todas las magnitudes relevantes en función de coordenadas estáticas y globales.

Con la ayuda algebraica de la matrices auxiliares $\bf{G}$ y $\bf{E}$, en las Ecuaciones \eqref{Eqn:PRE:Corrot:DefE} y \eqref{Eqn:PRE:Corrot:DefG} es posible vincular los ángulos diferenciales locales $\delta\bf{\overline{\theta_i}}$ con los incrementos globales $\delta \bf{d_g}$. Esto permite conocer los momentos flectores y torsores de la viga en coordenadas globales.

Análogamente el vector auxiliar $\bf{r}$ contiene a $\bf{r}_1$ según el sentido axial de la barra, por lo que reescribir este permite expresar la fuerza de directa $fa1$  en términos de la base $\bf{E_i}$. Al unir los razonamientos detallados en los párrafos anteriores, se obtienen las Ecuaciones \eqref{Eqn:PRE:FuerzaInterna} y \eqref{Eqn:PRE:DifFuerzaInterna} para el cálculo de la fuerza interna y su diferencial:

\begin{eqnarray}\label{Eqn:PRE:FuerzaInterna}
	\bf{f_g^{int}}&=&\bf{B}^T\bf{f_l^{int}}= \begin{bmatrix}
		\bf{r}\\
		\bf{PE^T}
	\end{bmatrix}\bf{f_a}\\
    \label{Eqn:PRE:DifFuerzaInterna}
	\bf{f_g^{int}}&=&\bf{B}^T\delta\bf{f_l^{int}}+\delta\bf{r}^T \text{$f_{a1}$}+\delta(\bf{EP^T}) \boldsymbol{m}
\end{eqnarray}

Una vez calculadas las fuerzas internas es de sumo interés obtener sus derivadas recepto de los desplazamientos. La matriz tangente \gls{MatrizTangenteGlobal} representa esta magnitud y es un operador indispensable para la resolución mediante métodos numéricos iterativos. Este cálculo de derivadas respecto a desplazamientos globales de la expresión \eqref{Eqn:PRE:FuerzaInterna} concluye en la Ecuación \eqref{Eqn:PRE:MatrizKest1} a continuación:

\begin{equation}\label{Eqn:PRE:MatrizKest1}
\bf{K_g}=\bf{B}^T\bf{K_l}\bf{B}+\frac{\partial (\bf{B}^T\bf{f}_l)}{\partial \bf{d}_g}
\end{equation}

Operando con la regla del producto y sustituyendo la Ecuación \eqref{Eqn:PRE:DifFuerzaInterna} para el diferencial para la fuerza interna la matriz tangente resulta :
\begin{equation} \label{Eqn:PRE:MatrizKest2}
	\bf{K_g}=\bf{B}^T\bf{K_l}\bf{B}+\bf{D} f_{a1}-\bf{E}\bf{Q}\bf{G^T}\bf{E^T} +\bf{EGar}
\end{equation}


La matriz $\bf{B}$ permite realizar el cambio de coordenadas $\delta \bf{d_a}$ a  $\delta \bf{d_g}$, de acuerdo con lo definido en \eqref{Eqn:PRE:CambioCoord}. Esta transformación de cambio de base multiplica la variable \gls{MatrizTangenteLocal} correspondiente al aporte de rigidez local del elemento. Esta depende de los estiramientos y rotaciones de la viga en su configuración local y también de la ley material implementada. Esto evidencia la versatilidad del planteo corrotacional ante diferentes tipos de elementos, donde solo hace falta modificar la matriz $\bf{K_l}$.


En la Ecuación \eqref{Eqn:PRE:MatrizKest2} la matriz  $\bf{D}$ es anti-simétrica y se calcula en función de los productos internos de los vectores $\bf{e_i}$, esta aporta la rigidez no lineal correspondiente al a fuerza axial $f_l1$ de la barra. Por otra parte, la matriz auxiliar $\bf{Q}$ se halla a partir del producto de $\bf{P}$ y los momentos nodales respecto de las coordenadas globales, y proviene de la componente no lineal de los momentos. Por último, se define el vector $\bf{a}$ agrupando así el resto. Dichas defunciones se encuentran en las siguientes Ecuaciones:

\begin{eqnarray}
	\label{Eqn:PRE:DefD}
		\bf{D}=\begin{bmatrix}
		\bf{D_3}& \bf{0}   & -\bf{D_3}   & \bf{0} \\
		\bf{0}  & \bf{0} & \bf{0}   & \bf{0}\\
		-\bf{D_3}  & \bf{0}   & \bf{D_3} & \bf{0} \\
		\bf{0}  & \bf{0}   & \bf{0}   & \bf{0}
	\end{bmatrix}&~~~\bf{D_3}=\frac{1}{\text{$l_n$}}(\bf{I}-\bf{r_1}\bf{r_1}^T)\\
    \label{Eqn:PRE:DefQ}
	\bf{Q}=\begin{bmatrix}
		\widetilde{\bf{p}^T\boldsymbol{m}} ~(1)\\
		\widetilde{\bf{p}^T\boldsymbol{m}} ~(2)\\
		\widetilde{\bf{p}^T\boldsymbol{m}} ~(3)\\
		\widetilde{\bf{p}^T\boldsymbol{m}} ~(4)
	\end{bmatrix}&~~~\bf{a} =\begin{bmatrix}
	0\\
	\eta(M_1^2+M_2^2)/l_n-(M_1^3+M_2^3)/l_n\\
	(M_1^3+M_2^3)/l_n
\end{bmatrix}
\end{eqnarray}


Se destaca que la matriz tangente de la Ecuación  \eqref{Eqn:PRE:MatrizKest2} es asimétrica, sin embargo según \cite{Nour-Omid1991} esta puede ser simetrizada sin perder la convergencia cuadrática
para el método de  Newton Raphson (N-R), siempre y cuando momentos externos nodales no sean aplicados. En este trabajo se simetrizó la matriz tangente, ya que en la aplicación los elementos serán cargados con fuerzas, esto conlleva a un numero mayor de iteraciones en converger para un determinado nivel de carga. No obstante, debido a la precisión y consistencia del vector de fuerza interna el método debe converger \cite{rankin1988use}.

\subsection{Fuerza inercial y matrices de masa tangentes}\label{Sec:PRE:Inercial}

A continuación se explayan las ecuaciones y razonamientos fundamentales para la deducción del vector de fuerzas inerciales y sus matrices tangentes asociadas. El atractivo principal de la referencia \citet{Le2014} se fragua en la consistencia de las matrices tangentes. Según el autor y otros el grado de complejidad matemático no permitía desarrollarlas \cite{Crisfield}. Esta coherencia se debe a la cabal derivación analítica del vector de fuerzas inerciales según el planteo cinemático de las variables descritas en \ref{Subsec:PRE:CinematicCorrot}. El abordaje será análogo al desarrollado para fuerzas internas y su matriz tangente. Se calculará primeramente la fuerza inercial y luego sus derivadas, con la salvedad que la magnitud primaria será la energía cinética del elemento \gls{EnergiaCineticaElem}. Esta propiedad escalar depende de las velocidades y aceleraciones de traslación globales ($\dot{\bf{u}}$,$\ddot{\bf{u}}$) como también angulares ($\dot{\bf{w}}$,$\ddot{\bf{w}}$). En las ecuaciones \eqref{Eqn:PRE:EnergiaCinetica} y \eqref{Eqn:PRE:DifEnergiaCinetica} a continuación, se presentan la energía cinética de un elemento y su diferencial. Para la obtención de la Expresión se aplicó \eqref{Eqn:PRE:DifEnergiaCinetica} la regla del producto de diferenciales y el teorema de Leibiniz para integrales de extremos fijos.

\begin{eqnarray}
		\textit{K}&=&\frac{1}{2}\int_{l_0} \dot{\bf{u}}^T A_{\rho} \dot{\bf{u}} +
		\dot{\bf{w}}^T \bf{I_{\rho}}\dot{\bf{w}}\text{$dl_0$}
		\label{Eqn:PRE:EnergiaCinetica}\\
		\delta\textit{K}&=&-\int_{l_0} \delta \bf{u}^T A_{\rho} \ddot{\bf{u}} +\delta
		\bf{w}^T[\bf{I_{\rho}}\ddot{\bf{w}}+\widetilde{\dot{\bf{w}}}\bf{I_{\rho}}\ddot{\bf{w}}]
		\text{$dl_0$}
		\label{Eqn:PRE:DifEnergiaCinetica}
\end{eqnarray}


Se hace notar que por conveniencia se omitieron los subindices "g" para las magnitudes dinámicas ($\bf{u}$,$\bf{w}$)  y sus respectivas derivadas. De igual forma, las variables del integrando en las Ecuaciones  \eqref{Eqn:PRE:EnergiaCinetica} y \eqref{Eqn:PRE:DifEnergiaCinetica} se omitió la nomenclatura OG referida al centroide del área transversal a la viga. Los elementos serán de área constante siendo  $A_{\rho}$  el producto del área transversal  y la densidad del material, análogamente la matriz \gls{Inercia} es el tensor de inercia en la configuración deformada. Si se conoce el tensor en la configuración de referencia este se puede obtener al aplicarle las rotaciones $\bf{R}^g$ y $\bf{R}_o$ consecutivamente.

Análogo al vector de fuerzas internas, los términos dinámicos son responsables del cambio de energía cinética del elemento. De igual forma, al diferenciar el vector de fuerza inercial \gls{FuerzaInercial} se obtienen las matrices tangentes dinámicas. Esto se expresa en las Ecuaciones \eqref{Eqn:PRE:defFuerzaInercial} y \eqref{Eqn:DefFuerzaInercial}.

\begin{eqnarray}
		\label{Eqn:PRE:defFuerzaInercial}
	\delta\textit{K}&=&\bf{f_k^T}\delta\bf{d}_g\\
	\label{Eqn:DefFuerzaInercial}
	\delta\bf{f_k}&=& \bf{M}\delta \ddot{\bf{d_g}}+\bf{C}\delta
	\dot{\bf{d_g}}+\bf{K}\delta{\bf{d_g}}
\end{eqnarray}


En la Ecuación \ref{Eqn:DefFuerzaInercial} se diferencian tres matrices tangentes. Cada una de ellas asociada a la derivada parcial de la energía cinética respecto de los desplazamientos, velocidades y aceleraciones. Evidentemente, la matriz de masa consistente \gls{MatrizMasa} se corresponde con la derivada respecto de la aceleración, consecutivamente la matriz \gls{MatrizGiroscopica} giroscópica se asocia la velocidad. Por ultimo \gls{MatrizCentrifuga}, se le llama a la derivada en desplazamientos y recibe el nombre de matriz centrifuga. Determinados autores \cite{cardona1988beam} y \cite{hsiao1999consistent} proponen considerar unicamente $\bf{M}$, sin embargo exhaustivos estudios en \citep{hsiao1999consistent} prueban que agregar la matriz $\bf{C_k}$ mejora el desempeño computacional para numerosos casos.

Las expresiones detalladas de estas matrices, en conjunto con el vector de fuerzas, se deducen aplicando cambios de variables sucesivos. Esto resulta idéntico a la metodología aplicada para fuerzas internas. A diferencia de la energía elástica, la energía cinética depende, no solo de desplazamientos sino también de velocidades y aceleraciones del elemento, detalladas en la Sección \ref{Sec:PRE:VelAc}.

Sustituyendo la Ecuación \eqref{Eqn:DefFuerzaInercial} en \eqref{Eqn:PRE:DifEnergiaCinetica} se halla una fórmula para la fuerza inercial respecto de las variables cinemáticas y sus diferenciales. Al integrar los desarrollos en coordenadas globales de las Ecuaciones \eqref{Eqn:PRE:Udotdot}, \eqref{Eqn:PRE:UdotdotC1}, \eqref{Eqn:PRE:VelAngular} y \eqref{Eqn:PRE:AcelAngular}  es factible calcular el vector de fuerza inercial como se muestra a continuación:

\begin{equation}\label{Eqn:PRE:FuerzaInercial}
\bf{f}_k=\left [ \int _{l_0} \left \{ \bf{H}_1^T\bf{R_r}^T \text{$A_\rho$}\ddot{\bf{u}} +\bf{H}_2^T \bf{R_r} [\bf{I}_\rho\ddot{\bf{w}}+\widetilde{\dot{\bf{w}}}\bf{I}_\rho\dot{w}] \right \} \text{$d_l$} \right  ]
\end{equation}

Como se mencionó anteriormente para el obtener analíticamente las expresiones de la matriz consistente y giroscópica hace falta hallar analíticamente el diferencial fuerza interna. Una vez identificadas los términos que multiplican a cada incrementos de las magnitudes cinemáticas, se deducen ambas matrices. Finalmente esto se expresa de forma matemática en las Ecuaciones \eqref{Eqn:PRE:MatrizM} y \eqref{Eqn:PRE:MtarizC}.

\begin{eqnarray}
	\Delta \bf{f_k}&= &\bf{M} \Delta \ddot{\bf{d}_g}+\bf{C}_k \Delta \dot{\bf{d}_g}+\bf{K}_k \Delta \bf{d}_g\approx \bf{M} \Delta \ddot{\bf{d}_g}+\bf{C}_k \Delta \dot{\bf{d}_g}\\
	\label{Eqn:PRE:MatrizM}
	\bf{M}&=&\bf{E}\left [ \int _{l_0} \left \{ \bf{H}_1^T \text{$A_\rho$}\bf{H}_1 +\bf{H}_2^T\bf{I}_\rho\bf{H_2} \right \} \text{$d_l$} \right  ]\bf{E}^T\\
	\label{Eqn:PRE:MtarizC}
	\bf{C}_k&=&\bf{E}\left [ \int _{l_0} \left \{ \bf{H}_1^T \text{$A_\rho$}(\bf{C}_1+\bf{C}_3) +\int _{l_0}\bf{H}_2^T\bf{I}_\rho(\bf{C}_2+\bf{C}_4) +...\right \}\right  ]\bf{E}^T\\
	&...& \int_{l_0} \bf{H}_2 ^T(\widetilde{\dot{w}}\bf{I}_\rho-\widetilde{\dot{w}\bf{I}_\rho}) \text{$d_l$}
\end{eqnarray}

\newpage
