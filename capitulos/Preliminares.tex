\chapter{Conceptos preliminares}\label{Cap:Preliminares}


A continuación se presenta una descripción cualitativa y cuantitativa de la formulación corrotacional según lo propuesto por \cite{Le2014}, \citep{Battini2002}. La temática se abordará progresivamente según la naturaleza de las variables. En primera instancia se describen las magnitudes cinemáticas en las Secciones \ref{Subsec:PRE:CienmaticCorrot} y \ref{Sec:PRE:LocalFormul}. Una vez presentadas las magnitudes cinemáticas se desarrolla el análisis corrotacional para las variables estáticas y dinámicas en la Sección \ref{Subsec:PRE:DinamicCorrot}.


\section{Cinemática corrotacional}\label{Subsec:PRE:CienmaticCorrot}


El planteo corrotacional para elementos de viga 3D binodales, se basa en escindir la cinemática del movimiento en dos componentes. La primera de ellas representa grandes rotaciones y desplazamientos, dados por el movimiento de la viga considerándola como un elemento rígido. La segunda componente tiene en cuenta los desplazamientos locales asociados a la flexibilidad del material. Este enfoque se suele aplicar en casos estáticos, donde resulta intuitivo imaginar inicialmente como se deformaría la estructura de manera rígida para luego aplicarle la componente no rígida. Para poder realizar esta descomposición, hace falta introducir una serie de sistemas de coordenadas que permiten representar los desplazamientos de cada una de las componentes. Para encontrar la curva deformada que describe el elemento, hace falta la orientación y traslación de un sistema de coordenadas solidario a cada punto. Estas magnitudes se obtienen a partir de transformaciones representables matemáticamente con la artillería del álgebra matricial para rotaciones. Una presentación de la temática puede hallarse en el trabajo de \cite{kovzar1995finite}.

\subsection{Matrices de rotación}

Las configuraciones utilizadas son dos rotaciones consecutivas ilustradas en la Figura \ref{fig:PRE:IlusCorrotRot}. Para un elemento formado por los nodos 1 y 2 en sus extremos, se distinguen tres configuraciones. La primera de ellas en color azul representa el elemento en su configuración indeformada o de referencia. El color naranja identifica a la componente de deformación no rígida mientras que en gris se ilustra la configuración de deformación rígida del elemento.

Para realizar cambios  de coordenadas de una componente a otra se definen una serie de rotaciones, la primera de ellas denominada \gls{R0} lleva al elemento desde su configuración canónica a su configuración de referencia. A partir de esa configuración, se halla la geometría deformada aplicando las transformaciones \gls{R1g} o \gls{R2g}, dependiendo el nodo de interés. La notación con supraíndice ``g'' refiere a la palabra globales. Es ilustrativo referirse de esta forma a dicha transformación, ya que permite encontrar de forma ``macro'' cual es la configuración deformada partiendo del sistema de coordenadas canónico.

\begin{figure}[htbp]
	\centering
	\def\svgwidth{100mm}
	\input{./imagenes/Preliminares/Corrotacional/IlusCorrotacional2.pdf_tex}
	\caption{Rotaciones a cada configuración.}
	\label{fig:PRE:IlusCorrotRot}
\end{figure}

En la Figura \ref{fig:PRE:IlusCorrotRot}, tanto las rotaciones locales \gls{Rroof1}, \gls{Rroof2} como globales  \gls{R1g} o \gls{R2g} se utiliza el subíndice $1$ o $2$, mientras que para la rotación de deformación rígida no hace falta esta distinción. Este detalle resulta clave para comprender la metodología corrotacional. Dado que la componente de deformación rígida es rectilínea, la orientación de cada nodo es idéntica por lo que es posible prescindir del subíndice. Para hallar la configuración deformada del elemento a partir de su configuración de referencia. Una alternativa dado un nodo arbitrario, por ejemplo el 1, consiste en aplicar consecutivamente las transformaciones \gls{Rr} y \gls{Rroof1} encontrando así el configuración deformada.

\subsection{Sistemas de coordenadas}
Habiendo descrito las rotaciones del elemento, para deducir las matrices asociadas a cada transformación, resulta imprescindible definir un conjunto de sistemas de coordenadas que permitan seguir al elemento en cada configuración. Estas tríadas de vectores se muestran gráficamente a continuación en la Figura \ref{fig:PRE:IlusCorrot}.

\begin{figure}[htbp]
	\centering
	\def\svgwidth{100mm}
	\input{./imagenes/Preliminares/Corrotacional/IlusCorrotacional.pdf_tex}
	\caption{Descripción de los sistemas de coordenadas corrotacionales.}
	\label{fig:PRE:IlusCorrot}
\end{figure}


Primeramente se define un sistema de referencia canónico integrado por el sistema de coordenadas ortogonal (\gls{E1},\gls{E2},\gls{E3}). Al aplicarle a estos vectores la transformación \gls{R0}, se obtienen los vectores (\gls{e1},\gls{e2},\gls{e3}). Estos permiten ubicar al elemento en su configuración de referencia. Consecuentemente, para definir el sistema de coordenadas (\gls{r1},\gls{r2},\gls{r3}) solidario a la configuración de deformación rígida, basta con aplicar la transformación \gls{R1g}. Por último, para los nodos 1 y 2, denominados arbitrariamente con el subíndice $i$, el sistema de coordenadas (\gls{t1i},\gls{t2i},\gls{t3i}) permite identificar la orientación y posición del nodo $i$ en su configuración deformada. Esta es posible obtenerla rotando el sistema de coordenadas (\gls{e1},\gls{e2},\gls{e3}) por la matriz $\bf{R}_i^g$. 

%Se hace énfasis en el hecho de que tanto la configuración inicial como la de deformación rígida requieren un único sistema de coordenadas. Por el contrario, la configuración deformada debido a la flexibilidad del elemento, requiere dos sistemas, denotados con la letra $\bf{t}_j^i$ donde el supra-indice $i$ identifica el nodo y el sub-indice $j$ la dirección.


La definición de los sistemas de coordenadas mencionados en el párrafo anterior no es arbitraria. Una vez definidas las matrices de rotación resulta, intuitivo y oportuno escribirlas a partir de los sistemas de coordenadas solidarios a cada configuración. Esa relación intrínseca entre matrices y los sistemas de referencia se establecen en la Tabla \ref{Table:PRE:RelacionVM}:

\begin{table}[htbp]
	\begin{center}
		\begin{tabular}{|c|c|}
			\hline
			Matriz & Vínculo de sistemas de referencia \\
			\hline \hline
			$\bf{R}_0$ &$(\bf{E_1},\bf{E_2},\bf{E_3})$ $\rightarrow$
			$(\bf{e_1},\bf{e_2},\bf{e_3})$   \\ \hline
			$\bf{R}_i^g$ & $(\bf{e_1},\bf{e_2},\bf{e_3})$ $\rightarrow$
			$(\bf{t_1^i},\bf{t_2^i},\bf{t_3^i})$ \\ \hline
			$\bf{\overline{R}}_i$ &
			$(\bf{r_1},\bf{r_2},\bf{r_3})$$\rightarrow$$(\bf{t_1^i},\bf{t_2^i},\bf{t_3^i})$
			\\ \hline
			$\bf{R}_r$ &
		$(\bf{t_1^i},\bf{t_2^i},\bf{t_3^i})$$\rightarrow$$(\bf{r_1},\bf{r_2},\bf{r_3})$ \\
			\hline
		\end{tabular}
		\caption{Caracterización de matrices en términos de los sistemas de referencia.}
		\label{Table:PRE:RelacionVM}
	\end{center}
\end{table}


Los vínculos descritos en la Tabla \ref{Table:PRE:RelacionVM} se desprenden de las definiciones para cada matriz. Los vectores a la izquierda refieren al dominio de la matriz y a su derecha hacen a su respectiva imagen. A modo de ejemplo para la primera fila se tiene: $\bf{R}_0$. $(\bf{E_1})$ = $\bf{e_1}^T$. Al plantear este tipo de vínculos entre el sistema de coordenadas $(\bf{t_1^i},\bf{t_2^i},\bf{t_3^i})$ se puede hallar a partir del sistema de coordenadas canónico $(\bf{E_1},\bf{E_2},\bf{E_3})$ de dos formas. La primera consiste en aplicar consecutivamente las rotaciones $\bf{R}_0$ y $\bf{R}_i^g$ y la segunda en aplicar $\bf{R_r}$ y luego ${\bf{R_i}}$. Esto se muestra en la ecuación a continuación: 

\begin{equation}
	\label{eq:PRE:Vincul1}
	\bf{R_i^g}\bf{R_o}= \bf{R_r}\overline{\bf{R_i}}.
\end{equation}

A partir de la Ecuación \eqref{eq:PRE:Vincul1} se puede obtener la matriz de rotación $\bar{\bf{R_i}}$. Para esto se hace uso de la propiedad de matrices ortonormales $\bf{R}^T=\bf{R}^{-1}$ y se obtiene la ecuación que prosigue: 
\begin{equation}
	\label{eq:PRE:Vincul2}
	\bar{\bf{R_i}}=(\bf{R_r})^T\bf{R_i^g}\bf{R_o}.\\
\end{equation}


\subsection{Desplazamientos lineales y angulares}

El propósito de la descripción anterior, responde a la necesidad de crear herramientas analíticas que permitan vincular los desplazamientos lineales y angulares, para las distintas configuraciones, ubicando a cada elemento en coordenadas locales y globales. Las coordenadas globales se referencian al sistema de vectores (\gls{e1},\gls{e2},\gls{e3}) mientras que las locales a (\gls{t1i},\gls{t2i},\gls{t3i}). El vector de desplazamientos locales del elemento es compuesto por: el desplazamiento axial, etiquetado con la letra \gls{uloc}, y sus desplazamientos angulares nodales con el nombre \gls{theta1} y \gls{theta2}. El escalar \gls{uloc} representa el estiramiento del elemento respecto de su largo inicial ($l_0$). A su vez, el ángulo \gls{theta1} se asocia con la rotación del sistema de coordenadas (\gls{t11},\gls{t21},\gls{t31}) respecto de $(\bf{r_1},\bf{r_2},\bf{r_3})$ indicados en la Figura \ref{fig:PRE:IlusCorrot}. Estos siete grados de libertad se compactan en el vector \gls{DispLocal}=(\gls{uloc},\gls{theta1},\gls{theta2}). 


El vector de desplazamiento axial \gls{uloc} se descompone en tres componentes según el sistema de vectores $\bf{r_i}$, solidario a la configuración de deformación rígida. A este vector de desplazamientos se le denomina \gls{dispAxialLocalRigid}. Además, los desplazamientos de la viga se pueden expresar en coordenadas globales.  Para esto se utilizan las 6 magnitudes clásicas \gls{GlobalDisp}$=($\gls{GlobalDispU},\gls{GlobalDispW}$)$. Estas tienen origen en la configuración de referencia y permiten encontrar los desplazamientos en la configuración deformada. Para el nodo 1 los $\bf{\delta{w_1^g}^T}$ hacen referencia a la rotación de los vectores $(\bf{e_1},\bf{e_2},\bf{e_3})$ hasta (\gls{t11},\gls{t21},\gls{t31}). Además, los desplazamientos globales del nodo 1 $\bf{\delta u_1^g}$ se corresponden con los desplazamientos del nodo desde su configuración de referencia hasta la deformada. Esto se puede observar en la Figura \ref{fig:PRE:IlusCorrot}.
%%BorrandoEspacio
%\begin{figure}[htbp]
%	\centering
%	\def\svgwidth{100mm}
%	\input{./imagenes/Preliminares/Corrotacional/IlusDisp.pdf_tex}
%	\caption{Desplazamientos locales y globales.}
%	\label{fig:PRE:IlusCorrotDisps}
%\end{figure}

Para resolver el problema mediante métodos numéricos, es necesario definir variaciones. Estas emplearán un rol esencial para el cálculo de matrices tangentes y fuerzas internas. Las variaciones infinitesimales de los desplazamientos se definen según:

\begin{eqnarray}\label{eq:PRE:DifDisps1}
		\bf{\delta d_l} &=& [\delta\bar{u}, \boldsymbol{\delta\overline{ \theta _1}^T},	\boldsymbol{\delta \overline{ \theta _2}^T}]^{\bf{T}},\\
	\label{eq:PRE:DifDisps2}
	\bf{\delta d_g} &=& [\bf{\delta u_1^g}^T, \bf{\delta u_2^g}^T, \bf{\delta{w_1^g}^T}, \bf{\delta{w_2^g}^T}]^T.
\end{eqnarray}

Consecuente con los desplazamientos infinitesimales presentados en las Ecuaciones \eqref{eq:PRE:DifDisps1} y \eqref{eq:PRE:DifDisps2}, se deben calcular los diferenciales asociados a las transformaciones de giro $\bf{R_r^g}$, $\bf{R_i^g}$, $\bf{R_0}$ y $\bf{\overline{R}}_i$.
Para esto, primeramente deben obtenerse las matrices según lo explicitado en la Tabla \ref{Table:PRE:RelacionVM}. Las entradas de $\bf{R}_r$ y  $\bf{R}_i^g$ se hallan siguiendo las ecuaciones:
%tieneqandar
\begin{eqnarray}
	\label{eq:PRE:Rr}
	\bf{R_r}&=&[\bf{r_1} ~ \bf{r_2} ~ \bf{r_3}]\\
	\label{eq:PRE:Rg}
	\bf{R}_i^g&=&[\bf{t_1} ~ \bf{t_2} ~ \bf{t_3}]
\end{eqnarray}

 Los vectores $\bf{r_i}$ se hallan a partir del vector director $\bf{r_1}$ que apunta del nodo 1 al 2.  El versor $\bf{r_1}$ tiene como dirección la recta que une los puntos 1 y 2 en la configuración deformada, esto es equivalente a $\bf{r_1}=\frac{\bf{x}_2+\bf{u}_2^g-\bf{x}_1-\bf{u}_1^g}{\text{$l_n$}}$, donde \gls{LargoLn} es la distancia entre 1 y 2 en la configuración deformada. Dadas las posiciones iniciales de los nodos en coordenadas globales \gls{CoordX1} y \gls{CoordX2}, sus desplazamientos $\bf{u}_1^g$ y $\bf{u}_2^g$, el largo una vez deformado se calcula como $l_n=||\bf{X}_2+\bf{u}_2-\bf{X}_1-\bf{u}_1||$.
 
El vector auxiliar $\bf{p}$ se define para hallar los vectores $\bf{r}_i$ y a partir de estos la base $\bf{t}_i$. Estos vectores son solidarios al movimiento ya que se encuentran anidados a la configuración de deformación rígida y local respectivamente. El constante cambio de estas configuraciones en cada iteración, conduce a la necesidad de expresarlos en función de vectores asistentes. Para esto se definen
$\bf{p}$, $\bf{p_1}$ y $\bf{p_2}$ en las siguientes ecuaciones:

\begin{equation}\label{Eqn:Corrot:DefAuxp}
	\bf{p}=\frac{1}{2}(\bf{p}_1+\bf{p}_2),~~~~~~\bf{p_i}=\bf{R}_i^g\bf{R}_0[0~1~0]^T.
\end{equation}

En la expresión anterior la matriz $\bf{R}_0$ se obtiene colgando los vectores $\bf{e}_i$ escritos como combinación lineal de la base $\bf{E_i}$. Una vez calculada esta matriz y evaluado las expresiones de las Ecuaciones \eqref{Eqn:Corrot:DefAuxp} se obtienen los restantes vectores asociados a la componente de deformación rígida según la siguiente ecuación:


\begin{equation}\label{Eqn:Corrot:VectorsR}
	\bf{r}_3=\frac{\bf{r_1}~x~\bf{p}}{||\bf{r_1}~x~\bf{p}||},~~~~~~\bf{r_2}=\bf{r_3}~x~\bf{r_1}.
\end{equation}


Habiendo definido las matrices de rotación  es útil calcular las variaciones de las mismas. Estos cálculos son fundamentales para la transformación de variables y sus respectivos diferenciales.

\begin{equation}\label{eq:PRE:DifMatrix}
	\delta \overline{\bf{R_i}}=\delta\bf{R_r}^T\bf{R_i^g}\bf{R_0}+\bf{R_r}^T\delta \bf{R_i^g}\bf{R_0}
\end{equation}

En la Ecuación \eqref{eq:PRE:DifMatrix} se aplica la regla de la cadena para el cálculo de diferenciales matriciales. Dado que la matriz de rotación $\bf{R_0}$ vincula la configuración canónica con la de referencia, dado que ambas son fijas esta matriz es constante. Por lo tanto, su variación es nula. Definiendo el vector de ángulos de la componente de deformación rígida con el símbolo $\delta\bf{w}_r^g$, las matrices de giro $\overline{\bf{R_i}}$, $ \bf{R_i^g}$ y sus variaciones pueden hallarse según las expresiones:

\begin{eqnarray}
	\label{eq:PRE:DifMatrix2}
	\delta \bf{R_i^g} &=& \widetilde{\delta\bf{w}_i^g}~\bf{R}_i^g,\\
	\label{eq:PRE:DifMatrix3}
	\delta \bf{R_r^g} &=& \widetilde{\delta\bf{w}_r^g}~\bf{R}_r.
\end{eqnarray}

En las Ecuaciones \eqref{eq:PRE:DifMatrix2} y \eqref{eq:PRE:DifMatrix3} el término $\widetilde{\delta\bf{w}_r^g}$ refiere a la operación \textit{skew} del vector $\delta\bf{w}_r^g$. Esta operación simplifica el producto vectorial de forma matricial y es sumamente útil para el cálculo de diferenciales asociados a matrices de rotación. La función \gls{Skew} aplicada al vector $\bf{\Omega}=\text{$(\Omega_1,\Omega_2,\Omega_3)$}$ toma la siguiente forma:

\begin{equation}\label{eq:PRE:Skew}
	(\bf{\Omega})=\widetilde{\bf{\Omega}}
	=
	\begin{bmatrix}
		0 &-\Omega_3  &\Omega_2   \\
		\Omega_3&0  & -\Omega_1  \\
		-\Omega_2  & \Omega_1 & 0
	\end{bmatrix}.
\end{equation}

Para vincular los diferenciales de ángulos locales en términos de las variaciones globales se definen las matrices $\bf{E}$: 
\begin{equation}\label{Eqn:PRE:Corrot:DefE}
	\bf{E}=\begin{bmatrix}
		\bf{R_r}& \bf{0}   & \bf{0}   & \bf{0} \\
		\bf{0}  & \bf{R_r} & \bf{0}   & \bf{0}\\
		\bf{0}  & \bf{0}   & \bf{R_r} & \bf{0} \\
		\bf{0}  & \bf{0}   & \bf{0}   & \bf{R_r}
	\end{bmatrix}\rightarrow \delta \bf{d_g}=E^T \bf{d_g},
\end{equation}
 según los cocientes entre las componentes de los vectores auxiliares $\bf{p}_j$ y $\bf{p_{ij}}$ de la Ecuación \eqref{Eqn:Corrot:DefAuxp}, el vector  $\bf{p_{j}}=(\bf{R_r}^T\bf{p})$ y  $\bf{p_{ij}}=\bf{R_r}^T\bf{p}_i$ se calcula la matriz $\bf{G}$ de acuerdo con las siguientes expresiones:

\begin{eqnarray}\label{Eqn:PRE:Corrot:DefG}
		\bf{G}&=&\frac{\partial \bf{w_r^g}}{\partial \bf{d}^g}=[\bf{G}_1 ~~ \bf{G}_2]\\
		\bf{G}_1&=&\begin{bmatrix}
			0 &  0      &  \frac{p_1}{p_2 l_n} &  \frac{p_{12}}{2p_2} &-\frac{p_{11}}{2p_2}  &  0  \\
			0 &  0      &   1/l_n   &       0       &      0       &  0   \\
			0 & -1/l_n  &      0    &       0       &      0       &  0
		\end{bmatrix}\\
		\bf{G}_2&=&\begin{bmatrix}
			0  &  0    &   -1/l_n   &      0      &     0        &    0 \\
			0  &  0     &-\frac{p_1}{p_2 l_n}  & \frac{p_{2}}{2p_2} &-\frac{p_{21}}{2p_2} &    0 \\
			0  &  1/l_n &       0   &      0      &     0        &    0
		\end{bmatrix}
\end{eqnarray}

Nótese que la matriz $\bf{R}_r$ tiene dimensión 3x3. Para respetar dichas dimensiones, $\bf{0}$ es una matriz nula de 3x3 e $\bf{I}$ una matriz identidad del mismo número de filas y columnas. La relación entre los diferenciales anteriores, se pueden combinar de manera matricial, logrando así expresar los incrementos de ángulos locales en términos globales, según la siguiente ecuación:

\begin{equation}\label{Eqn:RPE:IncrementosAngulos}
	\begin{bmatrix}
		\bf{\delta\overline{\theta_1}}\\
		\bf{\delta\overline{\theta_2}}
	\end{bmatrix}=\left ( \begin{bmatrix}
		\bf{0} &\bf{I}  & \bf{0} &\bf{0} \\
		\bf{0}&\bf{0}  &\bf{0}  & \bf{I}
	\end{bmatrix}-\begin{bmatrix}
		\bf{G}\\
		\bf{G}
	\end{bmatrix} \right )\bf{E}^T \delta \bf{d_g}=\bf{P}\bf{E}^T \delta \bf{d_g}.
\end{equation}

Análogamente se debe transcribir la fuerza axial en función de las coordenadas globales. Con este objetivo se define un versor auxiliar  $ \bf{r}$ que vincula los incrementos del desplazamiento axial $\delta \overline{u}$ con los globales. Esto permite escribir la Ecuación \eqref{eq:PRE:DifDisps1} en relación a  \eqref{eq:PRE:DifDisps2} haciendo uso de la expresión:

\begin{equation}\label{eq:PRE:DefincionR}
\delta \overline{u} = \bf{r}~ d_g ~~~~~~\bf{r} = [ -\bf{r}_1^T~ \bf{0}_{1,3}~ \bf{r}_1^T~ \bf{0}_{1,3}  ].
\end{equation}

\section{Formulación local}\label{Sec:PRE:LocalFormul}
La fundamental ventaja y atractivo de la formulación corrotacional es su versatilidad ante diferentes tipos de elementos. Esto se debe al desacoplamiento analítico en la caracterización de los desplazamientos locales y globales. En esta sección se detallan las magnitudes cinemáticas en la configuración local para el cálculo de los vectores y matrices dinámicas de la Sección \ref{Subsec:PRE:DinamicCorrot}.

Sea una sección transversal de un punto G ubicado a una distancia $x$ del nodo 1 en la configuración rotada, el movimiento local de una sección ubicada a una distancia \gls{CentroideX} de la viga, desde su configuración inicial, se define a partir de la rotación y traslación de dicha sección. Una ilustración de esto se muestra en la Figura \ref{fig:PRE:IlusLocalDisp}, donde la configuración de deformación rígida se identifica en punteado y la configuración deformada en color naranja.



\begin{figure}[htbp]
	\centering
	\def\svgwidth{100mm}
	\input{./imagenes/Preliminares/Corrotacional/IlusLocalDisp.pdf_tex}
	\caption{Esquema de desplazamientos locales.}
	\label{fig:PRE:IlusLocalDisp}
\end{figure}

\begin{figure}[htbp]
	\centering
	\def\svgwidth{100mm}
 	\input{./imagenes/Preliminares/Corrotacional/IlusLocalAng.pdf_tex}
	\caption{Ilustración de grados de libertad locales.}
	\label{fig:PRE:IlusLocalAng}
\end{figure}



%\begingroup
%\begin{figure}[htbp]
%	\centering
%	\subfigure[Esquema de desplazamientos locales ]{	\def\svgwidth{70mm}
%		\input{./imagenes/Preliminares/Corrotacional/IlusLocalDisp.pdf_tex}}\label{fig:PRE:IlusLocalDisp}
%	\subfigure[Esquema de angulos locales ]{	\def\svgwidth{70mm}
%		\input{./imagenes/Preliminares/Corrotacional/IlusLocalAng.pdf_tex}}\label{fig:PRE:IlusLocalAng}
%	\caption{Ilustración grados de libertad locales} 	\label{fig:PRE:IlusLocal}
%\end{figure}
%\endgroup

El movimiento de la base $\bf{t_i}$ con respecto al sistema $\bf{r_i^G}$ está dado por los desplazamientos  $\bar{u}_3$ según el versor  $\bf{r_3^G}$ y análogamente para los vectores $\bar{u}_2$ y $\bar{u}_1$. Esto determina la ubicación del baricientro G. Su orientación se define a partir del plano punteado en color negro. La rotación de este respecto de tres ejes está dada por el plano en naranja. Este se define por dos vectores $\bf{t_3^G}$ y $\bf{t_2^G}$ dentro del plano y un versor perpendicular $\bf{t_1^G}$. La transformación $\bf{\overline{R}}_G$ permite encontrar  los transformados de la base $\bf{r_i^G}$ etiquetados con las letras $\bf{t_i^G}$ de acuerdo con la Figura \ref{fig:PRE:IlusLocalAng}. En esta también se observa el desplazamiento axial de la barra $\bar{u}$ correspondiente al del nodo 2 en la dirección $\bf{r_1}$.

Las interpolaciones para los puntos interiores al elemento se basan en las hipótesis de Bernoulli. Consecuentemente las interpolaciones son lineales para los desplazamientos axiales $\bar{u}_1$ y el ángulo de torsión $\bar{\theta}_1^G$, según las ecuaciones:

\begin{equation}
    \label{Eqn:PRE:FuncInterpol1}
	N_1 = 1 - \frac{x}{l_0},   		~~~~ 	N_2= \frac{x}{l_0}.
\end{equation}
Tanto para los desplazamientos transversales $\bar{u}_2^G$ y $\bar{u}_3^G$  como para los ángulos de flexión $\bar{\theta}_2^G$ y $\bar{\theta}_3^G$, las interpolaciones se realizan través de los polinomios cúbicos expresados en las ecuaciones a continuación:

\begin{eqnarray}
 		\label{Eqn:PRE:FuncInterpol2}
 		N_3 = x\left(1 - \frac{x}{l_0}\right)^2 	&~~~~&  N_4 - \left( 1 - \frac{x}{l_0} \right ) \frac{x^2}{l_0} \\
 		\label{Eqn:PRE:FuncInterpol3}
 		N_5 = \left(1 - \frac{3x}{l_0}\right) \left(1 - \frac{x}{l_0}\right) 	&~~~~&  N_6 =\left( \frac{3x}{l_0}-2\right) \left(\frac{x}{l_0}\right).
\end{eqnarray}

Los desplazamientos del baricentro G de la sección, respecto del sistemas de coordenadas locales, se expresan en el vector $\bf{d_l}^G$. Los valores en términos de la componente de deformación rígida $\bf{r_i}$ se calculan aplicando la siguiente ecuación:

\begin{equation}\label{Eqn:PRE:LocalDispsTransform}
	\begin{bmatrix}
		\bar{u}_1^G\\
		\bar{u}_2^G\\
		\bar{u}_3^G\\
		 \bar{\theta}_1^G\\
		 \bar{\theta}_2^G\\
		 \bar{\theta}_3^G\\
	\end{bmatrix}
=
\begin{bmatrix}
	N_2 &0  & 0 & 0 & 0 & 0 & 0   \\
	0 & 0 & 0 & N_3 & 0 &0  &N_4  \\
	0& 0 &-N_3  & 0 &0  & -N_4 &0 \\
	0 & N_1 & 0 & 0 &N_2  & 0 &0   \\
	0&0  & N_5 & 0 & 0 & N_6 &0   \\
	0 & 0 & 0 & N_5 & 0 & 0 & N_6   \\
\end{bmatrix} \bf{d_l^G}.
\end{equation}

Debido a que la matriz anterior presenta una gran cantidad de entradas nulas es útil agrupar las funciones de interpolaciones en matrices más pequeñas. De esta forma se construyen las matrices $\bf{P_1}$ y $\bf{P_2}$.  Estas expresan los desplazamientos transversales  $\bar{u}_2^G, \bar{u}_3^G $  como también los ángulos $\bar{\theta}_1^G$  y $\bar{\theta}_2^G$ y $\bar{\theta}_3^G$ según los desplazamientos lineales del baricentro y los ángulos locales  \gls{theta1} y  \gls{theta2} para el nodo 1 y 2 respectivamente. Analíticamente esto es:


 \begin{eqnarray}
 	\label{Eqn:PRE:CompactTransveralDisp}
		\begin{bmatrix}
			0     \\
		\bar{u}_2^G \\
		\bar{u}_3^G
	\end{bmatrix}=\bf{u_l}=\bf{P}_1
	\begin{bmatrix}
		\boldsymbol{\overline{\bf{\theta_1}}}\\
		\boldsymbol{\overline{\bf{\theta_2}}}
	\end{bmatrix} &~~~~~ &
	\bf{P}_1 = \begin{bmatrix}
		 0 &   0  & 0 & 0  & 0 & 0 \\
		 0  &  0 & N_3 & 0 & 0 & N_4 \\
		 0  &  -N_3  & 0 & 0 & -N_4 & 0
	\end{bmatrix} \\
	\label{Eqn:PRE:CompactTransversalAngs}
		\begin{bmatrix}
			\bar{\theta}_1^G\\
			\bar{\theta}_2^G \\
			\bar{\theta}_3^G
		\end{bmatrix}=\boldsymbol{\theta_l}=\bf{P}_2
		\begin{bmatrix}
			\boldsymbol{\overline{\bf{\theta_1}}}\\
			\boldsymbol{\overline{\bf{\theta_2}}}
		\end{bmatrix}& ~~~~~ &\bf{P}_2=
	\begin{bmatrix}
	N_1 & 0 & 0 & N_2 & 0 & 0 \\
	0  & N_5 & 0 & 0 &N_6 &0 \\
	0  &  0  & N_5 & 0 & 0 &N_6
	\end{bmatrix}
 \end{eqnarray}

%Las hipótesis de Bernoulli desprecian las deformaciones por fuerzas cortantes, esto se refleja en sus polinomios de interpolación. Esta premisa no tiene perjucios sobre la aplicación con la que se modelará el elemento. La estructura de cables es extremadamente esbelta, con relaciones de diámetro respecto a largo ínfimas. Por la tanto, las deformaciones por cortante son efectivamente despreciables respecto a las inducidas por los momentos flectores.

\subsection{Variaciones en desplazamientos} \label{Sec:PRE:VariacionesDesplazamientos}
Ya se ha remarcado en reiteradas ocasiones la importancia de los desplazamientos diferenciales para el desarrollo de matrices tangentes y fuerzas. Antes de introducir al lector en la siguiente sección, es preciso realizar una descripción previa para el cálculo de variaciones. En función de la Figura \ref{fig:PRE:IlusLocalDisp} queda definida la ubicación del baricentro OG partiendo desde el nodo 1. Esto se expresa según la siguiente ecuación con notación simplificada:

\begin{equation}\label{Eqn:PRE:DispsOG}
 	\text{OG} = \bf{x}_1^g+\bf{u}_1^g + (x+\text{$\bar{u}_1$}) \bf{r_1} + (\text{$\bar{u}_2$})\bf{r_2}+ (\text{$\bar{u}_3$})\bf{r_3}
\end{equation}

Sustituyendo los polinomios interpolantes anteriormente en las Ecuaciones \eqref{Eqn:PRE:FuncInterpol1}, \eqref{Eqn:PRE:FuncInterpol2} y \eqref{Eqn:PRE:FuncInterpol3} es posible escribir los desplazamientos del baricentro según: 

\begin{eqnarray}
	\bf{N}&=&[\bf{N_1}~~ \bf{I}~~ \bf{0}~~ \bf{N_2}~~ \bf{I}~~ \bf{0}]\\
	OG    &=& \bf{N_1}(\bf{x}_1^g+\bf{u}_1^g)+\bf{N_2}(\bf{x}_2^g+\bf{u}_2^g)+ \bf{R}_r \bf{u}_l
\end{eqnarray}
y su diferencial asociado se calcula de la siguiente forma:
\begin{equation}
	\label{Eq:PRE:DifOG}
	\delta OG = \delta \bf{u} = \bf{N}\delta\bf{d}_g+\bf{R}_r \delta\bf{u}_l + \delta \bf{R}_r \bf{u}_l.
\end{equation}

 La Ecuación \eqref{Eq:PRE:DifOG} depende de los desplazamientos locales. Esto dificulta el cálculo de su magnitud, ya que dicha variable es solidaria a sistemas de coordenadas móviles. Para solucionar este problema, se sustituyen las Ecuaciones \eqref{Eqn:PRE:Corrot:DefE}, \eqref{Eqn:PRE:Corrot:DefG}, \eqref{Eqn:RPE:IncrementosAngulos} y \eqref{eq:PRE:DifMatrix2} lográndose de este modo, escribir a $ \delta \bf{u}$ en coordenadas globales según la siguiente ecuación:

 \begin{equation}
	\label{Eqn:PRE:DifUDefH1}
	\delta \bf{u}= \bf{R}_r \bf{H}_1 \bf{E}^T \delta \bf{d}_g.
\end{equation}

 Además se compacta la notación definiendo la matriz $\bf{H}_1$ según la ecuación a continuación:
 
  \begin{equation}
 	\bf{H}_1 = \bf{R}_r(\bf{N}+\bf{P}_1\bf{P}-\widetilde{\bf{u}_l}\bf{G}^T)\bf{E}^T \delta \bf{d}_g.
 \end{equation}


 Para deducir la igualdad anterior se asumió que los incrementos angulares de las componentes locales, definidas en la Ecuación \eqref{eq:PRE:DifDisps1}, son despreciables frente a los de la componente de deformación rígida. Para el autor \cite{Le2014}, las reducidas variaciones en la geometría de dos iteraciones consecutivas implican que no hay diferencias asociadas a los incrementos de ángulos locales y rígidos, matemáticamente : $\delta \overline{\boldsymbol{\theta}_{ri}}= \overline{\delta \bf{w}_i}$.

 Un procedimiento similar se aplicará en los siguientes párrafos a las magnitudes angulares. Consecuentemente el diferencial rotación del centro de masa se puede calcular en función de los desplazamientos nodales globales según:

 \begin{equation}\label{Eqn:PRE:AngularBaricentro}
 	\delta \bf{w^g} (OG) =\delta \bf{w} =\bf{R}_r (\bf{P}_2\bf{P}+\bf{G}^T)\bf{E}^T\delta \bf{d}_g .
 \end{equation}

Análogamente a la Ecuación \eqref{Eqn:PRE:DifUDefH1}, se compacta la notación definiendo la matriz $\bf{H}_2$ según la siguiente expresión:

\begin{equation}
	\delta \bf{w}= \bf{R}_r\bf{H}_2\bf{E}^T \delta \bf{d}_g.
\end{equation}




\section{Dinámica corrotacional}\label{Subsec:PRE:DinamicCorrot}

Una vez descritas las magnitudes cinemáticas de la Sección \ref{Subsec:PRE:CienmaticCorrot} resulta plausible calcular los efectos dinámicos que generan sus variaciones. A continuación se presentan brevemente las variables más relevantes y una explicación concisa de su obtención. Estas
variables son el vector de fuerzas internas, inerciales y sus respectivas matrices tangentes según las referencias  \citep{Le2014} y \citep{Battini2002}. Acompasando con el desarrollo histórico de la materia, resulta natural definir las velocidades y aceleraciones para luego obtener los vectores de fuerza interna e inercial y sus matrices tangentes asociadas. 

\subsection{Velocidades y aceleraciones}\label{Sec:PRE:VelAc}

Las magnitudes dinámicas desempeñan un papel primordial en el análisis implementado y tanto velocidades como aceleraciones deben ser calculadas en términos globales. Para calcular estas expresiones hace falta expresar las derivadas temporales de las matrices $\bf{E}$ y $\bf{R}_r$. Esta operatoria matricial, se traduce en derivar cada una de las entradas que integran la matriz. Dado que la variable $\bf{E}$ depende de $\bf{R}_r$ se calculan inicialmente sus derivadas según la Ecuación \eqref{eq:PRE:DifMatrix3}. Al derivar se obtiene: 
\begin{equation}
\dot{\bf{R}_r} = \bf{R_r}\widetilde{\dot{\bf{w}}_r}.
\end{equation}
 Al sustituir esta ecuación en $\dot{\bf{E}}$ se deduce la siguiente expresión:

\begin{equation}
	\dot{\bf{E}}=\begin{bmatrix}
		\dot{\bf{R}_r} & \bf{0} & \bf{0} & \bf{0} \\
		\bf{0}&\dot{\bf{R}_r}  & \bf{0} & \bf{0}\\
		\bf{0}& \bf{0} & \dot{\bf{R}_r} & \bf{0}\\
		\bf{0}& \bf{0} & \bf{0} & \dot{\bf{R}_r}
	\end{bmatrix}\begin{bmatrix}
		\widetilde{\dot{\bf{w}}_r} & \bf{0} & \bf{0} & \bf{0} \\
		\bf{0}& \widetilde{\dot{\bf{w}}_r}  & \bf{0} & \bf{0}\\
		\bf{0}& \bf{0} & \widetilde{\dot{\bf{w}}_r} & \bf{0}\\
		\bf{0}& \bf{0} & \bf{0} & \widetilde{\dot{\bf{w}}_r}
	\end{bmatrix}=\bf{E}\bf{E}_t.
\end{equation}
 Derivando respecto al tiempo la Ecuación \eqref{Eqn:PRE:DifUDefH1} se deduce la siguiente expresión para la velocidad lineal \gls{VelocidadLineal}:
\begin{equation}
	\label{Eqn:PRE:Udot}
	\dot{\bf{u}} =\bf{R}_r \bf{H}_1 \bf{E}^T \dot{\bf{d}_g}.
\end{equation}
%fixing space
Aplicando la regla del producto a la Ecuación \eqref{Eqn:PRE:Udot} se halla la aceleración lineal \gls{AceleracionLineal} del baricentro:
%fixing space
\begin{equation}
	\label{Eqn:PRE:Udotdot}
	\ddot{\bf{u}} = \bf{R}_r \bf{H}_1 \bf{E}^T \ddot{\bf{d}_g} + (\dot{\bf{R}_r} \bf{H}_1 \bf{E}^T+\bf{R}_r \dot{\bf{H}_1} \bf{E}^T+\bf{R}_r \bf{H}_1 \dot{\bf{E}^T})\dot{\bf{d}_g}.
\end{equation}

El valor \textit{skew} de las velocidades angulares sobre la componente de deformación rígida $\widetilde{\dot{\bf{w}}_r}$ se obtiene a partir del operador definido en la Ecuación \eqref{eq:PRE:Skew}, aplicado al vector $\dot{\bf{w}}_r=\bf{G}^T\bf{E}^T\dot{\bf{d}_g}$. Además para simplificar la notación a futuro, se condensa la Ecuación \eqref{Eqn:PRE:Udotdot} definiendo la matriz $\bf{C}_1$ como se escribe a continuación:

\begin{equation}
	\label{Eqn:PRE:DefC1}
	\bf{C}_1 = \widetilde{\dot{\bf{w}}_r}\bf{H_1}+\dot{\bf{H}_1}-\bf{H}_1\bf{E}_t,
\end{equation}
quedando definida la aceleración lineal de la siguiente manera:
\begin{equation}
	\label{Eqn:PRE:UdotdotC1}
	\ddot{\bf{u}} = \bf{R}_r\bf{H}_1\bf{E}^T\ddot{{d}_g}+\bf{R}_r\bf{C}_1\bf{E}^T\dot{{d}_g}.
\end{equation}

Al igual que para las velocidades de traslación, por practicidad se simplificó la nomenclatura para evitar el abuso de notación. Derivando la Ecuación \eqref{Eqn:PRE:AngularBaricentro} respecto a la variable temporal, se obtiene la siguiente expresión para la velocidad angular \gls{VelocidadAngular}:

\begin{equation}
	\label{Eqn:PRE:VelAngular}
	\dot{\bf{w}}=\bf{R}_r\bf{H}_2\bf{E}^T\dot{{\bf{d_g}}}.
\end{equation}

Utilizando la regla del producto se deduce la siguiente expresión para la aceleración angular \gls{AceleracionAngular}:

\begin{equation}
	\label{Eqn:PRE:AcelAngular}
	\ddot{\bf{w}}=\bf{R}_r\bf{H}_2\bf{E}^T\ddot{{\bf{d_g}}}+ (\dot{\bf{R}_r} \bf{C}_2 \bf{E}^T + \bf{R}_r \dot{\bf{C}_2} \bf{E}^T + \bf{R}_r \bf{C}_2 \dot{\bf{E}^T}   ) \dot{ \bf{d}_g }
\end{equation}
A partir de esto, se compacta la expresión de la Ecuación \eqref{Eqn:PRE:AcelAngular} definiendo la matriz $\bf{C}_2$ de la siguiente manera:

\begin{eqnarray}
	\label{Eqn:PRE:DefC}
	\bf{C}_2&=&\widetilde{\bf{w}_r}\bf{H}_2+\bf{\dot{H}_2}-\bf{H}_2\bf{E}_t\\
\end{eqnarray}

Una descripción detallada puede encontrarse en \citep{Le2014}. Dentro del apéndice de este trabajo, se desglosa las operaciones para calcular las derivadas temporales de las matrices $\bf{H}_1$ y $\bf{H}_2$. También es posible profundizar en la deducción de las matrices $\bf{C_1}$, $\bf{C_2}$, $\bf{C_3}$ y $\bf{C_4}$.

\subsection{Fuerza interna y matriz tangente}\label{Sec:PRE:Interna}

En esta sección se busca obtener las expresiones de fuerza interna del elemento y su matriz tangente estática. El vector de fuerza interna \gls{FuerzaIntLocal} para el nodo $i$ se compone, de acuerdo a la nomenclatura desplazamiento-ángulo, por la fuerza axial \gls{FuerzaAxial}, dos momentos flectores \gls{MomentoFlector1}, \gls{MomentoFlector2} y un momento torsor \gls{MomentoTorsor} para cada nodo en su configuración deformada. Esta elección de nomenclatura para el vector \gls{FuerzaIntLocal} de fuerza interna se presenta a continuación:

\begin{equation}\label{Eqn:PRE:FuerzElem}
\bf{ f_l^{int}} =\text{$[~fa_l ~M^1_1~ M^1_2~ M^1_3~ M^2_1~ M^2_2~ M^2_3~]$} = [\text{$~fa_l$}~\boldsymbol{m}].
\end{equation}

La fuerza interna se calculará inicialmente para coordenadas locales denominada \gls{FuerzaIntLocal}, donde su obtención es relativamente sencilla, para luego transcribir estos resultados en términos globales \gls{FuerzaIntGlobal}. Con este cometido se define la matriz $\bf{B}$ de cambio de base según la siguiente expresión:

\begin{equation}\label{Eqn:PRE:CambioCoord}
	\bf{\delta d_l}=\bf{B}~\bf{\delta d_g} ~~~~~~ 	\bf{f_g^{int}}=\bf{B}^T~\bf{ f_l^{int}}.
\end{equation}

Haciendo uso de la descomposición corrotacional, el cambio de variables se realiza en dos etapas sucesivas. El primer cambio de coordenadas permite expresar los grados de libertad locales referenciados a la configuración de deformación rígida. Para clarificar, se ejemplificarán estos cambios de base para los desplazamientos, siendo análogo para el resto de las magnitudes. Según los sistemas de referencia de la Figura \ref{fig:PRE:IlusCorrot}, los cambios de variables refieren a escribir primeramente los desplazamientos locales en términos de los rígidos ($\bf{t_i}$ $\rightarrow$ $\bf{r_i}$). Consecutivamente, el segundo cambio de variables, transforma los desplazamientos desde la configuración de deformación rígida a la de referencia ($\delta \bf{d_l}$ $\rightarrow$ $\delta \bf{d_g}$). De esta manera se logra expresar todas las magnitudes relevantes en función de coordenadas inmóviles y globales.

Con la ayuda algebraica de la matrices auxiliares $\bf{G}$ y $\bf{E}$, definidas en las Ecuaciones \eqref{Eqn:PRE:Corrot:DefE} y \eqref{Eqn:PRE:Corrot:DefG} es posible vincular los ángulos diferenciales locales $\delta\bf{\overline{\theta_i}}$ con los incrementos globales $\delta \bf{d_g}$. Esto permite conocer los momentos flectores y torsores de la viga en coordenadas globales. Análogamente el vector auxiliar $\bf{r}$ contiene a $\bf{r}_1$ según el sentido axial de la barra, por lo que reescribir este último permite expresar la fuerza de directa del elemento $fa_l$ en términos de la base $\bf{E_i}$. Estos razonamientos se plasman en las ecuaciones a continuación:
\begin{eqnarray}\label{Eqn:PRE:FuerzaInterna}
	\bf{f_g^{int}}&=&\bf{B}^T\bf{f_l^{int}}= \begin{bmatrix}
		\bf{r}\\
		\bf{PE^T}
	\end{bmatrix}\bf{ f_l^{int}}\\
    \label{Eqn:PRE:DifFuerzaInterna}
	\delta\bf{f_g^{int}}&=&\bf{B}^T\delta\bf{f_l^{int}}+\delta\bf{r}^T \text{$fa_{l}$}+\delta(\bf{EP^T}) \boldsymbol{m}.
\end{eqnarray}

Una vez calculadas las fuerzas internas es de sumo interés, para la resolución empleando métodos numéricos, obtener sus derivadas respecto de los desplazamientos. La matriz tangente \gls{MatrizTangenteGlobal} representa esta magnitud y su expresión se escribe a continuación:

\begin{equation}\label{Eqn:PRE:MatrizKest1}
\bf{K_g}=\bf{B}^T\bf{K_l}\bf{B}+\frac{\partial (\bf{B}^T\bf{f}_l)}{\partial \bf{d}_g}
\end{equation}


La matriz $\bf{B}$ permite realizar el cambio de coordenadas $\delta \bf{d_a}$ a  $\delta \bf{d_g}$, de acuerdo con lo definido en \eqref{Eqn:PRE:CambioCoord}. A su vez, se define la variable \gls{MatrizTangenteLocal} correspondiente al aporte de rigidez local del elemento. Esta depende de los estiramientos y rotaciones de la viga en su configuración local y también de la ley material implementada. Esto evidencia la versatilidad del planteo corrotacional ante diferentes tipos de elementos, donde solo hace falta modificar la matriz $\bf{K_l}$.

Para calcular las matrices tangentes se define la matriz  $\bf{D}$  anti-simétrica y se calcula en función de los productos internos de los vectores $\bf{e_i}$, esta aporta la rigidez no lineal correspondiente a la fuerza axial $fa_l$ de la barra. Esta se calcula según las siguiente ecuaciones:

\begin{equation}
	\label{Eqn:PRE:DefD}
	\bf{D}=\begin{bmatrix}
	\bf{D_3}& \bf{0}   & -\bf{D_3}   & \bf{0} \\
	\bf{0}  & \bf{0} & \bf{0}   & \bf{0}\\
	-\bf{D_3}  & \bf{0}   & \bf{D_3} & \bf{0} \\
	\bf{0}  & \bf{0}   & \bf{0}   & \bf{0}
\end{bmatrix},~~~~\bf{D_3}=\frac{\text{1}}{\text{$l_n$}}(\bf{I}-\bf{r_1}\bf{r_1}^T)
\end{equation}

 Por otra parte, se define la matriz auxiliar $\bf{Q}$  a partir del producto de $\bf{P}$ y los momentos nodales respecto de las coordenadas globales, de acuerdo con la ecuación:
 \begin{equation}
 	\label{Eqn:PRE:DefQ}
 	\bf{Q}=\begin{bmatrix}
 		\widetilde{\bf{p}^T\boldsymbol{m}} ~(1)\\
 		\widetilde{\bf{p}^T\boldsymbol{m}} ~(2)\\
 		\widetilde{\bf{p}^T\boldsymbol{m}} ~(3)\\
 		\widetilde{\bf{p}^T\boldsymbol{m}} ~(4)
 	\end{bmatrix}
 \end{equation}
Ademas el vector auxiliar $\bf{a}$ se construye de la siguiente forma:
\begin{equation}
	\label{Eqn:PRE:Defa}
\bf{a} =\begin{bmatrix}
	0\\
	\eta(M_1^2+M_2^2)/l_n-(M_1^3+M_2^3)/l_n\\
	(M_1^3+M_2^3)/l_n
\end{bmatrix}.
\end{equation}



Operando con la regla del producto al diferencial de fuerza interna de la Ecuación  \eqref{Eqn:PRE:DifFuerzaInterna} y sustituyendo las definiciones postuladas en las Ecuaciones \eqref{Eqn:PRE:Defa}, \eqref{Eqn:PRE:DefD} y \eqref{Eqn:PRE:DefQ}, la matriz tangente resulta:
\begin{equation} 
	\label{Eqn:PRE:MatrizKest2}
	\bf{K_g}=\bf{B}^T\bf{K_l}\bf{B}+\bf{D} f_{a1}-\bf{E}\bf{Q}\bf{G^T}\bf{E^T} +\bf{EGar}.
\end{equation}

Se destaca que la matriz tangente de la Ecuación  \eqref{Eqn:PRE:MatrizKest2} es asimétrica, sin embargo según \cite{Nour-Omid1991} esta puede ser simetrizada sin perder la convergencia cuadrática
para el método de  \gls{N-R}, siempre y cuando los momentos externos nodales no sean aplicados. En este trabajo se simetrizó numericamente la matriz tangente, ya que en la aplicación los elementos serán cargados con fuerzas, esto conlleva a un número mayor de iteraciones en converger para un determinado nivel de carga. No obstante, debido a la precisión y consistencia del vector de fuerza interna el método debe converger según lo publicado en \citep{rankin1988use}.

\subsection{Fuerza inercial y matrices de masa tangentes}\label{Sec:PRE:Inercial}

A continuación se explayan las ecuaciones y razonamientos fundamentales para la deducción del vector de fuerzas inerciales y sus matrices tangentes asociadas. El atractivo principal de la referencia de \cite{Le2014} se basa en la consistencia de las matrices tangentes. Según el autor y otros el grado de complejidad matemático no permitía desarrollarlas \citep{Crisfield}. Esta consistencia se debe al encare analítico del vector de fuerzas inerciales, según el planteo cinemático de las variables descritas en la Sección \ref{Subsec:PRE:CienmaticCorrot}. El abordaje será análogo al desarrollado para fuerzas internas y su matriz tangente. Se calculará primeramente la fuerza inercial y luego sus derivadas, con la salvedad de que la magnitud primaria será la energía cinética del elemento \gls{EnergiaCineticaElem}. Esta propiedad escalar depende de las velocidades y aceleraciones de traslación globales ($\dot{\bf{u}}$,$\ddot{\bf{u}}$) como también angulares ($\dot{\bf{w}}$,$\ddot{\bf{w}}$) según la siguiente ecuación:
\begin{equation}
	\label{Eqn:PRE:EnergiaCinetica}
	\textit{K}=\frac{1}{2}\int_{l_0} {\dot{\bf{u}}^T A_{\rho} \dot{\bf{u}} +
	\dot{\bf{w}}^T \bf{I_{\rho}}\dot{\bf{w}}} ~\text{$dl_0$}
\end{equation}

 Dada la Ecuación \eqref{Eqn:PRE:EnergiaCinetica} se calcula la variación de energía cinética del elemento. Para la obtención de esta expresión se aplicó la regla del producto de diferenciales y el teorema de Leibiniz para integrales de extremos fijos, obteniéndose la siguiente expresión:

\begin{equation}
	\label{Eqn:PRE:DifEnergiaCinetica}
	\delta\textit{K}=-\int_{l_0} \delta \bf{u}^T A_{\rho} \ddot{\bf{u}} +\delta
	\bf{w}^T[\bf{I_{\rho}}\ddot{\bf{w}}+\widetilde{\dot{\bf{w}}}\bf{I_{\rho}}\ddot{\bf{w}}]
	\text{$dl_0$}
\end{equation}


Se hace notar que por conveniencia se omitieron los subindices ``g" para las magnitudes dinámicas ($\bf{u}$,$\bf{w}$) y sus respectivas derivadas. De igual forma, en las variables del integrando en las Ecuaciones  \eqref{Eqn:PRE:EnergiaCinetica} y \eqref{Eqn:PRE:DifEnergiaCinetica} se omitió la nomenclatura OG referida al centroide del área transversal a la viga, que si el elemento es de densidad uniforme coincide con el centro de masa de la sección. Los elementos serán de área constante siendo  $A_{\rho}$  el producto del área transversal y la densidad del material, análogamente la matriz \gls{Inercia} es el tensor de inercia en la configuración deformada. Si se conoce el tensor en la configuración de referencia este se puede obtener al aplicarle las rotaciones $\bf{R}^g$ y $\bf{R}_o$ consecutivamente.

Análogo al vector de fuerzas internas, los términos dinámicos son responsables del cambio de energía cinética del elemento. De igual forma, al diferenciar el vector de fuerza inercial \gls{FuerzaInercial} se obtienen las matrices tangentes dinámicas según las siguientes ecuaciones:

\begin{eqnarray}
		\label{Eqn:PRE:defFuerzaInercial}
	\delta\textit{K}&=&\bf{f_k^T}\delta\bf{d}_g\\
	\label{Eqn:DefFuerzaInercial}
	\delta\bf{f_k}&=& \bf{M}_k\delta \ddot{\bf{d_g}}+\bf{C}_k\delta
	\dot{\bf{d_g}}+\bf{K}_k\delta{\bf{d_g}}.
\end{eqnarray}


En la Ecuación \eqref{Eqn:DefFuerzaInercial} se diferencian tres matrices tangentes. Cada una de ellas asociada a la derivada parcial de la energía cinética respecto de los desplazamientos, velocidades y aceleraciones. Evidentemente, la matriz de masa consistente \gls{MatrizMasa} se corresponde con la derivada respecto de la aceleración, consecutivamente la matriz \gls{MatrizGiroscopica} giroscópica se asocia a la velocidad. Por ultimo \gls{MatrizCentrifuga}, se le llama a la derivada en desplazamientos y recibe el nombre de matriz centrífuga. Determinados autores como \cite{cardona1988beam} y \cite{hsiao1999consistent} proponen considerar únicamente \gls{MatrizMasa}, sin embargo exhaustivos estudios en \citep{hsiao1999consistent} prueban que agregar la matriz $\bf{C_k}$ mejora el desempeño computacional para numerosos casos.

Las expresiones detalladas de estas matrices, en conjunto con el vector de fuerzas, se deducen aplicando cambios de variables sucesivos. Esto resulta idéntico a la metodología aplicada para fuerzas internas. A diferencia de la energía elástica, la energía cinética depende, no solo de desplazamientos sino también de velocidades y aceleraciones del elemento, detalladas en la Sección \ref{Sec:PRE:VelAc}.

Sustituyendo la Ecuación \eqref{Eqn:DefFuerzaInercial} en \eqref{Eqn:PRE:DifEnergiaCinetica} se halla una fórmula para la fuerza inercial respecto de las variables cinemáticas y sus diferenciales. Al integrar los desarrollos en coordenadas globales de las Ecuaciones \eqref{Eqn:PRE:Udotdot}, \eqref{Eqn:PRE:UdotdotC1}, \eqref{Eqn:PRE:VelAngular} y \eqref{Eqn:PRE:AcelAngular}  es factible calcular el vector de fuerza inercial como se muestra a continuación:

\begin{equation}\label{Eqn:PRE:FuerzaInercial}
\bf{f}_k=\left [ \int _{l_0} \left \{ \bf{H}_1^T\bf{R_r}^T \text{$A_\rho$}\ddot{\bf{u}} +\bf{H}_2^T \bf{R_r} [\bf{I}_\rho\ddot{\bf{w}}+\widetilde{\dot{\bf{w}}}\bf{I}_\rho\dot{w}] \right \} \text{$d_l$} \right  ]
\end{equation}

Como se mencionó anteriormente para obtener analíticamente las expresiones de la matriz consistente y giroscópica hace falta hallar analíticamente el diferencial fuerza interna. Una vez identificados los términos que multiplican a cada variación incremental de las magnitudes cinemáticas, se deducen ambas matrices. Finalmente, esto se expresa de forma matemática en las siguientes expresiones:

\begin{eqnarray}
	\Delta \bf{f_k}&= &\bf{M} \Delta \ddot{\bf{d}_g}+\bf{C}_k \Delta \dot{\bf{d}_g}+\bf{K}_k \Delta \bf{d}_g\approx \bf{M} \Delta \ddot{\bf{d}_g}+\bf{C}_k \Delta \dot{\bf{d}_g}\\
	\label{Eqn:PRE:MatrizM}
	\bf{M_k}&=&\bf{E}\left [ \int _{l_0} \left \{ \bf{H}_1^T \text{$A_\rho$}\bf{H}_1 +\bf{H}_2^T\bf{I}_\rho\bf{H_2} \right \} \text{$d_l$} \right  ]\bf{E}^T\\
	\label{Eqn:PRE:MtarizC}
	\bf{C}_k&=&\bf{E}\left [ \int _{l_0} \left \{ \bf{H}_1^T \text{$A_\rho$}(\bf{C}_1+\bf{C}_3) +\int _{l_0}\bf{H}_2^T\bf{I}_\rho(\bf{C}_2+\bf{C}_4) ...\right \}\right  ]\bf{E}^T\\
	&...& +\int_{l_0} \bf{H}_2 ^T(\widetilde{\dot{w}}\bf{I}_\rho-\widetilde{\dot{w}\bf{I}_\rho}) \text{$d_l$}
\end{eqnarray}

\newpage
