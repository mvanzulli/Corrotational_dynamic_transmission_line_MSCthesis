\chapter{Metodología}\label{Cap:Metodologia}
\linenumbers

En este capitulo se exponen los fundamentos para la obtención de resultados numéricos. El problema de modelado computacional de lineas eléctricas afectadas por fenómenos de vientos extremos se construyó sobre dos etapas sucesivas. En primer lugar, se explican cuestiones sobre el modelado físico. Los protagonistas del fenómeno son el viento y la estructura. Respecto al primero se describe  en la Sección \ref{Sec:PRE:Modeloviento} el campo de velocidades absoluto, relativo y las fuerzas que el viento genera sobre el conductor. Análogamente se despliegan las condiciones iniciales y de borde consideradas para el modelado estructural en la Sección \ref{Sec:MET:ModeladoEstructural}.  Posteriormente, dentro de la Sección \ref{Sec:MET:ModeladoComput} se describe la deducción del algoritmo de HHT aplicado a la formulación corrotacional para modelado de conductores con fuerzas aerodinámicas. Este desarrollo no se ha registrado en la biografía consultada y tampoco los pseudocódigo que permiten incorporar dicha formulación al software \href{https://github.com/ONSAS/ONSAS/}{ONSAS}. Por último una vez explicada la deducción se postulan las hipótesis del modelado físico y computacional en las Secciones \ref{Sec:MET:HipotesisdeTrabajo}  y \ref{Sec:MET:HipotesisModeladoNumerico} respectivamente. 
 
\section{Aspectos de modelado físico}
\subsection{Condiciones iniciales y de borde para la estructura}\label{Sec:MET:ModeladoEstructural}
El abordaje científico computacional consiste en abstraer un fenómeno de la realidad, para crear un modelo en el computador, que se comporte de forma análoga. Permitiendo emular y controlar determinadas variables de estudio relevantes al observador. En este acto de representación existen simplificaciones inherentes, que reducen los factores incidentes al sistema como objeto de estudio. 

Una vez aislado el objeto de su entorno, es necesario imponer determinadas condiciones que representan la interacción del entorno sobre el sistema. Estas imposiciones efectuadas por el contexto, del cual el objeto esta siendo desvinculado, se nominan condiciones de borde o contorno. En particular, para esta investigación, se consideraron las siguientes hipótesis del modelado estructural respecto a sus condiciones de contorno y de borde.

\subsubsection{Hipótesis de modelado estructural}
\begin{enumerate}
	\item Las torres del sistema de transmisión se encuentran a la misma altura, ignorándose cualquier variación en el perfil tipográfico del terreno. Como consecuencia, los puntos de anclaje que unen las cadenas a las torres (D y A), pertenecen a un mismo plano paralelo a la superficie terrestre.
	\item  El conductor es conformado por un único cable continuo que discurre el espacio sujetado por aisladores eléctricos. Su proceso de fabricación es mediante una trenza con lingas de acero y aluminio, que poseen una significativa rigidez a flexión. Esta razón conduce inevitablemente a modelarlo con elementos de vigas, las cuales tienen un variación de ángulo continuo. Consecuentemente, al escindir el vano BC de su continuación (en color gris), se deben imponer las condiciones de ángulo nulo en $x$ para los nodos C y B. Esta condición es la única que respeta las condiciones de deformación angulares impostadas por la geometría del sistema. \item Como el conductor no presenta fuerzas en el sentido axial, los puntos B y C no se deforman según el eje $x$, ergo sus trayectoria pertenecen al plano $z$-$y$. Esto debe forzarse en los nodos B y C. 
	\item La exigua resistencia a flexión de los elementos aisladores DC y AB, obliga a instalarlos con sus extremos articulados. Es por esto que se modelaron a partir de barras de Green. Las condiciones de borde dependen del ejemplo al que se haga referencia. Para el Ejemplo 3 los puntos D y A se encuentran anclados a la torre, acompañando sus movimientos, mientas que para el 2 se encuentran fijos.
	\item A partir de la configuración de referencia, dibujada con línea punteada en La Figura \ref{fig:MET:EsquemaCondiciones}, se aplica una condición inicial de desplazamiento $\bf{u}_0$. Esta se corresponde con la solución estática del sistema sometido por el preso propio en la dirección de $z$ de la gravedad.
\end{enumerate} 


\begin{figure}[htbp]
	\centering
	\def\svgwidth{80mm}
	\input{./imagenes/Metodologia/EsquemaInicial.pdf_tex}
	\caption{Esquema de condición inicial y de borde.}
	\label{fig:MET:EsquemaCondiciones}
\end{figure}

Una vez plasmadas las condiciones de borde cinemáticas anteriores. Se establecen a continuación las principales condiciones de borde dinámicas:

\begin{enumerate}
	\item  No se consideran las fuerzas internas trasmitidas por los vanos aledaños.
	\item  Se desprecian las fuerzas de tensado y las condiciones de desplazamiento no homogéneas durante el proceso de instalación en la línea.
	\item Las simulaciones se realizan en dos etapas sucesivas, primeramente se imponen la condiciones iniciales de desplazamientos y una vez estabilizada la respuesta por el amortiguamiento interno y aerodinámico se ejerce la fuerza del viento sobre el cable. 
	\item Los vínculos dinámicos entre los elementos de vigas y de barra se implementaron de forma tal, que no se trasmiten los momentos de unos a otros. Por lo tanto, en los nodos de sujeción los elementos de barra CD y AB trasmiten unicamente fuerzas direccionales en C y B . 
\end{enumerate} 



\subsection{Modelo de viento}\label{Sec:PRE:Modeloviento}
Un cuerpo inmerso en un fluido en movimiento sufre determinadas cargas debido al campo de presiones en su superficie. Este campo suele producir fuerzas de arrastre (drag), en la dirección del flujo y fuerzas perpendiculares (lift). Las cargas de drag son el resultado de integrar las tensiones rasantes, en la capa limite a lo largo de la frontera del cuerpo. Y luego proyectarla la fuerza neta en la dirección del flujo medio. A diferencia de estas, las fuerzas lift que aparecen sobre el sólido, se deben a la asimetría del campo de presiones entre el intradós (sona de menor presión) y el extradós del sólido inmerso. Esta diferencia de presiones puntales entre dos superficies contrarias, genera una circulación circundante en el campo de velocidades relativos. Al integrar ese campo en curva cerrada, correspondiente a la silueta del cuerpo, se induce una fuerza. Ambos efectos dinámicos sobre el cable se ilustran en la Figura \ref{fig:MET:Viento:DragLift}. Para cuerpos perfectamente simétricos, en términos tangenciales, la competente de lift es nula.  Esto se debe a la simetría de revolución del cuerpo, garantiza que la circulación sea nula, pues no hay diferencias, ni geométricas, ni dinámicas entre las superficies del sólido. 


\begingroup
\begin{figure}[htbp]
	\centering
	\subfigure[Vista en perspetiva ]{	\def\svgwidth{60mm}
		\input{./imagenes/Metodologia/EsquemaViento.pdf_tex}}\label{fig:MET:Viento:EsqViento}
	\subfigure[Vista lateral ]{	\def\svgwidth{50mm}
		\input{./imagenes/Metodologia/DragLift.pdf_tex}}\label{fig:MET:Viento:DragLift}
	\caption{Ilustración del viento y sus efectos.} 	\label{fig:Met:Viento:Esquemas}
\end{figure}
\endgroup

La componente unidireccional del flujo a una altura dada, puede ser desglosada en un termino medido y otro fluctuante $w_v(t)=w_m(t)+{w}'(t)$. A partir de esto, la velocidad media para un período $T$ toma la expresión de la Ecuación \eqref{Eq:MET:ComponentsVel}:


\begin{equation}\label{Eq:MET:ComponentsVel}
	w_m(t)=\frac{1}{T}\int_{t-T/2}^{t+T/2}w_v(\tau)d\tau
\end{equation}


El valor del periodo $T$ debe ajustarse minimizando la desviación estándar asociada a la intensidad de turbulencia, esta se define como el cociente entre la desviación estándar de la velocidad fluctuante y la media para un instante de tiempo dado. Sin embargo, para este trabajo no se consideran las fluctuaciones debido a la presencia de vórtices en el flujo, por lo que el valor de $T=1/30$ s y de velocidad media, se extrajo del artículo \citep{stengel2017measurements}.

Considerando el aire como un fluido no newtoneano, $\rho$ su densidad asociada a determinada temperatura, $C_d$ el coeficiente de drag para como función del número de Reynolds, entonces la fuerza media en el sentido del flujo (``drag'') para un elemento cilíndrico de diámetro $d_c$ y largo $l_e$ se calcula según la Expresión \eqref{Eq:MET:FuerzaViento}:


\begin{equation}
	\label{Eq:MET:FuerzaViento}
	F_d(t)=\int_{l_0}\frac{\rho (T)C_d(Re)}{2}{d_c w_m(t)^2} dl= \frac{\rho C_d}{2}d_c w_m(t)^2l_{e}
\end{equation}

Para este cálculo se asumió constantes las magnitudes al interior del elemento, es por esto, que el valor de la integral, es simplemente el producto del integrando por el largo del intervalo. Además se para este trabajo la carga del viento sobre el elemento se modeló como una fuerza nodal equivalente a la mitad de $F_v$. Si bien la fuerza del viento es distribuida, los momentos nodales que estas inducen, se cancelan con los elementos aledaños. Por otra parte, los valores de $C_d$ se extrajeron de las referencias \citep{Foti2016} y se verificaron con el estudio para estos coeficientes durante tormentas conectivas \citep{mara2007effects}. La densidad $\rho$ del aire se consideró la usual para presión atmosférica y una temperatura de $20$ ºC.

\subsubsection{Campo de velocidades relativos, absolutos y fuerzas asociadas.}\label{SubSec:MET:CampoVelRealt}
En este trabajo no se resuelve un sistema acoplado fluido-estructura. No obstante, es preciso notar determinadas consideraciones sobre el amortiguamiento introducido. Dada una sección  transversal al cable arbitraria, donde el viento tiene determinada componente transversal según $z$ y perpendicular (según $y$). En la figura \ref{fig:MET:Viento:VelAbs} se indican con el nombre $w$ y $q$. En esta figura las velocidades se referencian a un observador solidario con la tierra y por tanto  absoluto. Asimismo en esta imagen se representan las velocidades media y fluctuante $w_m$ y $w_a$, que sumada a la velocidad $v$, resulta en el vector $V_{tot}$ formando un ángulo $\beta$ con la horizontal. Debido a la fuerza que el viento ejerce sobre el conductor, este despliega una determinada velocidad rígida en ambas direcciones identificadas con las letras $\bf{\dot{u}}_{ry}$ y $\bf{\dot{u}}_{rz}$. 
% 
%\begingroup
%\begin{figure}[htbp]
%	\subfigure[Esquema en sistema de referencias absoluto. ]{	\def\svgwidth{80mm}
%		\input{./imagenes/Metodologia/VelRel.pdf_tex}}\label{fig:MET:Viento:VelAbs}
%	\subfigure[Esquema en sistema de referencias relativo. ]{	\def\svgwidth{90mm}
%	\input{./imagenes/Metodologia/FuerzaZY.pdf_tex}}\label{fig:MET:Viento:VelRel}
%	\caption{Ilustración de fuerzas y velocidades} 	\label{fig:Met:Viento:EsquemasAmort}
%\end{figure}
%\endgroup

\begin{figure}[htbp]
	\centering
	\def\svgwidth{60mm}
	\input{./imagenes/Metodologia/VelAbs.pdf_tex}
	\caption{Esquema en sistema de referencias absoluto.}
	\label{fig:MET:Viento:VelAbs}
\end{figure}

Si el observador se encuentra solidario al rígido, en un sistema de referencia anidado a el, la velocidad percibida de viento, sería la diferencia entre las velocidades absolutas y las rígidas. Esto se muestra en la figura \ref{fig:MET:Viento:VelRel}. Este campo de velocidades relativos es el responsable de las fuerzas de drag $F_d$ y  de lift $F_l$. Estas pueden ser proyectada en el sistema de ejes globales, ocasionando dos fuerzas $F_z$ y $F_y$.


\begin{figure}[htbp]
	\centering
	\def\svgwidth{80mm}
	\input{./imagenes/Metodologia/VelRel.pdf_tex}
	\caption{Esquema en sistema de referencias relativo.}
	\label{fig:MET:Viento:VelRel}
\end{figure}

Habiendo descrito las variables que intervienen en este análisis plano, donde no se consideran cambios de orientación en sentido axial del conductor, resulta natural escudriñar en las fórmulas que vinculan las magnitudes cinemáticas y dinámicas. La velocidad relativa absoluta, es el cuadrado de los catetos, tal y como se expresa en la Ecuación \eqref{Eq:MET:VrelTot}. Tomando como hipótesis que las velocidades relativas del rígido y la componente vertical $v$, son mucho menores que las asociada al flujo medio, en el sentido de $z$ se deduce la Ecuación \eqref{Eq:MET:VerlTotRatio}. 


\begin{eqnarray}
	\label{Eq:MET:VrelTot}
	V_{rel}^2&=&(w_m+w_a-\bf{\dot{u}}_{rx})^2+(v-\bf{\dot{u}}_{ry})^2\\
	\label{Eq:MET:VerlTotRatio}
	\frac{V_{rel}^2}{w_m}&=&w_m + 2 (w_a-\bf{\dot{u}}_{rz})
\end{eqnarray}

La carga de drag postulada en la Ecuación \eqref{Eq:MET:ComponentsVel} se escribe por unidad de longitud y se reescribe en \eqref{Eq:MET:DragPerLength}. Además, se muestra que para las asunciones de velocidad media predominante, el ángulo de ataque es cercano a $0^{\circ}$. Para formular esto matemáticamente se plantean las Ecuaciones \eqref{Eq:MET:DragPerLength} y \eqref{Eq:MET:AngleBeta}.

\begin{eqnarray}
	\label{Eq:MET:AngleBeta}
	\tan (\beta_r) = \frac{v-\bf{\dot{u}}_{ry}}{w_m - \bf{\dot{u}}_{rz} + \text{$w_a$}}=\frac{\frac{v-\bf{\dot{u}}_{ry}}{w_m}}{1 \frac{- \bf{\dot{u}}_{rz} +\text{$w_a$}}{w_m}}\approx 0 \\
	\label{Eq:MET:DragPerLength}
	F_d = \frac{\rho d_c C_d}{2} (w_m + 2 (w_a-\bf{\dot{u}}_{rz}))\text{$w_m$}
\end{eqnarray}

Resulta relevante descomponer la fuerza de arrastre según las componentes $z$ e $y$. Estas son importantes ya que permiten, en un sistema de coordenadas absoluto, calcular la carga a la que se somete el conductor. A partir de estas se hallan el campo de desplzamientios, velocidad y aceleraciones del sólido. Considerando que el ángulo $\beta$ es ínfimo y por lo tanto $\tan(\beta)\approx\sin(\beta)\approx 0$ y $\cos(\beta)=1$ al aplicar trigonometría se obtienen los siguientes valores de fuerza:

\begin{eqnarray}
	F_z &=& \frac{\rho d_c C_d}{2} (u_m^2+w_a^2-2 w_a \text{$\bf{\dot{u}}_{rz}$})\cos(\beta_r)=\bar{F_x}+F_a-F_{vis}\\
	F_y &=& \frac{\rho d_c C_d}{2} (w_m^2+w_a^2-2 w_a \text{$\bf{\dot{u}}_{rz}$})\sin(\beta_r)\approx 0
\end{eqnarray}

Al igual que las variables cinemáticas, las dinámicas se pueden desglosar en componentes alternantes y medias. La parte media de cada magnitud, es una promedio móvil a lo largo del tiempo y naturalmente, las fuerzas de este tipo, se vinculan con las velocidades medias. En contraste, los términos alternantes tienen media nula y emanan de las velocidades fluctuantes. Ahora bien, un tercer termino surge al desarrollar la Ecuación \eqref{Eq:MET:DragPerLength}. Este factor depende del producto entre la velocidad media de viento y la del rígido. Vinculando al fluido y al sólido, es por esto que recibe el nombre de amortiguamiento aerodinámico. Por otra parte, desde la perspectiva del autor resulta sporepresivo el sentido de esta fuerza, siendo contrario a la ejercida por el viento.  A esta descomposición de fuerzas según $z$ se le llaman $\bar{F_x}$, $F_a$, $-F_{vis}$ a la componente media, alternante y de amoritguamentio dinámico respectivamente. Sus expresiones se detallan a continuación: 

\begin{eqnarray}
	\bar{F_x} &=&  \frac{\rho d_c C_d}{2} (w_m^2)\\
	F_a &=&  \frac{\rho d_c C_d}{2} (w_a^2)\\
	\label{Eq:MET:AmortAerodniamic}
	F_{vis}  &=& \frac{\rho d_c C_d}{2} (2 \text{$\bf{\dot{u}}_{rz}$} w_m)
\end{eqnarray}

\subsubsection{Hipótesis aplicadas al modelado de viento} \label{Sec:MET:HipotesisdeTrabajo}
Una vez descrito el análisis general de los anteriores párrafos, se postulan las premisas en las cual se fragua este trabajo. Estas evidencian las limitaciones de la metodología sobre el modelado de viento. Este si bien no el eje central de la investigación, es el agente externo principal y el causante de este estudio. Dicho esto es menester establecer las hipótesis del modelo y sus implicancias:

\begin{enumerate}
	\item No se consideran cambios en la orientación axial del conductor.
	\item La velocidad incide en el sentido $z$ de forma perpendicular a la linea. 
	\item La velocidad relativa transversal $v-\bf{\dot{u}}_{ry}$ al igual que la componente alternante son mucho menores en magnitud a la velocidad media en el sentido de $z$ llamada $w_m$.
	\item La fuerza lift debido a la simetría de revolución del conductor se considera despreciable frente al drag.
	\item Para la fuerza en el sentido de $z$ se desprecia la componente fluctuante $F_a$.
	\item Para cálculo del amortiguamiento aerodinámico $f_{vis}$ se promedió la velocidad media en un valor constante igual al valor medio para todo el dominio temporal de simulación.
\end{enumerate}

El primer supuesto parte del modelo figurado en \ref{fig:MET:Viento:VelRel}, para poder realizar este análisis plano, se obvian las fluctuaciones espaciales en el sentido axial del conductor. Esta asunción no es del todo correcta, pues la turbulencia del fenómeno provoca fluctuaciones en las cargas a lo largo dela linea, cambiando así, su orientación.  Esto se asocia directamente con la hipótesis 4, donde la fuerza alternante proveniente de la presencia de vórtices se desprecia. 

Por otra parte el flujo se consideró unidimensional según el eje $z$ en la Figura \ref{fig:MET:Viento:EsqViento}, siendo este el caso más amenazante para el conductor. Esta hipótesis proviene de diferentes trabajos publicados, donde la componente perpendicular a la superficie terrestre o ascendente (según $y$) suele ser significativamente menor a la paralela (en el sentido de $z$) \citep{duranona2009analysis} \citep{stengel2017measurements} \cite{yang2016nonlinear}. Si bien simplifica lo hace de forma conservadora.  Puesto que supone al sistema de trasmisión, en el tiempo inicial, dispuesto completamente perpendicular al sentido del viento, es así que este descarga su mayor fuerza sobre el sistema (Hipótesis 2). 

Este escenario es el más peligroso y desafiante para la seguridad e integridad de la línea. Otro argumento posible a favor de esta hipótesis, se sustenta en la mayor rigidez del cable en la dirección perpendicular al flujo, además del peso que se opone a la fuerza de sustentación. De todos modos, esta fuerza en sentido ascendente se despreció frente al drag, consecuencia de la simetría de revolución tangencial del conductor. Esto de establece en la Hipótesis 4. 

Otra hipótesis a clarificar refiere al amortiguamiento aerodinámico (Hipótesis 6). Se utilizó una simplificación adicional en la velocidad de viento para su cálculo. Se consideró una velocidad constante, igual al promedio de viento en todo el dominio temporal. Este es el valor que insertó para el cálculo de D según la Ecuación \eqref{Eq:MET:AmortAerodniamic}. Por último se explicitan las premisas 3 y 5 que fueron consideradas para calcular el campo de velocidades relativo y sus fuerzas asociadas.  
\section{Aspectos de modelado computacional}\label{Sec:MET:ModeladoComput}
\subsection{Ecuación de equilibrio}\label{Sec:MET:EqEquilibrio}

En esta sección se desarrolla la ecuación de equilibrio del sistema dinámico con valores de fuerzas externas, internas e inerciales. No se ha encontrado registros de este planteo analítico en la referencia consultada. Resulta imprescindible formular esta deducción para comprender los argumentos e hipótesis que subyacen a las expresiones postuladas en \citep{Le2014}. Por añadidura, se construye paso a paso la linealización aplicada a la ecuación de movimiento no lineal, insumo fundamental para el abordaje numérico. 

Para cada punto del cuerpo debe cumplirse el balance vectorial entre fuerzas internas $\bf{f}_{int}$, inerciales $\bf{f}_{ine}$ y externas $\bf{f}_{ext}$. Además según la Ecuación \eqref{Eq:MET:AmortAerodniamic} dentro de las fuerzas externas aparece un término aerodinámico $f_{vis}$ que depende de la velocidad lineal del rígido. Este termino debe tratarse aparte ya que su naturaleza función de el estado cinemático del problema, lo que es la incógnita a resolver.

 El equilibrio de fuerzas es equivalente al postulado de PTV donde el incremento diferencial en la energía interna y cinética se debe a un trabajo externo. La Ecuación de balance \eqref{Eq:MET:EquilibrioExacto} debe satisfacerse para todo instante temporal, en particular para $t+\Delta_t$. Dadas determinadas propiedades materiales y geométricas en la configuración de referencia, las fuerzas dependen de las magnitudes cinemáticas globales en ese instante. Estas son: el desplazamientos $\bf{d}$ $(t+\Delta_t)$, las velocidades $\dot{\bf{d}}$ $(t+\Delta_t)$ y aceleraciones $\ddot{\bf{d}}$ $(t+\Delta_t)$. Es plausible entonces plasmarlo matemáticamente de manera exacta en la Ecuación \eqref{Eq:MET:EquilibrioExacto}.

\begin{equation}\label{Eq:MET:EquilibrioExacto}
	\begin{split}
	\bf{f}_{ext,t+\Delta_t}+&\bf{f}_{vis,}(\dot{\bf{d}}\text{$(t+\Delta_t)$})-\bf{f}_{int}(\bf{d}\text{$(t+\Delta_t)$})...\\
	...~&-\bf{f}_{ine}(\bf{d} \text{$(t+\Delta_t)$},\dot{\bf{d}}\text{$(t+\Delta_t)$},\ddot{\bf{d}}\text{$(t+\Delta_t)$})=0
	\end{split}
\end{equation}
Los métodos numéricos, a groso modo, si son consistentes y estables construyen una sucesión que al discretizar infinitamente converge a la solución exacta. El método de Newton-Raphson (N-R) vectorial consiste en linealizar una ecuación a través de su diferencial de primer orden. Esta aproximación tiene como consecuencia que la Ecuación \eqref{Eq:MET:EquilibrioExacto} ya no será nula sino igual a un resto $\bf{r}$. A su vez, tal y como se detalla en las Ecuaciones \eqref{Eq:MET:AproxVel} y \eqref{Eq:MET:AproxAcel}, los métodos numéricos para la solución de problemas dinámicos, escriben las variables de aceleración y velocidad, en el instante $t +\Delta_t$, en función de los desplazamientos para ese tiempo y las magnitudes cinemáticas en el paso anterior. Como los vectores desplazamiento, velocidad y aceleración para el paso anterior se encuentran dados, el vector resto depende indirectamente de los desplazamientos. Para diferenciar las variables aproximadas de las exactas, se introduce la siguiente nomenclatura: ($\bf{d}\text{$(t+\Delta_t)$}$ $\rightarrow$ $\bf{d}_{t+\Delta_t}$), ($\dot{\bf{d}}\text{$(t+\Delta_t)$}$ $\rightarrow$ $\dot{\bf{d}}_{t+\Delta_t}$) y ($\ddot{\bf{d}}\text{$(t+\Delta_t)$}$ $\rightarrow$ $\ddot{\bf{d}}_{t+\Delta_t}$). 


\begin{eqnarray}\label{Eq:MET:AproxVel}
\dot{\bf{d}}_{t+\Delta_t}&=&F_v(\bf{d_{t+\Delta_t}},\bf{d_t},\bf{\dot{d}_t},\bf{\ddot{d}_t})\\
\label{Eq:MET:AproxAcel}
\ddot{\bf{d}}_{t+\Delta_t}&=&F_a(\bf{d_{t+\Delta_t}},\bf{d_t},\bf{\dot{d}_t},\bf{\ddot{d}_t})
\end{eqnarray}

Según el procedimiento descrito en el párrafo anterior, se buscan las aproximaciones cinemáticas tal que el residuo para un instante $t +\Delta_t$ sea próximo al vector nulo. Esto se expresa matemáticamente en Ecuación \eqref{Eq:MET:Resto}.
\begin{equation}\label{Eq:MET:Resto}
\begin{split}
	\bf{r}(\bf{d}_{t+\Delta_t})&=(-\bf{f}_{ext,t+\Delta_t}+\bf{f}_{int}(\bf{d}_{t+\Delta_t})+\bf{f}_{vis}(\dot{\bf{d}}_{t+\Delta_t})...\\	
	&...+\bf{f}_{ine}(\bf{d_{t+\Delta_t}},\dot{\bf{d}}_{t+\Delta_t}(d_{t+\Delta_t},\bf{d_t},\bf{\dot{d}_t},\bf{\ddot{d}_t}),
	\ddot{\bf{d}}_{t+\Delta_t}(d_{t+\Delta_t},\bf{d_t},\bf{\dot{d}_t},\bf{\ddot{d}_t}))
	\approx 0
\end{split}
\end{equation}

Por otro lado, según el método de N-R presentado en \cite{quarteroni2010numerical} es posible construir una sucesión iterativa en  $k$, de forma tal que en el paso siguiente, el vector resto se acerque al nulo. Para aplicar esto se utiliza el teorema de Taylor aplicado a la función resto, obteniéndose la siguiente expresión:

\begin{equation}\label{Eq:MET:Residuo}
	\bf{r}(\bf{d}^{k+1}_{t+\Delta_t})=\bf{r}(\bf{d}^k_{t+\Delta_t}) +
	\frac{\partial  \bf{r}(\bf{d}_{t+\Delta_t})}{\partial
		\bf{d}_{t+\Delta_t}}|_k~\Delta \bf{d}^{k+1}_{t+\Delta_t}=0
\end{equation}

Para calcular la derivada del residuo, se utiliza la regla de la cadena aplicada a las funciones de velocidades y aceleraciones, expresando las derivadas en función de los desplazamientos. Esta operatoria en términos analíticos, se presenta en la siguientes ecuaciones: 
\begin{equation}\label{Eq:MET:DerivadaResiduo}
	\begin{split}
	\frac{\partial  \bf{r}(\bf{d}_{t+\Delta_t})}{\partial
	\bf{d}_{t+\Delta_t}}&=\frac{\partial  \bf{r}}{\partial
	\bf{\dot{d}}_{t+\Delta_t}}\frac{\partial 
	\bf{\bf{\dot{d}}_{t+\Delta_t}}}{\partial \bf{{d}}_{t+\Delta_t}}+ \frac{\partial 
	\bf{r}}{\partial \bf{\ddot{d}}_{t+\Delta_t}}\frac{\partial 
	\bf{\bf{\ddot{d}}_{t+\Delta_t}}}{\partial \bf{{d}}_{t+\Delta_t}}+\frac{\partial 
	\bf{r}}{\partial \bf{{d}}_{t+\Delta_t}}\\
	\frac{\partial  \bf{r}(\bf{d}_{t+\Delta_t})}{\partial \bf{d}_{t+\Delta_t}}&= \frac{\partial  \bf{r}}{\partial\bf{\dot{d}}_{t+\Delta_t}}\frac{\partial F_v}{\partial \bf{{d}}_{t+\Delta_t}}+\frac{\partial  \bf{r}}{\partial\bf{\ddot{d}}_{t+\Delta_t}}\frac{\partial F_a}{\partial \bf{{d}}_{t+\Delta_t}}+\frac{\partial 
		\bf{r}}{\partial \bf{{d}}_{t+\Delta_t}}
	\end{split}
\end{equation}

En las expresiones anteriores se distinguen varios factores. En primer lugar las derivadas de la función residuo respecto de: desplazamientos, velocidades y aceleraciones. Estas son las matrices tangentes $\bf{K_g}$ $\bf{M}$ y $\bf{C_k}$ descritas en el Capítulo \ref{Cap:Preliminares}. Asimismo, al derivar la función de fuerza residual  aparece un termino $\bf{C}_{vis}$ correspondiente la derivada de la fuerza viscosa respecto de la velocidad del viento. Esto resulta una matriz diagonal esparsa con valores nulas salvo las entradas correspondientes a la dirección del viento, con valor $\rho d_c C_d w_m$. Incorporando estas matrices se obtiene a la Ecuación \eqref{Eq:PRE:DerivadaResiduo}.

\begin{equation}\label{Eq:PRE:DerivadaResiduo}
	\frac{\partial  \bf{r}(\bf{d}_{t+\Delta_t})}{\partial \bf{d}_{t+\Delta_t}}\Big|_k~
	= \left (\bf{K}_g+\frac{\partial\text{$ F_a$}}{\partial \bf{{d}}_{t+\Delta_t}}\bf{M} + \frac{\partial\text{$ F_v$}}{\partial \bf{{d}}_{t+\Delta_t}}(\bf{C}_k+\bf{C}_{vis}) \right)\Big|_k
\end{equation}

Sustituyendo la expresión anterior en la Ecuación \eqref{Eq:PRE:DerivadaResiduo} de N-R se halla el paso en desplazamientos en $k+1$ a partir de las magnitudes en k $\Delta \bf{d}^{k+1}_{t+\Delta_t}$. Matemáticamente:



\begin{equation}\label{Eq:Pre:Incremento}
	\left (\bf{K}_g+\frac{\partial\text{$ F_a$}}{\partial \bf{{d}}_{t+\Delta_t}}\bf{M} + \frac{\partial\text{$ F_v$}}{\partial \bf{{d}}_{t+\Delta_t}}(\bf{C}_k+\bf{C}_{vis} )  \right)\Big|_k^{-1} \left(-\bf{r}(\bf{d}^k_{t+\Delta_t})\right)=  \Delta\bf{d}^{k+1}_{t+\Delta_t}
\end{equation}

Una vez planteada la ecuación de equilibrio no lineal y su método de resolución numérico hace falta conocer explícitamente las funciones $F_a$ y $F_v$. Para esto se implementó el Método de HHT presentado a continuación en La sección \ref{Sec:MET:HHT}. 



\subsection{Resolución numérica mediante HHT}\label{Sec:MET:HHT}
%
 Este método consiste en una innovadora propuesta respecto del algoritmo de Newmark presentado en \cite{newmark1959method}. Según el articulo \cite{hilber1977improved} el método de HHT, es incondicionalmente estable para la integración de ecuaciones dinámicas en el área estructural. Esto implica que el paso de tiempo puede incrementarse considerablemente conservando la convergencia numérica del método. Además de esta ventaja, cuando se buscan representar modos de baja frecuencia, el factor de disipación que atenúa la energía del sistema, no depende del incremento de tiempo elegido. Complementario a esto,  evita la aparición indeseada de altas frecuencias numéricas, sin eliminar los modos de baja frecuencia endógenos a la estructura. 
 
 En la publicación \citep{hilber1977improved} se compara el método de HHT con otros métodos del clásicos en el área de análisis numérico estructural, como ser: el Método del Trapecio, el de Wilson y la familia de algoritmos de Newmark:. El autor concluye que HHT además de su mayor grado de ajuste, es mas preciso para bajas frecuencias. Dado que esto se ajusta a la perfección para la aplicación de conductores, superpuesto que este se implementó en \cite{Le2014}, resulta oportuno aplicarlo a esta investigación.
 
 Para este abordaje inicialmente se deben distinguir las magnitudes lineales de las angulares, para esto se utiliza la nomenclatura $\bf{d}=(\bf{u},\bf{w}$). Se presentan entonces las funciones de aproximación para aceleraciones y velocidades lineales globales en función de los desplazamientos. Estas ecuaciones se escribirán inicialmente en términos de los parámetros de Newmark $\alpha$ y $\beta$ para luego vincularlo con el método de HHT. Esto permite ejecutar fácilmente uno u otro, dependiendo de las necesidades. Consecuentemente, las funciones de actualización para el instante $t+\Delta_T$ se escriben:
 
\begin{align}
	 \ddot{\bf{u}}_{t+\Delta t} &=
	\begin{aligned}[t]
		\frac{1}{\alpha_{NW}(\Delta t)^2} \bf{u}_{t+\Delta t} - \frac{1}{\alpha_{NW} (\Delta t)^2} \bf{u}_{t} 	- \frac{1}{\alpha_{NW} (\Delta t)  }& \dot{\bf{u}}_{t}-... \\
		...-\frac{1}{2 \alpha_{NW}}  (1-2\alpha_{NW} ) &\ddot{\bf{u}}_{t} \\
	\end{aligned}
	 \label{Eq:MET:AproxAceleracion} \\[\jot]
	\dot{\bf{u}}_{t+\Delta t} &=
	\begin{aligned}[t]
	 \frac{\beta_{NW}}{\alpha_{NW} (\Delta t)} \bf{u}_{t+\Delta t}
	- \frac{\beta_{NW}}{\alpha_{NW} (\Delta t)} \bf{u}_{t}
	+ \left( 1 - \frac{\beta_{NW}}{\alpha_{NW}} \right) &\dot{\bf{u}}_{t}+... \\
	+ ...\left( 1- \frac{\beta_{NW}}{2 \alpha_{NW}} \right) &\ddot{\bf{u}}_{t} \Delta t
	\end{aligned}
 \label{Eq:MET:AproxVelcoity}
\end{align}


Para implementar HHT basta unicamente con definir los parámetros $\alpha_{NW}$ y $\beta_{NW}$ en términos del valor de $\alpha_{HHT}$. Esto se realiza mediante las Ecuaciones \eqref{Eq:MET:betaNW} y \eqref{Eq:MET:alphaNW}. En estas funciones, es posible notar las equivalencias, parentescos y similitudes entre los métodos. El de Newmark clásico con $\beta_{NW}=1/2$ y $\alpha_{NW} =1/4$ se logra ajustando el parámetro $\alpha_{HHT}=0$. 

\begin{eqnarray}
	\label{Eq:MET:betaNW}
 	\beta_{NW} &=& \frac{1-2\alpha_{HHT}}{2}\\
 	\label{Eq:MET:alphaNW}
 	\alpha_{NW} &=& \frac{(1-\alpha_{HHT})^2}{4}
\end{eqnarray}

Se calculan entonces las derivadas respecto al desplazamiento para las funciones de aproximación. Estas se expresan a partir del parámetro $\alpha_{HHT}$ y el incremento $\Delta_T$ ente dos tiempos consecutivos $t$ y $t+\Delta_t$.

\begin{eqnarray}
\label{Eq:Met:DerivadasAcelLinear}
\frac{\partial \bf{\ddot{u}_{t+\Delta t}}}{\partial \bf{{u}_{t+\Delta_T}}}& = &
\frac{4}{(1-\alpha_{HHT})^2 \Delta_T^2}\\
\label{Eq:Met:DerivadasVelLinear}
\frac{\partial \bf{\dot{u}_{t+\Delta_t}}}{\partial \bf{{u}_{t+\Delta_T}}}& = &
\frac{1-\alpha_{HHT}}{2\Delta_T}
\end{eqnarray}

A diferencia de la aproximación para velocidades y aceleraciones lineales, las magnitudes angulares deben actualizarse mediante otras funciones. Este tipo de variables no cumple la propiedad de conmutativiad. Es por esto, que los vector de velocidades y aceleraciones angulares para el paso $k+1$, en el instante $t+\Delta_t$, deben calcularse según  las Ecuaciones \eqref{Eq:MET:ActualizacionesAngulares1} y \eqref{Eq:MET:ActualizacionesAngulares2} presentadas en la referencias \citep{ibrahimbegovic1998finite} y \citep{ibrahimbegovic2002energy}. % y la aditividad aunque no se hizo asi en este trabajo


 \begin{eqnarray}\label{Eq:MET:ActualizacionesAngulares1}
 	\dot{\bf{w}}_{t+\Delta t}&=&\bf{\Lambda}_{t+\Delta
 		t}^g\left[\frac{\alpha}{\beta\Delta_t}\bf{\theta}_{t+\Delta_t}+
 	\frac{\beta-\alpha}{\beta}\bf{\dot{w_t}+\frac{(\beta-0.5\alpha)\Delta_t}{\beta}}
 	\ddot{\bf{w}}_{t} \right ]\\\label{Eq:MET:ActualizacionesAngulares2}
 	\ddot{\bf{w}}_{t+\Delta t}&=&\bf{\Lambda}_{t+\Delta
 		t}^g\left[\frac{1}{\beta\Delta_t^2}\bf{\theta}_{t+\Delta_t}-
 	\frac{1}{\beta\Delta_t}\bf{\dot{w_t}-\frac{(0.5-\beta)}{\beta}}
 	\ddot{\bf{w}}_{t+\Delta t} \right ]
 \end{eqnarray}

En las Ecuaciones \eqref{Eq:MET:ActualizacionesAngulares1} y \eqref{Eq:MET:ActualizacionesAngulares2} la transformación $\bf{\Lambda}_{t+\Delta t}^g$ es la composición de las rotaciones globales para dos instantes consecutivos: 

\begin{equation}\label{Eq:Met:CalculodeLamda}
	\bf{\Lambda}_{t+\Delta
		t}^g=\exp({\widetilde{\bf{\theta_{t+\Delta_t}^g}}})=\bf{R}^g_{t+\Delta_t}(\bf{R}^g_{t})^T
\end{equation}

Un procedimiento análogo al de las funciones angulares se aplican a las lineales. Esto se obtiene a partir de la derivación analítica de las Ecuaciones expresadas en \eqref{Eq:MET:ActualizacionesAngulares1} y \eqref{Eq:MET:ActualizacionesAngulares2}.

\begin{eqnarray}
	\label{Eq:MET:DerivadasAngulares1}
	\frac{\partial \bf{\ddot{w}_{t+\Delta T}}}{\partial \bf{{w}_{t+\Delta_t}}}& = 
	\frac{4}{(1-\alpha_{HHT})^2 \Delta_T^2}&  \bf{T_s^{-T}}(\bf{\theta^g_{1,{t+\Delta_t}}})\\
	\label{Eq:MET:DerivadasAngulares2}
	\frac{\partial \bf{\dot{w}_{t+\Delta_T}}}{\partial \bf{{w}_{t+\Delta t}}}& = 
	\frac{1-\alpha_{HHT}}{2\Delta_T}& \bf{T_s^{-T}}(\bf{\theta^g_{1,{t+\Delta_t}}})
\end{eqnarray}

Es posible compactar las derivadas lineales y angulares de las Ecuaciones \eqref{Eq:MET:DerivadasAngulares1}, \eqref{Eq:MET:DerivadasAngulares2}, \eqref{Eq:Met:DerivadasAcelLinear} y \eqref{Eq:Met:DerivadasVelLinear} al definir convenientemente la matriz $\bf{B_t}$. En función de esta es posible escribir los incrementos de velocidades y aclaraciones globales en términos del vector de desplazamientos inceremental. Estas relaciones se expresan a continuación:
	 
\begin{eqnarray}
	\label{Eq:MET:DefBt}
	\bf{B}_t & = & \bf{\begin{bmatrix}
			I& 0 & 0 & 0\\ 
			0 & \bf{T_s^{-T}}(\bf{\theta^g_{1,{t+\Delta_t}}}) & 0 & 0\\ 
			0& 0 & I & 0\\ 
			0 & 0 & 0 & \bf{T_s^{-T}}(\bf{\theta^g_{2,{t+\Delta_t}}})
	\end{bmatrix}}\\
	\label{Eq:MET:IncrementoVelGlobalBt}
	\Delta \dot{\bf{d_g}}&=& \left(\frac{1-\alpha_{HHT}}{2\Delta_T} \bf{B_t}\right) \Delta
	\bf{d_{g,{t+\Delta_t}}}\\
	\label{Eq:MET:IncrementoAcelGlobalBt}
	\Delta \ddot{\bf{d_g}}&=& \left( \frac{4}{(1-\alpha_{HHT})^2 \Delta_T^2} \bf{B_t} \right)\Delta
	\bf{d_{g,{t+\Delta_t}}}
\end{eqnarray}	



Al escindir las Ecuaciones \eqref{Eq:MET:IncrementoVelGlobalBt} y \eqref{Eq:MET:IncrementoAcelGlobalBt} se identifican las funciones $F_a$ y $F_v$ de la sección \ref{Sec:MET:EqEquilibrio}. Estas relaciones matemáticas deben de integrarse a la Ecuación linealizada de equilibrio \eqref{Eq:Pre:Incremento} para obtener el incremento en $k$ que permita conocer el vector desplazamientos en el paso $k+1$ para el instante $t+\Delta_T$. Finalmente, eso se plantea en la Ecuación \eqref{Eq:MET:FinalIncremento}.


\begin{equation}\label{Eq:MET:FinalIncremento}
	\begin{split}
		\bf{r}(\bf{d}^k_{t+\Delta_t})=\\
		-\left(\bf{K}_g+\text{$\left( \frac{4}{(1-\alpha_{HHT})^2\Delta_T^2} \right)$} \bf{M}\bf{B}_t + \left(\text{$\frac{1-\alpha_{HHT}}{2\Delta_T}$}\right) (\bf{C}_k+\bf{C}_{vis}) \bf{B}_t \right)\Delta\bf{d}^{k+1}_{t+\Delta_t}
	\end{split}
\end{equation}

Se aclara que para despejar la Ecuación \eqref{Eq:MET:FinalIncremento} anterior, la matriz entre paréntesis curvos debe invertirse y por tanto ser no singular. De lo contrario, el método podría presentar un número de condición nulo arrojando infinitas soluciones o ninguna.  Esto se encuentra garantizado por la naturaleza de las matrices que la integran (de masa, centrifuga y tangente). Las matrices tangentes fueron simetrizadas "artificialmente" como se aclaró anteriormente, manteniendo el orden de convergencia de N-R. Las matrices centrifugas y de masa devienen de un potencial asociado (la energía cinética) como los parámetros $\alpha_{HHT}$ son menores a uno, en general en el intervalo $[-0.1;0.1]$, la suma de esta matrices suele ser definidas positivas. Por lo que $\bf{K}_{tot}$ será invertible.

\subsubsection{Hipótesis de modelado numérico}\label{Sec:MET:HipotesisModeladoNumerico}
Se esclarecen las premisas y simplificaciones durante la implementación numérica de los códigos creados: 
\begin{enumerate}
\item Los incrementos angulares no se calcularon componiendo dos rotaciones consecutivas sino de forma aditiva, es decir: $\bf{\theta_{t+\Delta_t}^{k+1}}=\bf{\theta_{t+\Delta_t}^{k}}+\bf{\Delta \theta_{t+\Delta_t}^{k+1}}$.
\item La matriz de amortiguamiento viscoso $\bf{C}_{vis}$ se considero una diagonal con elementos no nulos en las componentes asociadas a los desplazamientos transaccionales. Se copió el valor del amortiguamiento aerodinámico  con el valor correspondiente a la coordenada lineal $\rho d_c C_d w_m$ para el resto de los desplazamientos. Esto garantiza la estabilidad y atenuación de la respuesta en la primer etapa asociada al peso.
\item La simulación se separó en dos etapas consecutivas, en primer lugar se carga con la fuerza de la gravedad (a partir de la condición inicial) y una vez que la respuesta es constante se aplica la carga del viento. 
\end{enumerate}


\subsection{Implementación numérica en ONSAS}\label{Sec:MET:ImplementNumeric}
En la sección que prosigue se detallan los códigos implementados en el software: \emph{An Open Non Linear Structural Analysis Solver} (\href{https://github.com/ONSAS/ONSAS/}{ONSAS}). Este código de carácter abierto y se desarrolló de forma general integrando distintos elementos, materiales y geometrías dentro del mismo modelo. Además permite resolver mediante diversos algoritmos numéricos y visualizar gráficamente sus salida en 3D a través del programa de código abierto \emph{Paraview} difundido en \citep{ahrens2005paraview}.

Las líneas de código relacionadas con la formulación local, las funciones matemáticas de rotación, las fuerzas internas y sus matrices tangentes fueron aportadas por el Dr. Jean Mark Battini. Su intervención constituye uno de los pilares fundamentales en la construcción de este trabajo, no solo por ser pionero dela formulación corrotacional aplicada a estructuras, publicadas en los trabajos \citep{Battini2002} \citep{Le2014}, sino también por su predisposición a difundir los códigos de su investigación, cuyo valor es invaluable. A continuación en \ref{Pseudo:MET:Corrot} se detalla un pseudo-código panorámico sobre el esqueleto ejecutado en \href{https://github.com/ONSAS/ONSAS/}{ONSAS}. 

\begin{algorithm}\label{Pseudo:MET:Corrot}
	\begin{algorithmic} 
		\REQUIRE: $tol_r$, $tol_u$, $\text{maxIter}$, $\Delta_T$, $\alpha_{HHT}$
		\STATE Iniciar cinemáticas: $\bf{d_t}\xleftarrow[]{}\bf{d_0}$
		$\bf{\dot{d}_t}\xleftarrow[]{}\bf{\dot{d}_0}$
		$\bf{\ddot{d}_t}\xleftarrow[]{}\bf{\ddot{d}_0}$ 
		\STATE Iniciar tiempo: $t\xleftarrow{}t_0$
		\WHILE {$t<t_f$}
			\STATE finDisp = $0$
			\STATE Definir: $\bf{d^k}\xleftarrow[]{}\bf{d_t}$, $\bf{\dot{d}^k}\xleftarrow[]{}\bf{\dot{d}_t}$, $\bf{\ddot{d}^k}\xleftarrow[]{}\bf{\ddot{d}_t}$.
			\STATE Evaluar $\bf{f_{ext,t+\Delta t}}$	
			\WHILE{FinDisp = 0}
				\STATE Calcular fuerzas: $\bf{f}_{ine}^k(\bf{d}^k,\bf{\dot{d}}^k,\bf{\ddot{d}}^k)$,  $\bf{f}_{int}^k(\bf{d}^k)$ y $\bf{res}^{k}(\bf{d}^k,\bf{\dot{d}}^k,\bf{\ddot{d}}^k)$. 
				\STATE Calcular y ensamblar matrices Tangentes: $\bf{K}_g^k$ $\bf{M}^k$  $\bf{C}_k^k$. 
				\STATE Despejar $\Delta \bf{d}^{k+1}$ 
				\STATE Actualizar desplazamientos globales:  $\bf{d}^{k+1} = \bf{d}^{k} + \Delta\bf{d}^{k+1}$ 
				\STATE Recalcular velocidades y aceleraciones lineales: ($\bf{\dot{u}}^{k+1}$), ($\bf{\ddot{u}}^{k+1}$).
				\STATE Recalcular velocidades y aceleraciones angulares: ($\bf{\dot{w}}^{k+1}$), ($\bf{\ddot{w}}^{k+1}$).
				\STATE Ensamblar velocidades: $\bf{\dot{d}}^{k+1}$ $\leftarrow$ ($\bf{\dot{u}}^{k+1}$,$\bf{\dot{w}}^{k+1}$)
				\STATE Ensamblar aceleraciones: $\bf{\ddot{d}}^{k+1}$ $\leftarrow$ ($\bf{\ddot{u}}^{k+1}$,$\bf{\ddot{w}}^{k+1}$)
´				\STATE Actualizar fuerzas: 	$\bf{f}_{ine}^{k+1}(\bf{d}^{k+1},\bf{\dot{d}}^{k+1},\bf{\ddot{d}}^{k+1})$,  $\bf{f}_{int}^{k+1}(\bf{u}^{k+1})$ y $\bf{res}(\bf{d}^{k+1})$. 
				\STATE Calcular: 
				\IF {$\left \| \Delta \bf{d}^{k+1} \right \|< tol_d \left \| 
					\bf{d}^{k+1} \right \| $  $~V~$ $\left \| \bf{res}(\bf{d}^{k+1}) \right \|<tol_r \left \| \bf{f}_{ext} \right \| $  $~V~$ $ \text{k}\geq \text{max}_{iter}$ }
				\STATE finDisp = 1
				\ENDIF
			\ENDWHILE
			\STATE Actualizar $\bf{d_{t}}\xleftarrow[]{}\bf{d^{k+1}_{t+\Delta_T}}$, $\bf{\dot{d}_{t}}\xleftarrow[]{}\bf{\dot{d}^{k+1}_{t+\Delta_T}}$, $\bf{\ddot{d}_t}\xleftarrow[]{}\bf{\ddot{d}^{k+1}_{t+\Delta_T}}$.
			\STATE $t = t+\Delta_T$	
 		\ENDWHILE
	\end{algorithmic}
\caption{Pseudocódigo de iteración general. }
\end{algorithm}

En la estructura de códigos anterior se observan dos bucles en simultaneo. Inicialmente se ejecuta un primer $\bf{while}$ de avance cronológico, que permite incrementar la variable temporal en pasos de $Delta_T$. Además debe evaluar los valores que son constantes en el tiempo, como ser: la magnitud de $\bf{f}_{ext}$. Para resolver el estado del sistema en el tiempo $t+\Delta T$, hace falta resolver la ecuación no lineal del resto descrita en la Expresión \eqref{Eq:MET:Resto}. Con este cometido, se construye una sucesión en desplazamientos que tienda a la solución para ese paso, esto se realiza mediante (N-R) en el segundo $\bf{while}$ en desplazamientos. Para este bucle en el pseudocódigo \ref{Pseudo:MET:Corrot} se omitió la notación en $t+\Delta T$ para simplificar, mas todas las variables se corresponden a dicho tiempo.

Esta parte del código se pudría subdividir en dos estructuras, primeramente el cálculo del incremento que determina el paso $k+1$, a partir de los desplazamientos en el paso actual $k$. Luego se actualizan las variables cinemáticas de desplazamientos, velocidades y aceleraciones. Este conjunto de pasos se realiza mientras que la variable boolena finDisp sea nula. La alteración de estado, se encuentra atada a la operación lógica de la sentencia $\bf{if}$. Esta se rige por la operación lógica disyunta, aplicada a tolerancias en desplazamientos $tol_u$, en vector de fueras residuales $tol_{res}$ y número máximo de iteraciones $max_{Iter}$. Las primeras dos son relativas al valor de fuerzas externas y desplazamientos en ese tiempo, lográndose de este modo independizarse de las magnitudes absolutas desconocidas. Una vez que el segundo bucle en desplazamientos converge, la variable finDisp alcanza la unidad. A partir de esto, se actualizan tanto el valor del tiempo, como las magnitudes cinemáticas para el instante siguiente.

Habiendo explicado la estructura general del código, resulta importante profundizar y desplegar el cálculo de la función de fuerzas inerciales y matrices dinámicas tangentes. Este código se agregó a \href{https://github.com/ONSAS/ONSAS/}{ONSAS} procurando su versatilidad. De esta forma será posible aplicarlo a futuras aplicaciones que trascienden al alcance y foco de este trabajo. Se presenta a continuación un esquema tipo pseudocódigo de la función \href{https://github.com/ONSAS/ONSAS/blob/master/src/elementBeamForces.m}{elementbeamforces.m} implementada.

\begin{algorithm}\label{Pseudo:MET:elementBeamForces}
	\begin{algorithmic} 
	\REQUIRE $A_{\rho}$ $\bf{I}_{\rho}^{ref}$ $E$ $\nu$ $G$ $\bf{X}_1$ $\bf{X}_2$ $\bf{d}_g^e$
		 \FOR{1 \TO $N_{elem}$ } 
		 	\STATE Separar vector desplazamientos $\bf{d}_g=(\bf{u}^g,\bf{w}^g)$
			\STATE ----------------$ \text{Cálculo de matrices de rotación } $---------------- 
			\STATE Computar matrices de rotación global $\bf{R}_g^1$ y $\bf{R}_g^2$
			\STATE Evaluar matriz de rotación de referencia $\bf{R}_o$
			\STATE Hallar $\bf{q}_1$ $\bf{q}_2$ $\bf{q}$ y calcular $\bf{e}_1$ $\bf{e}_2$ y $\bf{e}_3$.
			\STATE Evaluar maitrz de rotación rígida $\bf{R_r}$
			\STATE Calcular matrices de rotación locales $\bf{R}_i = \bf{R}_r^T \bf{R}_g^i \bf{R}_o$
			\STATE -----------------------------------------------------------------------------  
			\STATE --------- $\text{Cálculo de fuerza interna y matriz tangente } $---------
			\STATE Calcular largos iniciales, actuales y estiramiento $l_0$ y $l$ $u=l-l_0$
			\STATE Invertir $\bf{R}_i$ y hallar ángulos locales $\bf{\bar{\theta}_i}$.
			\STATE Ejecutar \emph{beamLocalStaticForces } para fuerza interna $\bf{f}_{int}^{loc}$ y matriz tangente local $\bf{K}_{T}^{loc}$.
			\STATE Construir matrices auxiliares: $\bf{H}$ $\bf{G}$ $\bf{P}$ $\bf{B}$  $\bf{r}$ 
			\STATE Transformar a coordenadas globales:  $\bf{K}_{T}^{g}\leftarrow\bf{K}_{T}^{loc}$ y $\bf{f}_{int}^{g}\leftarrow\bf{f}_{int}^{loc}$.
			\STATE -----------------------------------------------------------------------------  
			\STATE ------ $\text{Cálculo de fuerza inerciales y matrices dinámcias} $------
			\STATE Todas las variables dependen de la coordenada (x)
			\STATE Definir funciones de interpolación $N_i$
			\STATE Calcular matrices: $\bf{P_1}(x)$, $\bf{P_2}$, $\bf{N}$ y $\bf{H}_1$.
			\STATE Hallar velocidades  $\dot{\bf{w}}$, $\dot{\bf{u}}$ y $\dot{\bf{w}}_r$ 
			\STATE Calcular matrices auxiliares: $\bf{H}_1$, $\dot{\bf{H}}_1$, $\bf{H}_2$, $\dot{\bf{H}}_2$, $\bf{C}_1$, $\bf{C}_2$, $\bf{C}_3$ y $\bf{C}_4$.        
			\STATE Hallar las aceleraciones: $\ddot{\bf{w}}$  $\ddot{\bf{u}}$.
			\STATE Girar el tensor de inercia a la configuración deformada: $\bf{I}_{\rho}$ $\leftarrow$ $\bf{I}_{\rho}^{ref}$
			\STATE Hallar expresiones e integrar en el elemento: $\bf{f}_{ine}$ $\bf{M}$ y $\bf{C}_k$
			\STATE  Ensamblar : $\bf{f}_{ine}$ $\bf{M}$, $\bf{C}_k$ $\bf{K}_{T}^{g}$ $\bf{f}_{int}^{g}$
			\STATE -----------------------------------------------------------------------------  
		\ENDFOR 
	\end{algorithmic}
	\caption{Pseudocódigo elementBeamForces. }
\end{algorithm}

El diagrama presentado en el Pseudocódigo \ref{Pseudo:MET:elementBeamForces}, puede dividirse en tres divisiones principales. Esto ordena el código consecutivamente según el desarrollo constructivo de las variables intervinientes. Primeramente se hallan las matrices de rotación, que vinculan las configuraciones: de referencia, rígida y deformada. Una vez representadas estas transformaciones, se procede a calcular las fuerzas internas y las matrices tangentes en la configuración local a través de la función \href{https://github.com/ONSAS/ONSAS/blob/master/src/beamLocalStaticForces.m}{beamLocalStaticForces}. Desafortunadamente, tanto entradas como salidas de esta función, se encuentran referidas al sistema de coordenadas locales. Es por esto, que resulta inevitable calcular los ángulos y desplazamientos locales. Asimismo transformar las salidas a coordenadas globales, para luego integrarlas al código general expuesto en \ref{Pseudo:MET:Corrot}.

De forma subsiguiente se arman las matrices dinámicas y los vectores de fuerza inercial asociados al elemento. Con este fin, se calculan primero las expresiones analíticas de las magnitudes cinemáticas en cada sección. Estas están referidas a su baricentro, ubicado a una distancia $x$ en la configuración de referencia. Como su obtención directa es algo compleja, se definen una serie de variables auxiliares y sus respectivas derivadas que permiten calcularlas.  

Una vez finalizado estos pasos, se integran las matrices tangentes y el vector de fuerzas inerciales, empleando el método de integración numérica de cuadratura de Gauss. Este se implementó con 3 puntos de integración. Por último, los valores obtenidos tanto para las matrices tangentes dinámicas y estáticas,  como para los vectores de fuerza inercial e internas se ensamblan a las matrices de todo el sistema en coordenadas globales.
