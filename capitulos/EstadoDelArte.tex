\chapter{Estado del arte}\label{Cap:EstadoDelArte}\linenumbers

Este capítulo incluye la revisión de la literatura, de los enfoques, teorías o conceptos pertinentes en que se fundamenta la investigación. Primeramente en la Sección \ref{Sec:EA:Historia} se presenta un relato cronológico del estudio de los cables desde el crepúsculo del Siglo XVIII. A continuación en la Sección \ref{Sec:EA:AplicadasConductores} se expone un recorrido a partir de los años 60's vinculado a simulaciones aplicadas a conductores de alta tensión. Consecutivamente en la Sección \ref{Sec:EA:TormentasConvectivas} se describen los fenómenos de CD que afectan las líneas a partir de trabajos nacionales e internacionales. Estas tormentas y otros fenómenos de viento afectan a las líneas produciendo inestabilidades aeroelásticas numerosos trabajos han estudiado dicha temática y un breve recorrido por ellos se presenta en el apartado \ref{Sec:EA:Galloping}. Por último, en la Sección \ref{Sec:EA:Corrotacional} se recorre la metodología corrotacional y los principales autores que desarrollaron esta formulación. 

\section{Historia de la temática}\label{Sec:EA:Historia}
El sistema masa resorte ha sido uno de los problemas principales abordados por la física y la matemática moderna. En particular, la aparición en escena del libro \emph{Philosophiæ naturalis principia mathematica} de Issac Newton en el 1657 revolucionó el conocimiento científico en occidente, tal es así que  un siglo y medio después, en consonancia con los avances de la termodinámica, devino en la aplicación de las principales invenciones de la revolución industrial.

El problema masa resorte no fue ajeno a las grandes eminencias científicas de la época, Brook Taylor,d'Alembert, Euler, Daniel Bernoulli aplicaron las ecuaciones diferenciales desarrolladas por Gottfried Leibniz y Newton al sistema masa resorte en los albores del siglo XVII \citep{Starossek1991}.  

Haciendo uso del problema abstracto elemental del oscilador masa resorte en 1788 Lagrange y otros autores anteriores, hallaron la solución para las vibraciones de un cable inextensible compuesto de un número finito de elementos, de masa despreciable, sometido a la acción de fuerzas externas. Posteriormente, Poisson en 1820 presentó la ecuación diferencial que debería de cumplir el sistema en el continuo, sin embargo las herramientas matemáticas analíticas desarrolladas hasta la fecha no permitían de hallar la solución general a dicha ecuación.\citep{Irvine1974}

No fue hasta 80 años mas tarde que en 1868 Routh presentó una solución exacta para un cable, también inextensible, de forma cicloidal (curva que describe un punto sobre una esfera girando a velocidad angular constante) \cite{routh1955dynamics}. En el año 1942 se logró modelar el comportamiento elástico del cable, el primero en su época fue Kloppel y Lie \citep{Kloppel1942}, a partir de esto Pugsley en 1949 determinó experimentalmente, para una relación entre la deflexión y el largo de vano entre 4 y 10 metros, desarrolló una fórmula para las frecuencias naturales de vibración (\cite{Pugsley1949}). En 1953 considerando un cable inextensible Saxon y Cahn resolvieron la expresión teórica, formulada por Poisson, de la curva catenaria para grandes deflexiones. Esto fue vital, ya que permitía calcular analíticamente los descensos máximos del vano entre dos torres \cite{Saxon1953}.

Tal es así que seguridad de las personas e integridad de los distintos elementos circundantes imprimen criterios de seguridad sobre el descenso de la línea. Actualmente la tensión del conductor durante el montaje, se ajusta de manera tal, que la altura mínima respete un valor exigido por norma. Esta imposición depende principalmente del grado de urbanización, los umbrales de contaminación magnética y la topografía del terreno.   

A pesar del avance en resultados teóricos y experimentales disponibles, las frecuencias naturales de un cable extensible, no concordaban con los de un sistema masa resorte cuando las deflexiones tendían a cero. En el año 1974 \cite{Irvine1974} halló el rango transitorio entre ambos estados, para corregir dicha discontinuidad se requiere una inclusión completa del modelo de elasticidad del cable. Su trabajo reveló la comprensión del fenómeno para cables horizontales (las cotas de sus extremos a la misma altura), para un ratio deflexión-largo del vano entre 1/8 y 0. El mismo autor Irivine extendió lo postulado para conductores con extremos desnivelados, aun bajo la hipótesis de que el peso se aplicaba perpendicular al conductor \citep{Irvine1974}.

A posteriori, el mismo investigador profundizó sobre la dinámica con extremos acelerados, obteniendo resultados experimentales para un movimiento tipo terremoto \citep{Irvine1976} y \citep{Irvine1978}. La teoría postulada por Irvine fue confirmada por Triafani en 1984 para distintos casos experimentales,  considerando variaciones espaciales en la geometría y tomando en cuenta las componentes del vector peso, colineales con el vector tangente al movimiento \cite{Triantafyllou1984}.

Autores contemporáneos estudiaron en simultaneo condiciones de borde dinámicas ejercidas por el viento. Este tipo de solicitaciones pueden inducir vibraciones y respuestas de resonancia. Los pioneros en la materia fueron Davenport y Steels (\citep{Davenport1965}) en 1965. Resultados más refinados se obtienen por Starossek (\cite{Starossek1991}) . En estas se exponen formulaciones dinámicas lineales para el movimiento de los cables sometidos a la acción del viento, mas estos trabajos no se desarrollan contemplando grandes desplazamientos ni tampoco se consideró no linealidad material. 

Ese tipo de solicitaciones revelaron el fenómeno de ``Galloping", este refiere a una respuesta de inestabilidad aeroelástica donde el movimiento del cable entra en régimen y en consonancia con las fuezas ejercidas por el viento. Teoricamente las geometrías perfectamente simétricas no inducen este tipo de fenómenos. Sin embargo, debido a la existencia de imperfecciones constrictivas y durante el montaje, el fenómeno es factible. En este caso, se genera un aporte de energía neto hacia el cable. Los primeros estudios de este tipo de respuesta se realizaron por Simu, quienes hallaron condiciones de velocidad crítica eólica en función de coeficientes experimentales, obtenidos mediante ensayos consumados en túnel de viento. \citep{Simiu1986}

Las vicisitudes del conocimiento viraron radicalmente el abordaje al problema de conductores eléctricos. El advenimiento del (\gls{MEF}) aplicado a armaduras en la década del 40 y 50 constituyó una herramienta sumamente potente e innovadora. Esto provocó que en los años venideros se desarrollaran vastas metodologías numéricas incorporando diferentes elementos y algorítmos de resolución computacional. En particular, en Italia un grupo de investigadores pertenecientes a La Universidad de Milan, aplicaron métodos numéricos a la simulación de conductores insoslayables. Un recorrido cronológico y descriptivo de los emblemáticos aportes de estos científicos se presenta a continuación en la Sección \ref{Sec:EA:AplicadasConductores}.

\section{Simulaciones numéricas aplicadas a conductores de trasmisión eléctrica}\label{Sec:EA:AplicadasConductores}
Los primeros artículos publicados en el primer lustro del corriente siglo por Di Pilatto y Martinelli estaban basados en elementos trinodales isoparamétricos. En esta metodología se asumió las hipótesis de pequeñas deformaciones unitarias, considerandose para el desarrollo no linealidades geométricas debido a grandes desplazamientos. No obstante, cuando las rotaciones de los elementos alcanzan valores significativos, estos modelos de barras presentan limitaciones para la representación y captura de la orientación del sistema. Además, este tipo de modelos presenta la debilidad de no satisfacer las condición de equilibrio dinámico para específicos tipos de balanceo.(\cite{martinelli2001numerical} y \cite{Martinelli2004}). En consonancia, estudios contemporáneos evidenciaban que la rigidez flexional y torsional toman un rol protagónico, por lo que despreciar estas magnitudes puede inducir a inestabilidades numéricas y predicciones erróneas sobre las frecuencias naturales de mayor orden. Tal y como se remarca en \cite{koh2004dynamic}.

Esta problemática fue inicialmente atacada por Di Pilato y otros en 2007. En este trabajo el cable se modelaba utilizando abordajes corrotacionales. Di Pillato presentó una formulación considerando elementos de viga tridimensionales corrotacionales, para calcular el vector de fuerzas internas e inerciales teniendo en cuenta grandes desplazamientos y rotaciones en coordenadas globales. Sin embargo, esta formulación basada en lo propuesto por (\cite{oran1973tangent}) tiene como desventaja principal que no es fiable ante grandes rotaciones locales de los nodos, como también, antes significativos incrementos angulares entre dos pasos de carga sucesivos. Consecuentemente para capturar dinámicas complejas resulta necesario e ineludible discretizar el dominio temporal y especial pequeños intervalos. Lo que conlleva a costos computacionales desmedidos.

El mismo autor y su equipo corrigieron las limitaciones relacionadas con los pequeñas rotaciones nodales al año siguiente en su trabajo: \cite{di2008corotational}.La solución consiste en localizar las coordenadas nodales en la configuración deformada utilizando el teorema de ángulos de Euler. En este marco el impedimento de grandes incrementos angulares, entre dos pasos de carga, se resuelve aplicando la metodología propuesta Simo and Vu-Quoc en \cite{simo1988dynamics}.  


Conforme las simulaciones numéricas avanzaron sobre la materia, la especificación del problema y el grado de complejidad del mismo se intensificó. Otro aspecto impulsor en el área se basaba en que los resultados experimentales en vanos largos, no reflejaban lo arrojado por el modelo predictivo para grandes desplazamientos. Dado esto, las hipótesis de no linealidad material y geométrica se fueron desvaneciendo y se publicaron resultados novedosos sobre el comportamiento no holomónico del fenómeno. Esto refiere a un modeló realista, que incorpora detalladamente las interacciones de contacto y fricción entre las diferentes hebras que conforman al conductor. Los pioneros en dicha temática fueron Papailou y Kutterer  en sus trabajos de la década del noventa \cite {Papailiou1997}  y \cite{Kutterer1992}.

Este tipo de estudios sugiere escindir la dinámica del problema en dos escenarios, ``full slip" donde las hebras se encuentran todas en deslizamiento relativo, por lo que cada una de ellas no ejerce contacto con sus hebras aledañas. El otro estado antagónico, es aquel donde no existe deslizamiento relativo entre ninguna de las partes que componen al conductor, este estado recibe el nombre de "full stick". En esta situación el conjunto se comporta como un rígido, he aquí la razón de su nomenclatura. En \cite {Papailiou1997} se establece la tensión máxima que se puede presentar en un cable, dadas determinadas condiciones de borde, para que exista deslizamiento en función del ángulo de giro. Estos resultados fueron contrastados con un análisis experimental. 


Según exponen los autores en estos trabajos, las deformaciones se traducen en momentos y fuerzas internas a cada cable que conforma al conductor. Estas se pueden vincular a la curvatura o deformación axial del conjunto. A partir de esto, se obtiene la matriz de rigidez global, derivando dichas fuerzas y momentos internos en función de la deformación y curvatura del conductor.

Esta matriz depende del estado en que se encuentre la dinámica del cable. Si el conductor se encuentra completamente bajo el régimen ``full slip" o ``full-sitck" la matriz es simétrica. No obstante, si partimos del caso ``full-stick" cuando ocurre el deslizamiento de algún cable que integra el conductor, la matriz de rigidez pierde su simetría. Consecuentemente no se le puede atribuir un potencial, esto se asocia al comportamiento no holomónico o histéresis del fenómeno. En dicho estado un modelo de viga uniforme no es aplicable.

Con el propósito de desatollar una formulación que sea capaz de representar el fenómeno computacionalmente se publicó el articulo \cite{Foti2016}. Aquí se implementa un modelo de contacto donde se desprecian las fuerzas tangenciales y axiales entre las hebras del cable. Estas hipótesis de carácter simplificadoras son estudiadas en \cite{costello1990average} y \cite{rawlins2005flexure}. Para el estudio de a los contactos radiales se asume: las superficies de contacto no se deforman debido a la interacción entre los mismos, los puntos de contacto entre cables se pueden aproximar por una linea continua, la fricción entre los cables se caracteriza a través del modelo de Coulomb y por último que la presión externa es idéntica para todos los cables de la misma capa. 

Planteando balances de fuerzas longitudinales y transversales en conjunto con la condiciones de no deslizamiento, se hallan los valores limites para la fuerza axial no lineal, para que no se produzca deslizamiento relativo.
El carácter innovador de estos trabajos se estriba en la detección y modelado sobre la pérdida de rigidez súbita que ocurre con el conductor, al producirse deslizamiento relativo al interior del elemento. Esta disminución  abrupta de rigidez puede producir mayores desplazamientos para elevados niveles de carga, esto puede intensificar o agudizar la problemática de balanceos excesivos. Estos movimientos son inminentes para determinadas condiciones atmosféricas, entre ellos las TC. Estas CD han sido objeto de estudio en los últimos 50 años por expertos en ingeniería del viento. En la siguiente Sección se presenta una somera descripción de la literatura investigada. 

\section{Tormentas convectivas}\label{Sec:EA:TormentasConvectivas}

Las TC son fenómenos atmosféricos que generan inestabilidades en el flujo debido a sus severos gradientes de temperatura y humedad. Cuando estas se ocasionan, masas de aire caliente ascienden hasta la parte superior de la nube, quedando depositado como una especie de domo o cúpula al interior de la misma. De pronto, ante un gradiente abrupto de presiones al interior de la tormenta, el domo colapsa arrastrando el aire frío que lo rodeaba por debajo. Esta corriente desciende a velocidades intensas e impacta con vehemencia sobre la superficie terrestre. Al chocar se produce una especie de anillo vorticoso que puede ser devastador con velocidades de hasta 270 km/h \textcite{fujita1985downburst}. En este trabajo se establecen escalas espaciales entre $40$ m  y $4$ km. No obstante recientes estudios plantean que se explayan en un diámetro entre 1 y 5 km \textcite{darwish2010dynamic}.

Para determinar las cargas de viento, sobre los elementos de trasmisión eléctrica, ciertas normativas se estriban en perfiles de vientos clásicos (sinópticos)tipo capa límite atmosférica. Esto se traduce en una subestimación de las presiones que se ejercen sobre la línea, un caso ejemplar es la norma \gls{IEC} 60826. Esto pone en riesgo al sistema es atacado por tornados o CD. La probabilidad de ocurrencia es baja para dominios de corta longitud, pero cuando las lineas discurren largas distancias estos vientos extremos suelen suceder esporádicamente \textcite{ang1984probability}. 

La altura de velocidad máxima es un variable crucial para el estudio de daños vinculado a este tipo de fenómenos. Según expresan investigadores contemporáneos el diámetro de desarrollo del anillo se encuentra intrínsecamente relacionadas con dicha altura  \textcite{holmes2002re}, \cite{abd2013coupled}. Complementando a esto, \textcite{stengel2017measurements} en Alemania capturó este fenómenos utilizando anemómetros colocados en lineas de trasmisión. Esto permitió establecer un perfil de velocidades media y la función de coherencia relacionada con la turbulencia a partir de datos experimentales. De este artículo se extrajo el perfil de vientos implementado en este trabajo.

En nuestro país investigadores integrantes del Grupo de Eolo Dinámica perteneciente a la Facultad de Ingeniería extrajeron datos durante TC trabajo de campo exhaustivo. El primer informe relevado en el articulo \cite{duranona2009analysis} se realiza un calculo del angulo de balanceo, simplificando cauasi-estáticamente que la tangente del mismo es igual al ratio de la fuerza de viento por unidad de peso. En este trabajo se mostró que para valores de velocidad de viento de 97.9 m/s el conductor alcanza los 85º.

 Dados los alarmantes resultados de \cite{duranona2009analysis} posteriormente se realizaron investigaciones con datos de hace un siglo  hasta la fecha en el trabajo \citep{duranona2015significance}. En este estudio se atisba que fenómenos de CD producen mayores velocidades de ráfaga en 10 minuto que los vientos tipo capa límite atmosférica. El valor máximo de velocidad registrado alcanzó los $40$ m/s en promedio de $10$ minutos. En el año 2019, este grupo de investigadores presentó un trabajo relevante donde se resalta que los vientos extremos afecta principalmente al norte del país \textcite{duranona2019first}. En este se sugiere que la norma (\gls{UNIT}:50-84, 1984) debe ser actualizada incluyendo cálculos de cargas por fenómenos de vientos no sinópticos. Pero los eventos de vientos extremos no son los únicos que afectan a los conductores, también pueden ocurrir inestabilidades estructurales inherentes a interacción entre fluido-estructura. 

\section{Análisis semi-analíticos de conductores}\label{Sec:EA:Galloping}

Los cables suspendidos en sus extremos e inmersos en un flujo de aire pueden experimentar oscilaciones aeroelásticas autoexcitadas de gran amplitud, principalmente en el plano vertical. Esta problemática ha sido ampliamente estudiada por distintos autores de la literatura. Como por ejemplo \cite{blevins1990van}, \cite{jones1992coupled}. Para vigas de gran esbeltez, o elementos de cuerdas tensados en sus bordes, se han aplicado formulaciones tanto lineales como no lineales.  En estos trabajos se implementaron elementos de uno o dos grados de libertad por nodo. Los objetivos de estas publicaciones consisten en abordar analíticamente el fenómeno de Galloping, examinando la relación intrínseca entre el movimiento vertical y horizontal y verificar estos resultados en la práctica. Algunos de ellos, estudiaron el efecto de perfiles geométricos sin simetría tangencial, debido a formaciones de escarcha o  hielo. En la temática destaca el trabajo \textcite{chabart1998galloping}, en este se propuso una aproximación innovadora teniendo en cuenta aspectos complejos del fenómeno como ser: la variación de ángulo de ataque durante la trayectoria y sus consecuencias en la fuerza lift ante la presencia de excentricidades geométricas. 

El fenómeno Galloping presenta las frecuencias del movimiento excesivo suelen ser bajas y son exuberantes a simple vista. Este fenónmeno devastador tiene consecuencias severas sobre todo en lineas que se encuentran en clímas gélidos, recientemente en Julio del 2020 derribó 55 torres sólidas en el sur de Argentina y las imágenes son impactantes \href{blob:https://www.clarin.com/df740f0f-cd9a-4b9a-8c55-c594c9ea262c}{(Ver vídeo)}. La principal causa del fenómeno es el ataque de vientos intensos y constantes. La presencia de irregularidades geométricas en las lineas induce inestabilidades aerodinámicas y cuanto mayor sea la cantidad y discontinuidad de las excentricidades más aguda será la respuesta inducida. Las velocidades requeridas de viento suelen ser mayor a $7$ m/s y las frecuencias de respuesta del conductor suelen oscilar entre los $ 0.15$ y $1$ Hz.

 Existen determinados componentes que pueden mitigar la inminente aproximación de las lineas, y por tanto la aparición de un cortocircuito. Los separadores si bien no evitan los desmedidos desplazamientos globales, si los relativos entre conductores, siendo una solución atenuante del problema. Otros elementos se han creado para suprimir el fenómeno en conductores propensos a la formación de hielo. Estos son amortiguadores de torsión. Este dispositivo en inglés (Torsional Damper Detuner) gira relativo al conductor anulando las formas irregulares producto de la formación de hielo.  

En el artículo \cite{jones1992coupled} se halló la solución a la ecuación de movimiento, despreciándose su componente axial. Bajo esta hipótesis, se presentaron los autovalores que permiten detectar analíticamente bajo que condiciones del sistema se efectiviza la inestabilidad. De manera complementaria, se desarrolló el estudio matemático de las trayectorias que describían las líneas, deduciéndose un perfil tipo helicoidal con una componente vertical significativamente mayor a la horizontal. Esto indica la potencial amenaza respecto a los excesivos e indeseables desplazamientos que el Galloping es capaz de generar en el eje vertical. Esto amenaza la seguridad y fiabilidad del sistema ya que esta componente, es limitada durante la instalación a través de cálculos estáticos. Al generarse desplazamiento dinámicos desmedidos, ya no hay garantías de salvaguardar la salud de las personas y los componentes cercanos. 

Los estudios de Jones y Blevins,  se fraguaban en premisas de linealidad geométrica. Sin embargo, autores han destacado que las efectos no lineales juegan un rol importante en el desarrollo, como ser: las referencias \cite{luongo1984planar} y \cite{lee1992nonlinear}. En el trabajo propuesto por Lee se incluyen componentes no lineales de tercer y cuarto orden en el estiramiento del conductor durante el movimiento. Se cotejan estos resultados con los de un modelo lineal de primer orden, concluyéndose que los términos de segundo y tercer orden influyen notoriamente en la respuesta al integrarse numericamente la ecuación diferencial del movimiento. 

Esta problemática fue abordada unos años mas tarde, por el trabajo publicado \cite{luongo1998non}. En este artículo se hallaron las soluciones no lineales de resonancia desencadenadas por un flujo transversal uniforme. 
Se contrastaron dos soluciones arrojadas por disimiles modelos, uno de pequeños desplazamientos y otro incorporando no linealidades geométricas. En este trabajo se distinguen dos régimes del movimiento, el primero de ellos nominado crítico refiere a valores de velocidad cercana a la crítica donde los movimientos no presetan gran amplitud. Al aumentar la velocidad de viento, las trayectorias se amplifican y el régimen es llamado post-crítico. De este análisis, se concluye que la solución para pequeños desplazamientos es simple y confiable para valores de velocidad media de viento correspondiente al estado crítico. Posteriormente al incrementar la velocidad de viento y se desata el fenómeno post-crítico y el incluir términos de grandes desplazamientos es imprescindible para representar cabalmente las trayectorias. Sin embargo, para perfiles simétricos, la velocidad crítica que lo origina puede ser hallada con un análisis lineal.

Según los autores del trabajo \cite{luongo2007linear}, hasta la fecha de publicación, era necesaria una formulación orientada al modelado no lineal de la dinámica del problema. En numerosos trabajos publicados, se calculaban las fuerzas en su régimen cuasi estacionario y los desarrollos en elementos finitos aplicados eran exiguos, en espacial para el régimen post-critico del Galloping. Por otra parte, escasos estudios consideraban las variaciones de angulo de ataque y velocidad relativa entre el conductor y del fuljo. Además eran despreciadas las rigideces a torsión del los elementos, estos se debe a que la rigidez según el eje axial suele ser mayor respecto a la rigidez felxional, principalmente por un argumento de esbeltez y disposición geométrica del conductor de estudio.  

El propósito de \cite{luongo2007linear} fue proponer un elemento de viga orientado a la simulación del cable, capaz de incorporar la rigidez de este a torsión. Estos términos representan diferencias notorias para secciones antisimétricas en los modos de respuesta. Por otra parte, se presentaron resultados numéricos utilizando el método de Galerkin para un caso simple con el objetivo de hallar las condiciones de inestabilidad incipiente. Se demostró, que el ángulo de balanceo es capaz de influir considerablemente en las condiciones críticas del sistema, a través de la matriz tangente, cuando se tienen en cuenta los modos simétricos. En particular, para valores pequeños de balanceo, la inclusión del angulo puede influir significativamente en el valor de velocidades críticas aeroelásticas.

 A psoterirí, en el trabajo \cite{luongo2009effect} se profundizó en los efectos del angulo de balanceo en la dinámica del fenómeno. Para esto se utilizó la formulación de vigas propuesta por los mismos autores dos años antes, como destacado resultado, se probo que mientras la rigidez de torsional no afecta significativamente los desplazamientos traslacionales, en cortaste si lo hace a la solución del angulo de giro. En especial para perfiles sin simetría de revolución. La consideración del balanceo en el lift y en el ángulo de ataque, afecta notoriamente las frecuencias naturales del cable, en particular las propiedades de la sección aerodinámica y por tanto su velocidades críticas. Por ende, se resalta la importancia de incorporar un modelo robusto y completo de vigas para el modelado del conductor, como ser un modelo de vigas corrotacional.



\section{Análisis corrotacional de vigas}\label{Sec:EA:Corrotacional}
Los modelos de vigas flexibles se utilizan en un amplio abanico de aplicaciones entre ellas: aeronaves, turbinas propulsoras, molinos eólicos marítimos y terrestres. A pesar de las formulaciones `` Updated " y ``Total Lagrangian" clásicas,  dentro de estas últimas el abordaje corrotacional es idóneo para este tipo de aplicaciones. Esto se fundamenta en la necesidad de incluir términos de no linealidad geométrica generados por los grandes desplazamientos den servicio. Destacados autores han contribuido al desarrollo histórico de esta metodología en las últimas décadas, entre ellos el emblemático trabajo de \cite{Nour-Omid1991} quienes sentaron las bases del método. 

Este modelado se funda principalmente en la descomposición cinemática del elemento finito en dos etapas sucesivas. Primeramente considerándolo como un rígido y luego incluyendo su carácter deformable. Para ubicar la componente rígida, se considera un sistema de coordenadas solidario que permite localizar al elemento en el espacio.Mientras que 0para la componente deformable se considera una formulación local esfuerzo-deformación, con su respectivo sistema de coordenadas, específica para cada material. La principal ventaja de la propuesta corrotacional es la versatilidad ante diferentes formulaciones locales. Permitiendo incorporar distintos tipos de elementos, fácilmente. Además, destaca el desacople de las no linealidades. La componente rígida del elemento representa términos de no linealidades geométricas mientras que la deformables incorpora no linealidad materiales. 

El cálculo de las matrices tangentes y los vectores de fuerzas internas se calculan en función de la fragmentación cinemática antes descrita. La variación de la componente rígida respecto al desplazamiento, resulta una matriz tangente anti-simétrica. La deducción consistente de la formulación conduce a esta propiedad anti-simétrica, esta característica depende principalmente del des-balanceo en el vector de fuerzas residuales. Representar las propiedades anti-simétricas de la matriz puede implicar grandes costos computacionales al resolver el sistema mediante métodos numéricos como (\gls{N-R}). Los autores \cite{Nour-Omid1991} con el objetivo de optimizar el método, demostraron que simetrizando la matriz tangente, N-R mantiene su orden de convergencia cuadrático.

Debido a voluble capacidad de la metodología corrotacional, en los años posteriores se publicaron numerosos trabajos aplicando diversos tipos de elementos y leyes materiales. La mayor cantidad de los trabajos se ciñeron al considerar funciones de interpolaciones lineales, matrices de masas concentrada y  elementos de viga de Timoshenko. Para estos elementos, es posible obtener de manera sencilla la matriz de masa al derivar los términos de fuerzas inerciales. Como habrá notado el sagaz lector, este cálculo conduce ineludiblemente a la matriz de masa constante de Timoshenko. 
Por otra parte, interpolaciones lineales asumen que los desplazamientos transversales al eje de la viga son nulos, esta hipótesis reduce el campo de aplicación del modelo, en especial para mallas de bajo numero de elementos, ya que la matriz de masa tangente y el vector de fuerzas inerciales no representan las componentes omitidas. 

En la referencia \cite{Crisfield} se sugiere que el proceso de obtención requerido para el cálculo de la matriz de masa concentrada es demasiado intrincado, debido a su grado de complejidad geométrico. El autor propone utilizar funciones de interpretación cúbicas, como por ejemplo las asociadas al elemento de Bernoulli. Este tipo de soluciones resultan controversiales a la hora de derivar el vector de fuerzas inerciales. Como consecuencia, el autor consideró un modelo simplificado híbrido. Este consiste en utilizar interpolaciones cúbicas para el vector de fuerzas internas y matriz tangente, considerando una matriz de masa constante. Esto resulta en una formulación no consistente pero numéricamente eficiente.  Esta forma de proceder también se aplico en \cite{pacoste1997beam}.

En paralelo otros autores, desarrollaron eficientes elementos de viga bidimensionales y tridimensionales, con el propósito de modelar estructuras en grandes desplazamientos bajo cargas estáticas (\cite{Battini2002} \cite{alsafadie2010corotational}).
Estos autores afirman que al seleccionar adecuadamente el largo de elemento, los desplazamientos locales son significativamente menores que los asociados a la componente rígida. Por esta razón, se compararon resultados con diferentes número y tipos de elementos para los mismos ejemplos. Estos estudios, en conjunto con lo publicado por \cite{alsafadie2010corotational}, concluyen que formulaciones cúbicas son más eficaces y precisas que las lineales bajo ciertas circunstancias. Estos trabajos sentaron las bases para la extensión analítica hacia las componentes dinámicas.

Investigadores de origen europeo trabajaron en este desafío en los últimos años. El primero de ellos fue \cite{behdinan1998co} a finales de siglo, pero las funciones de forma utilizadas para describir los desplazamientos globales no eran consistente con la formulación canónica del método corrotacional propuesta por \textcite{simo1988dynamics}. De hecho, según el conocimiento del autor, no existía hasta la fecha ninguna investigación publicada sobre una formulación consistente que derivara analíticamente, no solo los vectores de fuerza interna sino también, las componentes inerciales.   

Años mas tarde, \cite{Le2011} publicaron una formulación para vigas 2D implementando funciones de forma cúbicas del elemento de interpolación independiente "IIE" de la referencia \cite{reddy1997locking}. Estos elementos fueron desarrollados con el objetivo de obtener el vector de fuerzas inerciales y la matriz tangente fácilmente. Estas funciones de forma son una leve modificación basadas en los polinomios de Hermitian, con el propósito de incluir consideraciones adicionales sobre las deformaciones por flexión y cortante. Esta publicación es una de las primeras en obtener el vector fuerzas inerciales matemáticamente y su matriz respectiva de masa tangente. Para este cálculo, se introducen algunas aproximaciones con respecto a las cantidades cinemáticas locales. Además se comparan los resultados con respecto a las clásicas aproximaciones de la literatura, matriz de masa concentrada y de Timoshenko. Se concluyó que esta nueva formulación, con respecto a los dos enfoques clásicos,  permite reducir significativamente el número de elementos. Esta ventaja se debe a una mayor precisión en los términos inerciales y sus cambios temporales en función de los desplazamientos locales.    

Los mismos autores en conjunto con Lee extendieron la formulación en su trabajo del 2014 \cite{Le2014} agregando una dimensión, este desarrollo se vio dificultado debido a la carencia de propiedades como aditividad y conmutativiad en las matrices de rotación. Estas desempeñan un rol indispensables a la hora de caracterizar la cinemática angular del planteo. En este artículo, se presenta la parte estática desarrollada por Battini en \cite{Battini2002}, además de exponerse detalladamente la obtención del vector de fuerzas inerciales y su derivada. Asumiendo determinadas simplificaciones para las deformaciones angulares locales. Con respecto a la iteración temporal se selecciono el clásico método  (\gls{HHT}) con los parámetros convencionales \citep{hilber1977improved}. Este algoritmo es utilizado por reconocidos software comerciales (Abaqus,Lusas) e implica una disipación sobre la energía total del sistema para frecuencias de oscilación altas, mas presenta como ventaja la estabilidad para grandes incrementos temporales. 

En \cite{Le2014} se consideraron cuatro ejemplos numéricos para comparar la nueva formulación con otros dos enfoques. La primer comparación, se deriva de la nueva formulación reemplazando las intercalaciones cúbicas por lineales. El segundo enfoque es el TL clásico propuesto por \cite{simo1988dynamics}. En base a estos ejemplos de contraste se concluyen las siguientes afirmaciones: todas las formulaciones conducen a idénticos resultados refinando las mallas, no así con mayados gruesos. En este caso tanto la formulación bi-nodal de Simo y Vu-Quoc como la lineal corrotacional son significativamente mas imprecisas en comparación con la formulación cúbica corrotacional. Esto justifica el esfuerzo computacional y analítico en los términos dinámicos inerciales incluidos en el modelo. La formulación corrotacional es ligeramente mas lento ($12\%$) respecto a lo descrito por Simo and Vu-Quoc . Sin embargo, bajo ciertas condiciones altamente dinámicas, para un mismo nivel de precisión exigido, la formulación innovadora de este trabajo lo logra en menor tiempo.  

Debido a estas ventajas, esta metodología es implementada en diversos campos de aplicación ingenieril. La robustez, solidez y versatilidad del modelo es un atractivo para distintos investigadores del área. En \cite{albino2018co} Albino modelaron tuberías elevadoras flexibles, manufacturadas por materiales graduados,  para la carga o descarga de barcos petroleros en alta mar. En 2019 \cite{asadi2019multibody} simularon palas de aerogeneradores utilizando elementos de viga para el diseño de las componentes mecánicas, entre ellas el tren de trasmisión, los cojinetes y la soldadura de la raíz cuchilla-pala. En el mismo año el autor \cite{barzanooni2018modeling} atacó la problemática de anillos y interacciones de contacto aplicado a robots industriales también con la formulación propuesta por \cite{Le2014}.

Esto nos permite concluir que la formulación es idónea para la aplicación central de este trabajo. Donde se desarrollan grandes desplazamientos y términos inerciales. Estudios recientes se encuentran desarrollando softwares para ser aplicados a diferentes problemáticas de la ingeniería estructural y mecánica. No obstante, ningún software comercial hasta la fecha utiliza formulaciones corrotacionales para la solución de problemas dinámicos. 

