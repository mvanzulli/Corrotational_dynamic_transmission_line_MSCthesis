\chapter{Conclusiones}\label{Cap:Conlcusiones}
\linenumbers

El presente capítulo puede separarse en tres secciones que se relacionan con diferentes aristas o perspectivas del trabajo llevado a cabo. En primera instancia, se detallan las consideraciones finales y de síntesis, desde un punto de vista técnico sobre los resultados obtenidos. Posteriormente, se narran los aspectos del desarrollo académico de esta tesis como trabajo culmine dentro de una etapa formativa fundamental para quien escribe. Luego de esto, se realizan recomendaciones y posibles trabajos a futuro para finalizar con una reflexión sobre las limitaciones críticas de este trabajo y el método científico en general. 

\section{Conclusiones técnicas}

\subsection{Sobre el fenómeno}
Según la bibliografía consultada hay vasta evidencia de que el fenómeno de tormentas convectivas ha afectado severamente la calidad e integridad de vida a lo largo y ancho del globo terráqueo. En particular, debido condiciones climáticas singulares de la región, y el progresivo calentamiento global, han intensificado los daño devastadores en los sistemas de trasmisión y distribución eléctrica nacionales. Induciendo inevitablemente en costos millonarios de reparación sobre las instalaciones y mas las perdidas de ganancias durante interrupción del suministro. Además, estos eventos extremos se manifiestan en corrientes descendentes o tornados extra-tropicales que han puesto en peligro la salud y condiciones de vida de las personas. 

A partir de las bibliografías consultadas y los resultados del  ejemplo \ref{Sec:RN:TransmissionSystem}posible teorizar que la mayoría de las incidencias ocurridas en las líneas Palmar-Montevideo de 500kV pueden deberse al pasaje de tormentas severas sobre la zona. Estas tormentas producen corrientes descendentes que ejercen cargas desmesuradas sobre el conductor, en el orden de minutos, imponiendo ángulos de balanceo excesivos que acercarían los conductores a las torres a una distancia tal que inminentes descarga a tierra pueden sacar del serivcio a la linea. Además según los estudios, el diseño de sistemas de trasmisión considerando flujos tipo capa límite atmosférica poliandria estar subdimensionando ya que los periodos de retorno para velocidades de hasta 100 km/h es menor para corrientes descendentes respecto de vientos capa límite atmosférica. 

Dada la problemática esta investigación la atacó  generando herramientas de simulación computacional, capaces de emular los desmedidos desplazamientos y esfuerzos que estos eventos producen sobre los sistemas de trasmisión eléctrica de alta tensión. Para esto, inicialmente se consulto el estado del arte desde un foco de ingeniería del viento y estructural. Se analizaron bibliografías en materia de simulaciones numéricas aplicadas a conducentes eléctricos, con abordajes semi analíticos y computacionales. También, se estudiaron trabajos nacionales e internacionales, desde un punto de vista cualitativo y experimental de corrientes descendentes y sus posibles perjucios  en lineas de trasmisión eléctrica.  Asimismo, el autor se interiorizó y eligió la formulación corrotacional de vigas 3D. Una vez ahondado en la temática, se implementó y validó un modelo corrotacional consistente robusto y eficaz capaz de captar y reproducir desplazamientos de gran amplitud con numero reducido de elementos.


\subsection{Sobre los resultados}

Esta formulación se valido con el ejemplo \ref{Sec:RN:RightAngle} benchmark del folclore corrotacional presentado por \cite{simo1988dynamics}. Este es cargado con una fuerza abrupta y de severa magnitud, respecot al  rigidez de la estructura alcanzando un valor de 50 $N$ en apenas 2 segundos de simulación, tal y como se muestra en la Figura \ref{fig:RN:RA:Force}. Esta fuerza pose una esencia análoga al fenómeno de tormentas convectivas per se. Esta fuerza aumenta estrepitosamente en un corto lapso de tiempo, por ende la capacidad del modelo de reproducir este tipo de impactos es fundamental para poder emular el fenómeno central de este trabajo en simulaciones de sistemas eléctricos.

 En la Figura \ref{fig:RN:RA:Dispz} se observan amplitudes que alcanzan las 8 metros cuando la estructura mide 10. Esto evidencia, la fuerte presencia de grandes desplazamientos y rotaciones. Asimismo, en la dirección $z$, se puede observar el carácter no conservativo de la formulación corrotacional, ya que los valles y crestas de las respuesta prestan una tendencia decreciente con el tiempo. En relación con los desplazamientos en el sentido de $y$ del nodo A, presentados en la Figura \ref{fig:RN:RA:DispyA}, se observa el singo negativo de este, concordando con lo esperado intuitivamente según el sentido de la fuerza aplicada. Por último, el resultado mas importante de este ejemplo se destila al cotejar las respuestas del as Figuras \ref{fig:RN:RA:DispyA}, \ref{fig:RN:RA:DispzA} y \ref{fig:RN:RA:DispzB} con lo publicado por le articulo de referencia \citep{Le2014}. Al comparar estas figuras se concluye que el modelo implementado es capaz de representar cabalmente movimientos de gran amplitud, con apenas 10 elementos por miembro y unas paso temporal de 0.25 $s$. Esto permitió validar la formulación para este ejemplo y aplicarla a dominios mas complejos específicamente con el foco en el modelado de conductores eléctricos. 

 Como primer ejemplo aplicado al modelado de conductores se eligió un problema postulado en la publicación \citep{Foti2016}. Para esto, se investigó la normativa IEC 60815 que detalla propiedades geométricas y constructivas de conductores para alta y media tensión.  Con el fin de cotejar fielmente los resultados obtenidos, se extrajeron, tanto los parámetros del flujo, como las propiedades geométricas y materiales, del trabajo de referencia correspondientes con un conductor DRAKE ASCR 7/26. No obstante, con el objetivo acercar la representación a la real, se incorporaron dos elementos aisladores ilustrativos, que por sus condiciones de borde, no afectan el comportamiento dinámico y cinemático del problema. (Ver Figura \ref{fig:RN:FotiCable:Ilustracion})
 
 Para este ejemplo de la Sección \ref{Sec:RN:FotiCable},se aplicó un viento progresivo desde un valor nulo hasta una velocidad de un perfil Capa límite atmosférica en 20 segundos, según la Figura \ref{fig:RN:FotiCable:VelocidadCable}. Este cálculo se realizó considerando las propiedades extraídas del\citep{euroCode}, explicitadas en la Tabla \ref{table:RN:FotiCable:propiedadesCable}. Al espejar los perfiles de velocidad presentados en las Figuras \ref{fig:RN:FotiCable:DispY} y \ref{fig:RN:FotiCable:DispZ} con las fuerzas aplicadas de la Figura \ref{fig:RN:FotiCable:FuerzaZ} se observa una homología. Esto se fundamenta con un análisis de Foruier donde las salidas y entrdas son los desplazamientos y fuerzas respectivamente. 
 
 Las contribuciones principales del Ejemplo \ref{Sec:RN:FotiCable} se desprenden al contrastar los resultados del ángulo $\Phi$, gratificado en la Figura \ref{fig:RN:FotiCable:Angulo} con los presentados por L. Foti. De este análisis se extraen ciertos paralelismos y discordancias. En primer lugar las perfiles arrojados son semejantes, presentando un relación cuadrática con la velocidad. Esto se atribuye a la relación de dependencia cuadrática de la fuerza con la velocidad media de viento. Sin embargo, para el caso implementado en esta tesis se alcanzan mayores valores de ángulo. Esto puede deberse las múltiples diferencias entre los modelos: la omisión de las componentes turbulentas del flujo, el estado inicial de tensado y la presencia de hielo en las lineas. Los últimos dos factores parecen intuitivamente atenuar el angulo máximo alcanzado por la linea durante el transcurso del movimiento, por su mayor rigidez inicial e inercial. Dado estos resultados, se decidió llevarlas simulaciones a un grado mayor de complejidad, e implementar un modelo con múltiples elementos simulando un sistema de trasmisión eléctrica.  
 
 El ejemplo descrito en la sección \ref{Sec:RN:TransmissionSystem} es el resultado principal de este trabajo. Se acoplaron diferentes componentes de un sistema de alta tensión conductores, aisladores y torres. Para esto se validaron ejemplos intermedios integrando resultados lineales e inerciales conocidos con elementos de biela tipo Green y de viga corrotacional. Las geometrías y propiedades que integraron el modelo son extraídas de bibliografías experimentales y normativas buscando representar y emular el fenómeno de forma realista. 

 De igual modo, el perfil de viento se extrajo de estudios experimentales en el Norte de Alemania durante el transcurso de una tormenta convectiva tipo corriente descendente publicado en \citep{stengel2017measurements}. Esta es de una magnitud intensa, aunque no en comparación con los resultados medidos en diferentes estudios nacionales como \citep{duranona2009analysis} y el trabajo de \cite{duranona2019first} donde se alcanzan umbrales de 88.2 a 162 km/h a 45 m de altura. Otra diferencia respecto se encuentra en el grdiente de velcoidad, el flujo introducido numericamente de Stangel posee una menor aceleración. 
 
 La carga del viento se distribuyo en el primer vano provocando un perfil que ataque diferente a la linea en su coordenada axial. Esto genere un efecto de desfazaje entre los desplazamientos en los vanos. Esta propiedad del flujo busca representar un fenómeno de osicalcion axial relacoinado con las presencias de vórtices del flujo a lo largo del espacio. Las diferencias de los desplazamientos de los puntos de la cadena aisladora se evidencia en las Figuras \ref{fig:RN:Transmission:DispsCB} y \ref{fig:RN:Transmission:DispsAD}. Por mas que los movimientos posean diferentes amplitudes de banda, los perfiles obtenidos se encuentran infinitamente emparentados con el perfil de la tormenta de la Figura \ref{fig:RN:Transmission:ForceVelTormenta}.  
 
 Finalmente se creó un análisis de emparejamiento con dos modelos ampliamente utilizados para calcular de forma cuasiestaitca utilizando el modelo del arctoangente basado en un péndulo sin términos dinámicos e inerciales. Los trabajos de \cite{stengel2017measurements}, \cite{duranona2009analysis} y \cite{yan2009numerical} se aplica esta aproximación simplifcatoria. Si bien en los resultados del Ejemplo \ref{Sec:RN:TransmissionSystem} no son comprables, ya que la aproximación plana no funciona. Para este caso en particular, la curva numérica parece reflejar una linealidad que arrojaría un resultado de sobrestimado sobre el ángulo de aproximación a la torre según la curva carga desplaziamiento \ref{fig:RN:Trnsmission:CurvaCargaDisp}   

% Esta formulación se aplico específicamente a conductores de alta tensión sometidos perfiles de viento extraídos de artículos recientes aplicados a tormentas convectivas. Las respuestas del sistema evidencian el balanceo excesivo del conductor \ref{fig:DeformadasEqual}, ante este tipo de solicitaciones, los códigos generados pueden gestar una herramienta de análisis complementario para el diseño de sistemas de trasmisión eléctrica. Al vincular 		\ref{fig:CableDispY} y \ref{fig:CableFuerzaZ} se evidencian la idéntica forma que desarrollan ambo perfiles colmando las expectativas sobre dicha salida.


\section{Conclusiones de formación}
El desarrollo de este trabajo constituyó una instancia de formación fundamental y enriquecedora para el autor enmarcada dentro del programa de Magister en Ingeniería Estructural. Este documento es la síntesis y aplicación de un conjunto de conocimientos profundizados durante la actividad programada, aplicada al modelado numérico de estructuras. Desde la óptica del autor, la creación de herramientas endogenas con foco en atacar problemáticas a nivel nacional constituye un pilar fundamental en el desarrollo autónomo y original de la ingeniería uruguaya. Este trabajo es una muestra de la convicción y determinación, que el conocimiento académico, debe desarrollarse de forma transparente, comunitaria y democrática. Es por esto, que todos los códigos utilizados en esta investigación se implementaron en el software libre \href{https://github.com/ONSAS/ONSAS/}{ONSAS}. Esto abre la posibilidad a cualquier tercero ya sea una organización o persona de estudiar, modificar y difundir los códigos creados como también aplicarlos a sus propias necesidades. 

\section{Limitaciones }




\section{Trabajos a futuro}



%Con respecto a la norma se esclarecieron las metodologías propuestas para el diseño de lineas de trasmisión,  se corroboraron los valores supuestos por la norma con diferentes referencias, algunas expuestas en el curso, y otras investigadas. Se destaca como debilidad que para el diseño que esta no considera eventos de vientos no sinópticos, estos pueden vulnerar al sistema, como se observaron en distintas bibliografías, solo se consideran viento tipo capa límite atmosférica. 


%Para el análisis del  problema en particular, se creó una librería que contiene un conjunto de códigos que, mediante el método de elementos finitos e iteraciones de Newton-Raphson, resuelve dicho problema para diversas condiciones de borde.  Se concluye que este es capaz de reproducir de forma adecuada el balanceo del cable en contraste con \cite{Stengel2017}. Las desviaciones entre los 230 y 500 segundos pueden estar asociadas a errores durante la toma de datos del perfil de velocidades y/o términos no inerciales de magnitud apreciable, en este período no sería válido aplicar un modelo estático como se realiza en \cite{Stengel2017}. El software permite obtener los modos normales, y plotear sus modos asociados sobre la configuración indeformada, además se generaron vídeos para cada uno de estos modos los cuales se adjuntan en la carpeta del código. Los modos de vibración constituyen un importante resultado a contrastar con las frecuencias predominantes en el flujo, para el ejemplo se encuentran alejadas en al menos un orden de magnitud, sin embrago otras geometrías podrían producir resonancias dinámicas, aumentándose significativamente la amplitud de oscilación. 

%Como trabajo a futuro se debería incluir un modelo de vigas a flexión,  dinámico y no lineal que represente completamente el cambio de rigidez e inercia a lo largo de la trayectoria del conductor. En relación a esto un modelo correccional presentado en las referencias \cite{Le2014} y \cite{Le2011} se ajusta a las necesidades.