\chapter{Consideraciones finales}
\linenumbers
Se implementó y validó un modelo corrotacional consistente robusto y eficaz que es capaz de captar y reproducir desplazamientos de gran amplitud con numero reducido de elementos.  Esta formulación se aplico específicamente a conductores de alta tensión sometidos perfiles de viento extraídos de artículos recientes aplicados a tormentas convectivas. Las respuestas del sistema evidencian el balanceo excesivo del conductor \ref{fig:DeformadasEqual}, ante este tipo de solicitaciones, los códigos generados pueden gestar una herramienta de análisis complementario para el diseño de sistemas de trasmisión eléctrica. Al vincular 		\ref{fig:CableDispY} y \ref{fig:CableFuerzaZ} se evidencian la idéntica forma que desarrollan ambo perfiles colmando las expectativas sobre dicha salida. Como trabajos a futuro se debería verificar el no deslizamiento de las hebras internas según lo publicado en \citet{foti2018finite}. Este comportamiento de histéresis depende principalmente de las fuerzas normales al interior del cable, esto es imprescindible para asegurarse del correcto modelado del conductor como un solido circular. Como eventuales trabajos a futuros se propone la implantación de un modelo acoplado con la torre donde no se desprecien lo desplazamientos del punto de anclaje y las frecuencias de resonancia que este cambio pueda implicar. Por ultimo es oportuno mencionar la potencialidad de este trabajo para desarollar un solver integrado entre los softwares ONSAS-CAFFA \cite{usera2008parallel}.


