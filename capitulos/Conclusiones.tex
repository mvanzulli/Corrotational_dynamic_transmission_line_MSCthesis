\chapter{Conclusiones}\label{Cap:Conlcusiones}
\linenumbers

El presente capítulo puede separarse en tres secciones que se relacionan con diferentes aristas o perspectivas del trabajo llevado a cabo. En primera instancia, se detallan las consideraciones finales y de síntesis, desde un punto de vista técnico sobre los resultados obtenidos. Posteriormente, se narran los aspectos del desarrollo académico de esta tesis como trabajo culmine dentro de una etapa formativa fundamental para quien escribe. Luego de esto, se realizan recomendaciones y posibles trabajos a futuro para finalizar con una reflexión sobre las limitaciones críticas de este trabajo y el método científico en general. 

\section{Conclusiones técnicas}

\subsection{Sobre el fenómeno}
Según la bibliografía consultada hay vasta evidencia de que el fenómeno de tormentas convectivas ha afectado severamente la calidad e integridad de vida a lo largo y ancho del globo terráqueo. En particular, debido condiciones climáticas singulares de la región, y el progresivo calentamiento global, han intensificado los daño devastadores en los sistemas de trasmisión y distribución eléctrica nacionales. Induciendo inevitablemente en costos millonarios de reparación sobre las instalaciones y mas las perdidas de ganancias durante interrupción del suministro. Además, estos eventos extremos se manifiestan en corrientes descendentes o tornados extra-tropicales que han puesto en peligro la salud y condiciones de vida de las personas. 

A partir de las bibliografías consultadas y los resultados del  ejemplo \ref{Sec:RN:TransmissionSystem}posible teorizar que la mayoría de las incidencias ocurridas en las líneas Palmar-Montevideo de 500kV pueden deberse al pasaje de tormentas severas sobre la zona. Estas tormentas producen corrientes descendentes que ejercen cargas desmesuradas sobre el conductor, en el orden de minutos, imponiendo ángulos de balanceo excesivos que acercarían los conductores a las torres a una distancia tal que inminentes descarga a tierra pueden sacar del serivcio a la linea. Además según los estudios, el diseño de sistemas de trasmisión considerando flujos tipo capa límite atmosférica poliandria estar subdimensionando ya que los periodos de retorno para velocidades de hasta 100 km/h es menor para corrientes descendentes respecto de vientos capa límite atmosférica. 

Dada la problemática esta investigación la atacó  generando herramientas de simulación computacional, capaces de emular los desmedidos desplazamientos y esfuerzos que estos eventos producen sobre los sistemas de trasmisión eléctrica de alta tensión. Para esto, inicialmente se consulto el estado del arte desde un foco de ingeniería del viento y estructural. Se analizaron bibliografías en materia de simulaciones numéricas aplicadas a conducentes eléctricos, con abordajes semi analíticos y computacionales. También, se estudiaron trabajos nacionales e internacionales, desde un punto de vista cualitativo y experimental de corrientes descendentes y sus posibles perjucios en lineas de trasmisión eléctrica.  Asimismo, el autor se interiorizó y eligió la formulación corrotacional de vigas 3D. Una vez ahondado en la temática, se implementó y validó un modelo corrotacional consistente robusto y eficaz capaz de captar y reproducir desplazamientos de gran amplitud con numero reducido de elementos.

\subsection{Sobre la metodología}
En la Sección \ref{Sec:PRE:Modeloviento} se desarrolló un estudio general sobre los campos de velocidades absolutos y relativos, vinculados al efecto del movimiento del conductor respecto al viento. Este enofque no se encontró en la bibliografía consultada, esclareciéndose la dinámica del fenómeno. A su vez, según la Figura \ref{fig:MET:Viento:VelRel}, se develó que despreciar la velocidad perpendicular frente a la componente media, en el sentido transversal $z$,  es equivalente a que ángulos de ataque sean nulos y también así, la componente del drag según el sentido de $y$. Por otra parte, se concluyó que al considerar los campos relativos aparece un término aeroelástico, que emerge de la diferencia de velocidades, vista desde un refrencial solidario al conductor. A este termino se lo identifica en la materia con el nombre de amortiguaneinto aerodinámico o fuerza viscosas. 

Una vez descritas las hipótesis en este mismo capítulo, en la Sección \ref{Sec:MET:HHT} se generó un análisis analítico no explicado en la bibliografía de referencia \citep{Le2014}. En este apartado se aplicó el método de resolución para problemas dinámicos de HHT, incondicionalmente estable, para la formulación corrotacional. Explicando con detenimiento la deducción y premisas utilizadas. Posteriormente al despliegue teórico, se establecieron los principales pseudocódigo subyacentes a la implementación numérica en el Software \href{https://github.com/ONSAS/ONSAS/}{ONSAS}. Esta sección \ref{Sec:MET:ImplementNumeric} se desarrolló con el objetivo de esquematizar y explicar la implementación de la formulación, ademas de sentar las bases para posibles implementaciones y estudios futuros. 

En función de los avances originales de esta investigación mencionados en los párrafos anteriores. Esta tesis constituye un desarrollo complementario a la formulación propuesta, en \citep{Le2014}, incluyendo fuerzas aerodinámicas en el estudio analítico. Las estructuras representables por elementos de viga, con grandes desplazamientos y rotaciones, atacadas por el viento es enorme. Dado este diverso habaníco de aplicaciones, el interés de la comunidad científica puede ser un impulso catalizador para ciertas publicaciones a futuro.


\subsection{Sobre los resultados}

Esta formulación se valido con el ejemplo \ref{Sec:RN:RightAngle} benchmark del folclore corrotacional presentado por \cite{simo1988dynamics}. Este es cargado con una fuerza abrupta y de severa magnitud, respecot al  rigidez de la estructura alcanzando un valor de 50 $N$ en apenas 2 segundos de simulación, tal y como se muestra en la Figura \ref{fig:RN:RA:Force}. Esta fuerza pose una esencia análoga al fenómeno de tormentas convectivas per se. Esta fuerza aumenta estrepitosamente en un corto lapso de tiempo, por ende la capacidad del modelo de reproducir este tipo de impactos es fundamental para poder emular el fenómeno central de este trabajo en simulaciones de sistemas eléctricos.

 En la Figura \ref{fig:RN:RA:Dispz} se observan amplitudes que alcanzan las 8 metros cuando la estructura mide 10. Esto evidencia, la fuerte presencia de grandes desplazamientos y rotaciones. Asimismo, en la dirección $z$, se puede observar el carácter no conservativo de la formulación corrotacional, ya que los valles y crestas de las respuesta prestan una tendencia decreciente con el tiempo. En relación con los desplazamientos en el sentido de $y$ del nodo A, presentados en la Figura \ref{fig:RN:RA:DispyA}, se observa el singo negativo de este, concordando con lo esperado intuitivamente según el sentido de la fuerza aplicada. Por último, el resultado mas importante de este ejemplo se destila al cotejar las respuestas del as Figuras \ref{fig:RN:RA:DispyA}, \ref{fig:RN:RA:DispzA} y \ref{fig:RN:RA:DispzB} con lo publicado por le articulo de referencia \citep{Le2014}. Al comparar estas figuras se concluye que el modelo implementado es capaz de representar cabalmente movimientos de gran amplitud, con apenas 10 elementos por miembro y unas paso temporal de 0.25 $s$. Esto permitió validar la formulación para este ejemplo y aplicarla a dominios mas complejos específicamente con el foco en el modelado de conductores eléctricos. 

 Como primer ejemplo aplicado al modelado de conductores se eligió un problema postulado en la publicación \citep{Foti2016}. Para esto, se investigó la normativa IEC 60815 que detalla propiedades geométricas y constructivas de conductores para alta y media tensión.  Con el fin de cotejar fielmente los resultados obtenidos, se extrajeron, tanto los parámetros del flujo, como las propiedades geométricas y materiales, del trabajo de referencia correspondientes con un conductor DRAKE ASCR 7/26. No obstante, con el objetivo acercar la representación a la real, se incorporaron dos elementos aisladores ilustrativos, que por sus condiciones de borde, no afectan el comportamiento dinámico y cinemático del problema. (Ver Figura \ref{fig:RN:FotiCable:Ilustracion})
 
 Para este ejemplo de la Sección \ref{Sec:RN:FotiCable},se aplicó un viento progresivo desde un valor nulo hasta una velocidad de un perfil Capa límite atmosférica en 20 segundos, según la Figura \ref{fig:RN:FotiCable:VelocidadCable}. Este cálculo se realizó considerando las propiedades extraídas del\citep{IEC60826}, explicitadas en la Tabla \ref{table:RN:FotiCable:propiedadesCable}. Al espejar los perfiles de velocidad presentados en las Figuras \ref{fig:RN:FotiCable:DispY} y \ref{fig:RN:FotiCable:DispZ} con las fuerzas aplicadas de la Figura \ref{fig:RN:FotiCable:FuerzaZ} se observa una homología. Esto se fundamenta con un análisis de Foruier donde las salidas y entrdas son los desplazamientos y fuerzas respectivamente. 
 
 Las contribuciones principales del Ejemplo \ref{Sec:RN:FotiCable} se desprenden al contrastar los resultados del ángulo $\Phi$, gratificado en la Figura \ref{fig:RN:FotiCable:Angulo} con los presentados por L. Foti. De este análisis se extraen ciertos paralelismos y discordancias. En primer lugar las perfiles arrojados son semejantes, presentando un relación cuadrática con la velocidad. Esto se atribuye a la relación de dependencia cuadrática de la fuerza con la velocidad media de viento. Sin embargo, para el caso implementado en esta tesis se alcanzan mayores valores de ángulo. Esto puede deberse las múltiples diferencias entre los modelos: la omisión de las componentes turbulentas del flujo, el estado inicial de tensado y la presencia de hielo en las lineas. Los últimos dos factores parecen intuitivamente atenuar el angulo máximo alcanzado por la linea durante el transcurso del movimiento, por su mayor rigidez inicial e inercial. Dado estos resultados, se decidió llevarlas simulaciones a un grado mayor de complejidad, e implementar un modelo con múltiples elementos simulando un sistema de trasmisión eléctrica.  
 
 El ejemplo descrito en la sección \ref{Sec:RN:TransmissionSystem} es el resultado principal de este trabajo. Se acoplaron diferentes componentes de un sistema de alta tensión conductores, aisladores y torres. Para esto se validaron ejemplos intermedios integrando resultados lineales e inerciales conocidos con elementos de biela tipo Green y de viga corrotacional. Las geometrías y propiedades que integraron el modelo son extraídas de bibliografías experimentales y normativas buscando representar y emular el fenómeno de forma realista. 

 De igual modo, el perfil de viento se extrajo de estudios experimentales en el Norte de Alemania durante el transcurso de una tormenta convectiva tipo corriente descendente publicado en \citep{stengel2017measurements}. Esta es de una magnitud intensa, aunque no en comparación con los resultados capturados en diferentes estudios de campo nacionales como \citep{duranona2009analysis} y el trabajo de \cite{duranona2019first}, donde se alcanzan umbrales de 88.2 a 162 km/h a 45 m de altura. Otra diferencia al respecto, refiere al gradiente de velocidad, el flujo introducido numéricamente de Stangel posee una menor aceleración en comparación con tormentas en el territorio uruguayo. 
 
 La carga del viento se distribuyo en el primer vano provocando un perfil que ataque diferente a la linea en su coordenada axial. Esto genera un efecto de desfazaje entre los desplazamientos en los vanos. Esta variabilidad del flujo, busca representar un fenómeno de oscilación axial, relacionado con la presencia de vórtices a lo largo del espacio. Las diferencias en desplazamientos de los puntos A B C Y D de la cadena aisladora, se evidencia en las Figuras \ref{fig:RN:Transmission:DispsCB} y \ref{fig:RN:Transmission:DispsAD}. Por mas que los movimientos posean diferentes amplitudes de banda, los perfiles obtenidos se encuentran gráficamente emparentados con el perfil de la tormenta en la Figura \ref{fig:RN:Transmission:ForceVelTormenta}.  
 
 Finalmente se creó un análisis de contraste con un modelo ampliamente utilizados en el área de Ingeniería del Viento. Esta se utiliza para calcular de forma cuasiestaitca, utilizando una fórmula de arctoangente, basado en un péndulo sin términos dinámicos e inerciales. Los trabajos de \cite{stengel2017measurements}, \cite{duranona2009analysis} y \cite{yan2009numerical} se aplica esta aproximación simplificadora. Si bien en los resultados del Ejemplo \ref{Sec:RN:TransmissionSystem} no son comprables, la aproximación plana no funciona. Para este caso en particular, la curva numérica parece reflejar una linealidad, evaluar el ángulo de la cadena mediante el modelo estático, arrojaría un resultado de sobrestimado. Esto se detalla en la Figura \ref{fig:RN:Trnsmission:CurvaCargaDisp}.

Estos resultados presentan indicios que para enfrentar la problemática, los códigos generados pueden gestar una herramienta de análisis complementario para el diseño de sistemas de trasmisión de alta tensión. Según contactos establecidos con la empresa de transmisión eléctrica (UTE), las torres de alta y media tensión suelen encargarse a empresas privadas que obtienen la obra por licitación y entregan las instalaciones con llave en mano. Estos proyectos suelen importar soluciones del extranjero, que pueden ser no aplicables a las condiciones nacionales. Esto se explica por la carencia de las normas internacionales en materia de fenómenos de viento no sinópticos como corrientes descendentes y ciclones extratropicales. Esto se intensifica en el territorio para sistemas montados hace 30 años en superposición con la asiduidad y frecuencia en los periodos de retorno. 

\section{Conclusiones de formación}
El desarrollo de este trabajo constituyó una instancia de formación fundamental y enriquecedora para el autor enmarcada dentro del programa de Magister en Ingeniería Estructural. Este documento es la síntesis y aplicación de un conjunto de conocimientos profundizados durante la actividad programada, aplicada al modelado numérico de estructuras. Desde la óptica del autor, la creación de herramientas endogenas con foco en atacar problemáticas a nivel nacional constituye un pilar fundamental en el desarrollo autónomo y original de la ingeniería uruguaya. Este trabajo es una muestra de la convicción y determinación, que el conocimiento académico, debe desarrollarse de forma transparente, comunitaria y democrática. Es por esto, que todos los códigos utilizados en esta investigación se implementaron en el software libre \href{https://github.com/ONSAS/ONSAS/}{ONSAS}. Esto abre la posibilidad a cualquier tercero ya sea una organización o persona de estudiar, modificar y difundir los códigos creados como también aplicarlos a sus propias necesidades. 

\section{Trabajos a futuro}

Actualmente este trabajo presenta algunas limitaciones o falencias que deberían mejorarse de continuar esta línea de investigación. Como guías de futuras se proponen los siguientes lineamientos que buscan ampliar las potencias y capacidades del modelo:

\begin{enumerate}
	\item Incluir en el análisis teórico de la formulación corrotacional condiciones de Dirichlet no homogenas en desplazamientos que sean capaces de  representar el tensado del conductor durante la instalación. La hipótesis reduccionista sobre la tensión inicial es imprecisa y disminuye la exactitud en la representación del fenómeno. Según el punto de vista del autor, esta implementación en \href{https://github.com/ONSAS/ONSAS/}{ONSAS} es el punto de partida en la continuación de este trabajo. 
	\item Implementar un módulo modal dentro del \href{https://github.com/ONSAS/ONSAS/}{ONSAS} capaz de calcular los modos estructurales, insumo fundamental para realizar un análisis en frecuencia de posibles resonancias viento-conductor.
	\item Agregar al Software  \href{https://github.com/ONSAS/ONSAS/}{ONSAS} la posibilidad de incluir relaciones de fuerza viscosas no lineal con diferentes coeficientes de drag y lift de acuerdo al perfil geométrico de la sección. Incluir a partir de esto fuerzas viscosas no lineales, al desarrollo analítico de la formulación corrotacional y su implementación numérica. 
	\item Agregar al modelo los elementos separadores con mas de un conductor por aislador. En las instalaciones visitadas de forma presencial, se observaron una serie de separadores que mantienen distanciadas los conductores evitando el cortocircuito. Además, unen a cuatro conductores aportando una mayor rigidez e inercia en los tendidos. Este análisis deberá incluir diferentes valores de coeficientes de drag dada la proximidad entre conductores y sus efectos sobre las líneas de flujo.  
	\item Incorporar diferentes geometrías de torres presentes en los distintos tendidos de distribución del país. Según los datos recolectados en las lineas de distribución a partir de la década del 2000, las modelos de torres cambiaron respecto a los que se representaron el Ejemplo \ref{Sec:RN:TransmissionSystem}. Es importante este análisis para lograr emular la influencia de arquitectura de las torres en la aproximación excesiva del conductor las barras. De igual manera, adquirir datos reales aportados por UTE.
	\item Incorporar al modelo el agarre doble, que en determinadas ocasiones, se dispone en las lineas centrales de la torre. En algunos casos, una solución ante la aproximación inminente del aislador, consiste en instalar una cadena aisladora extra que oficia de sujetador adicional para los conductores. Rigidizando y evitando de este modo el balanceo desmesurado. Otro tipo de soluciones implantadas, consiste en agregar pesos sobre puntos estratégicos en las lineas, aumentando la inercia del sistema. En este caso, la elección del peso consiste en un compromiso entre los esfuerzos generados en el cable sin alcanzar la fluencia y la masa que disminuye el balanceo. Este tipo de soluciones paliativas resultan interesantes como objeto de simulación.	
	\item Generar un análisis de malla y sensibilidad respecto a las condiciones de borde establecidas y el numero de elementos por unidad de largo del conductor. Esto permitiría estudiar que grado de discretización sería el ópitmo para minimizar el error numérico sin incurrir en un tiempo excesivo de simulación. 
	\item Integrar la herramienta \href{https://github.com/ONSAS/ONSAS/}{ONSAS} con un solver de fluidos como por ejemplo el caffa.3d.MBRi basado en volúmenes finitos con paralelización multiforntal \cite{mendina2014general}. Esta ardua integración permitiría generar una herramienta sumamente potente para atacar problemas de interacción fluido-estructura.	
\end{enumerate}

\section{Reflexión}

Toda disciplina e investigación debería conocer sus propias fugas, fronteras y puntos ciegos. De lo contrario, cualquier pretensión hermética podría ser un síntoma de arrogancia y altanería.  A lo largo de este trabajo he canonizado una redacción en tercera persona, como si hubiese una determinada imparcialidad y transparencia en dicho escritor. Este sujeto, apuntado y enfocado en los párrafos siguientes, merece ensimismarse y cuestionarse a si mismo, según el proverbio del Oráculo de Delfos \emph{gnóthi sautón} o en castellano  \emph{Conócete a ti mismo}.

Durante el transcurso de este trabajó me surgieron las siguientes inquietudes ¿Es la realidad un conjunto de fenómenos externos o es siempre un acto de interpretación inmanente al sujeto? Ademas, ¿Ese sujeto accede la realidad (el objeto) a través de la razón para conocer y explicarla, o simplemente la experiencia es quien valida ese conjunto de fenómenos?. A partir de esta pregunta, emana una interrogante natural, ¿Es posible entonces, desligar al sujeto del objeto, o mas bien es ente,  (ex-siste) en el mundo, y esta siempre arrojado, eyectado y lanzado hacia el? Y de ser así, ¿No se encuentra entonces  yá sugestionado por el paradigma actual, su cultura nativa y sus experiencias personales?

Antes que nada, es necesario develar que sujeto en latín \emph{sub}-{iectum} significa lo que esta debajo, según una interpretación posmoderna. Es entonces el sustrato de esos entes que este dota de sustancia, colores, palabras y formas. Esas preguntas han sido abordadas por eminencias de la filosofía y la ciencia, desde la modernidad hasta hoy. Por un lado, el realismo científico concibe que es posible constatar la realidad a través de la experiencia experimental o a través del pensamiento. Para Descartes ese sujeto duda, piensa y por tanto \textbf{ya} en ese acto analítico, existe (\emph{Cogito ergo sum})\cite{descartes2004discurso}, osea el ente en tanto ente. El padre del racionalismo nos plantea que el es yo del sujeto, a través de la duda metódica puede acceder la verdad. Contrapuesto a este, el empirismo valida cualquier conocimiento sólo por la experiencia. Esta se define por lo que es captado por nuestros sentidos, es decir que la experiencia es sensorial. Estas dos posturas, la del racionalismo de Descartes y la del empirismo de Hume, pueden ser pensadas como una forma de abordaje a la relación realidad - conocimiento. Para Descartes: conozco en tanto analizo y pienso, y los objetos existen cuando yo realizo la abstracción. Para el empirismo: conozco en la medida en que incorporo la realidad ``objetiva", la de los objetos que puedo percibir por la experiencia sensorial. 

A mediados del sg XX surgió un pensador disruptivo viró absolutamente a la cuestión. Frederick Niezstche plantea en su libro Voluntad de Poder \cite{nietzsche2018voluntad}" El pensar no es para nosotros un medio para ``conocer" sino para designar el acontecer, para ordenarlo, para volverlo manejable para nuestro uso: así pensamos hoy acerca del pensar: mañana quizá de otro modo ". Esta frase alude, desde mi perspectiva, a un nihilismo que niega la posibilidad de conocer algo absoluto verdadero pues no es más que un desarrollo pragmático de poder. Sino mas bien es una cuestión de voluntad de voluntad, un dispositivo ordenatorio de la realidad según categorías y características en nuestro acto de querer/poder conocer. Antípoda a esta teoría nihilista aparece el relativismo. Este se estriba en el principio de incertidumbre Heisenberg, si existe ese conocimiento, es entonces indisoluble de cierta estructura. Thomas Khun en su libro \emph{La estructuras de las revoluciones científicas} \cite{kuhn2019estructura} plantea que el método científico revoluciona, cuando se produce un cambio de paradigma, no a partir de la observación de nuevos hechos o fenómenos. Junto con otros destacados sociólogos, acuñan la idea del concepto de ``cargado de teoría", un cierto conjunto de preconceptos anteriores a la observación, descripción y desarrollo de la cualquier investigación, que llevarán al científico demostrar lo que realmente quiere demostrar... deunuevo demostración de poder.

¿Como se demuestran los resultados de esta investigación?, construyendo un conjunto de artefactos experimentales/computacionales que constatan una supuesta realidad casí como por espejo o correspondencia. En ese proceso de creación o utilización de instrumentos como ser: un programa, un nanemómetro o código computacional existe una omnipresente intervención humana. ¿Vale entonces seguir redactando en tercera persona desde un racionalismo positivista heredado de hace dos siglos? ¿Es coherente no ser categórico en la descripción de un resultado, cuando \textbf{ya} todo el dispositivo ordenatorio que subyace es una construcción humana? ¿Debemos seguir defendiendo un cadáver \textbf{ya} asesinado por las ciencias humanas desde un sujeto que no es mas que un efecto cultural, histórico y económico?. Por una ciencia en primera persona! 


