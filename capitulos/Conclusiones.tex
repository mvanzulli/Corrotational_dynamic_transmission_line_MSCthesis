\chapter{Consideraciones finales}\label{Cap:Conlcusiones}
\linenumbers

El desarrollo de este trabajo constituyó una instancia de formación fundamental y enriquecedora para el autor enmarcada dentro del programa de Magister en Ingeniería Estructural. Este documento es la síntesis y aplicación de un conjunto de conocimientos profundizados durante la actividad programada, aplicada al modelado numérico de estructuras. Desde la óptica del autor, la creación de herramientas endogenas con foco en atacar problemáticas a nivel nacional constituye un pilar fundamental en el desarrollo autónomo y original de la ingeniería uruguaya. Este trabajo es una muestra de la convicción y determinación, que el conocimiento académico, debe desarollarse de forma transparente, comunitaria y democrática. Es por esto, que todos los códigos utilizados en esta investigación se implementaron en el software libre \href{https://github.com/ONSAS/ONSAS/}{ONSAS}. Esto abre la posibilidad a cualquier tercero ya sea una organización o persona de estudiar, modificar y difundir los códigos creados como también aplicarlos a sus propias necesidades. 



Se pudo verificar que la mayoría de las incidencias ocurridas en las líneas Palmar-Montevideo de 500kV coinciden con el pasaje de tormentas severas sobre la zona estudiada. Estas tormentas producen CD intensas que pueden ocasionar velocidades y cargas elevadas sobre los conductores por periodos del orden de los minutos, imponiendo ángulos de balanceo excesivos que acercarían los conductores a las torres a una distancia capaz de producir descarga a tierra e impedirían el reenganche de la línea durante ese lapso.
Dada una misma velocidad a 10m de altura, las CD imponen cargas mayores sobre los conductores, y por ende, mayores ángulos de balanceo que los flujos tipo CLA debido a su particular distribución de velocidad en altura y a sus mayores correlaciones sobre los vanos. Por otra parte, se han encontrado indicios que los periodos de retorno para velocidades elevadas, como por ej., de 100km/h o más, serían menores para las CD que para los flujos tipo CLA. Para poder cuantificar estos periodos de retorno se hace necesario llevar adelante una investigación más detallada sobre las CD en Uruguay.

Se implementó y validó un modelo corrotacional consistente robusto y eficaz que es capaz de captar y reproducir desplazamientos de gran amplitud con numero reducido de elementos.  Esta formulación se aplico específicamente a conductores de alta tensión sometidos perfiles de viento extraídos de artículos recientes aplicados a tormentas convectivas. Las respuestas del sistema evidencian el balanceo excesivo del conductor \ref{fig:DeformadasEqual}, ante este tipo de solicitaciones, los códigos generados pueden gestar una herramienta de análisis complementario para el diseño de sistemas de trasmisión eléctrica. Al vincular 		\ref{fig:CableDispY} y \ref{fig:CableFuerzaZ} se evidencian la idéntica forma que desarrollan ambo perfiles colmando las expectativas sobre dicha salida. Como trabajos a futuro se debería verificar el no deslizamiento de las hebras internas según lo publicado en \citet{foti2018finite}. Este comportamiento de histéresis depende principalmente de las fuerzas normales al interior del cable, esto es imprescindible para asegurarse del correcto modelado del conductor como un solido circular. Como eventuales trabajos a futuros se propone la implantación de un modelo acoplado con la torre donde no se desprecien lo desplazamientos del punto de anclaje y las frecuencias de resonancia que este cambio pueda implicar. Por ultimo es oportuno mencionar la potencialidad de este trabajo para desarollar un solver integrado entre los softwares ONSAS-CAFFA \cite{usera2008parallel}.




Con respecto a la norma se esclarecieron las metodologías propuestas para el diseño de lineas de trasmisión,  se corroboraron los valores supuestos por la norma con diferentes referencias, algunas expuestas en el curso, y otras investigadas. Se destaca como debilidad que para el diseño que esta no considera eventos de vientos no sinópticos, estos pueden vulnerar al sistema, como se observaron en distintas bibliografías, solo se consideran viento tipo capa límite atmosférica. 


Para el análisis del  problema en particular, se creó una librería que contiene un conjunto de códigos que, mediante el método de elementos finitos e iteraciones de Newton-Raphson, resuelve dicho problema para diversas condiciones de borde.  Se concluye que este es capaz de reproducir de forma adecuada el balanceo del cable en contraste con \cite{Stengel2017}. Las desviaciones entre los 230 y 500 segundos pueden estar asociadas a errores durante la toma de datos del perfil de velocidades y/o términos no inerciales de magnitud apreciable, en este período no sería válido aplicar un modelo estático como se realiza en \cite{Stengel2017}. El software permite obtener los modos normales, y plotear sus modos asociados sobre la configuración indeformada, además se generaron vídeos para cada uno de estos modos los cuales se adjuntan en la carpeta del código. Los modos de vibración constituyen un importante resultado a contrastar con las frecuencias predominantes en el flujo, para el ejemplo se encuentran alejadas en al menos un orden de magnitud, sin embrago otras geometrías podrían producir resonancias dinámicas, aumentándose significativamente la amplitud de oscilación. 

Como trabajo a futuro se debería incluir un modelo de vigas a flexión,  dinámico y no lineal que represente completamente el cambio de rigidez e inercia a lo largo de la trayectoria del conductor. En relación a esto un modelo correccional presentado en las referencias \cite{Le2014} y \cite{Le2011} se ajusta a las necesidades.