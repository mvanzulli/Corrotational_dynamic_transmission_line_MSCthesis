\chapter{Conclusiones}\label{Cap:Conlcusiones}
\linenumbers

El presente capítulo se separó separarse en tres secciones que se relacionan con diferentes aristas o perspectivas del trabajo llevado a cabo. En primera instancia, se detallan las consideraciones finales y de síntesis, desde un punto de vista técnico sobre los resultados obtenidos. Luego de esto, se analizan limitaciones que deberían de mejorarse en posibles trabajos a futuro.  Posteriormente, se narran los aspectos del desarrollo académico de esta tesis, como trabajo cúlmine de una etapa formativa fundamental para quien escribe.  

\section{Conclusiones técnicas}
Inicialmente se consultó el estado del arte en el área de Ingeniería del viento y estructural. Se analizaron bibliografías en materia de simulaciones numéricas aplicadas a conductores eléctricos, con abordajes semi-analíticos y computacionales. También, se estudiaron trabajos nacionales e internacionales, desde un punto de vista cualitativo y experimental de CD y sus posibles perjuicios sobre líneas de transmisión eléctrica. Existe vasta evidencia de que el fenómeno de \gls{TC} ha afectado severamente la calidad e integridad de la vida humana a lo largo y ancho del globo terráqueo. En particular, debido a las condiciones climáticas singulares de la región, y el progresivo calentamiento global, se han intensificado los daños devastadores en los sistemas de transmisión y distribución eléctrica nacionales. Induciendo inevitablemente, en costos millonarios de reparación sobre las instalaciones, y pérdidas durante la interrupción del suministro. Esta investigación construye una herramienta de simulación computacional, capaz de emular los desmedidos desplazamientos y esfuerzos que estos eventos producen sobre los sistemas de trasmisión eléctrica. 

Habiendo finalizado esta tesis, se enumeran las principales conclusiones técnicas de este trabajo:


\begin{description}
	\item[Conclusión T1:] \hfill \\ Se implementó y validó dentro del código abierto \href{https://github.com/ONSAS/ONSAS.m/}{ONSAS} una formulación corrotacional de vigas 3D para la simulación de problemas dinámicos no lineales.
	\item[Conclusión T2:] \hfill \\ Se extendió analíticamente la formulación corrotacional para elementos de cables incorporando términos de amortiguamiento aerodinámicos.
	\item[Conclusión T3:] \hfill \\ Se generó un modelo que representa el acoplamiento entre torres y conductores sometido por la acción de un perfil tipo CC. Según los resultados del modelo, las tormentas convectivas afectan a las líneas generando desplazamientos de casi 7 metros y ángulos de hasta 30º en la cadena aisladora. 
\end{description}

\paragraph*{Conclusión T1:}
Debido a numerosas ventajas se eligió la formulación corrotacional de vigas 3D para grandes desplazamientos y rotaciones. Una vez ahondado en la temática, se implementó y validó un modelo corrotacional consistente robusto y eficaz, capaz de captar y reproducir desplazamientos de gran amplitud con número reducido de elementos. Esta formulación se validó con el ejemplo \ref{Sec:RN:RightAngle} benchmark del folclore corrotacional presentado por \cite{simo1988dynamics}. Este es cargado con una fuerza abrupta y de severa magnitud, en relación a la rigidez de la estructura alcanzando un valor de $50$ N en apenas 2 segundos de simulación, tal y como se muestra en la Figura \ref{fig:RN:RA:Force}. Esta fuerza posee una esencia análoga al fenómeno de TC per se. Esa semejanza radica en el aumento súbito de su magnitud, en un corto lapso de tiempo, por ende, la capacidad del modelo de reproducir este tipo de impactos es fundamental para poder representar el fenómeno central de este trabajo.

En la Figura \ref{fig:RN:RA:DispzA} se observan amplitudes que alcanzan los 8 metros cuando la estructura mide 10. Esto evidencia, la fuerte presencia de grandes desplazamientos y rotaciones.  En relación a los desplazamientos en el sentido de $y$ del nodo A, presentados en la Figura \ref{fig:RN:RA:DispyA}, se observa el singo negativo de este, concordando con lo esperado intuitivamente según la fuerza aplicada. Por último, el resultado más importante de este ejemplo, se destila al cotejar las respuestas del as Figuras \ref{fig:RN:RA:DispyA}, \ref{fig:RN:RA:DispzA} y \ref{fig:RN:RA:DispzB} con lo publicado por el artículo de referencia \citep{Le2014}. Al comparar estas figuras se concluye que el modelo implementado es capaz de representar cabalmente movimientos de gran amplitud, con apenas 10 elementos por miembro y un paso temporal de 0.25 $s$. Esto permitió validar la formulación para este caso y aplicarla a dominios más complejos específicamente con el foco en conductores eléctricos. 

\paragraph{Conclusión T2:}
En la Sección \ref{Sec:PRE:Modeloviento} se desarrolló un estudio general sobre los campos de velocidades absolutos y relativos, vinculados al efecto relativo del movimiento del conductor respecto al viento. Este enfoque no se encontró en la bibliografía consultada, esclareciéndose la dinámica del fenómeno. A su vez, según la Figura \ref{fig:MET:Viento:VelRel}, se develó que despreciar la velocidad perpendicular frente a la componente media, en el sentido transversal $z$,  es equivalente a el ángulo de ataque sea nulo y también así, la componente del drag según el sentido de $y$. Por otra parte, se concluyó que al considerar los campos relativos aparece un término aeroelástico, que emerge de la diferencia de velocidades, vista desde un referencial solidario al conductor. A este término se lo identifica en la materia con el nombre de amortiguamiento aerodinámico y, según lo estudiado, no había sido incluido en la metodología corrotacional. 

Una vez descritas las hipótesis en este mismo capítulo, en la Sección \ref{Sec:MET:HHT} se generó un análisis analítico no explicado en la bibliografía de referencia \citep{Le2014}. En esta misma sección se aplicó el método de resolución para problemas dinámicos de HHT, incondicionalmente estable, explicando con detenimiento la deducción y premisas utilizadas. Complementario, al desarrollo teórico, se establecieron los principales pseudocódigos subyacentes a la implementación numérica en el Software \href{https://github.com/ONSAS/ONSAS.m/}{ONSAS}.

En función de los avances originales de esta investigación mencionados en los párrafos anteriores. Esta tesis constituye un desarrollo complementario a la formulación propuesta, por \cite{Le2014}, incluyendo fuerzas aerodinámicas linealizadas provenientes de la acción de vientos sobre conductores. Esto puede aplicarse a un espectro enorme de estructuras representables por elementos de viga, con grandes desplazamientos y rotaciones, atacadas por el viento. Dado este diverso abanico de aplicaciones, el interés de la comunidad científica puede ser un impulso catalizador para ciertas publicaciones a futuro.

\paragraph*{Conclusión T3:}
Se acoplaron diferentes componentes de un sistema de alta tensión conductores, aisladores y torres. Con este objetivo, se validaron ejemplos intermedios integrando elementos de biela tipo Green y de viga corrotacional con resultados lineales y dinámicos conocidos. Las geometrías y propiedades que integraron el modelo son extraídas de bibliografías experimentales y normativas buscando representar y emular el fenómeno de forma realista.  Con el mismo cometido, el perfil de viento se extrajo de estudios experimentales en el Norte de Alemania durante el transcurso de una tormenta convectiva, tipo corriente descendente, publicado en \citep{stengel2017measurements}. Esta es de una magnitud intensa, aunque no en comparación con los resultados capturados en diferentes estudios de campo nacionales, en \citep{duranona2009analysis} y \citep{duranona2019first}. En estos artículos se presentan medidas que alcanzan umbrales de 88.2 a 162 km/h a 45 m de altura. Otra diferencia al respecto, refiere al gradiente de velocidad, el flujo introducido numéricamente del autor \citeauthor{stengel2017measurements} posee una menor aceleración en comparación con tormentas en el territorio uruguayo. 

La carga del viento se distribuyó en el primer vano, provocando un perfil que ataque diferente a la línea en su coordenada axial. Esto genera un efecto de desfasaje entre los conductores de los vanos entre la torres 1-2 y 2-3 de la Figura \ref{fig:RN:Transmission:Deformadas}. Esta variabilidad del flujo, busca representar un fenómeno de oscilación axial, relacionado con la presencia de vórtices a lo largo del espacio. Las diferencias en desplazamientos de los puntos A B C Y D de la cadena aisladora, se evidencia en las Figuras \ref{fig:RN:Transmission:DispsCB} y \ref{fig:RN:Transmission:DispsAD}. Por más que los movimientos posean diferentes amplitudes de banda, los perfiles obtenidos se encuentran gráficamente emparentados con el perfil de la tormenta en la Figura \ref{fig:RN:Transmission:VelocidadTormentaX}. Vale destacar que en estos resultados se evidencia una cierta oscilación de alta frecuencia que puede deberse a inestabilidades numéricas.  

Se desarrolló un análisis de contraste con un modelo ampliamente utilizado en el área de Ingeniería del Viento. Esta se utiliza para calcular de forma cuasiestaitca, utilizando una fórmula de arctoangente. Esta se basa en un péndulo cuasiestático plano, omitiendo términos inerciales. Los trabajos de \cite{stengel2017measurements}, \cite{duranona2009analysis} y \cite{yan2009numerical} aplican esta aproximación simplificadora. Si bien en los resultados del Ejemplo \ref{Sec:RN:TransmissionSystem} no son comprables, la aproximación plana no funciona. Para este caso en particular, la curva numérica parece reflejar una linealidad, evaluar el ángulo de la cadena mediante el modelo estático, arrojaría un resultado de sobrestimado. Esto se detalla en la Figura \ref{fig:RN:Trnamission:CurvaCargaDisp}.

Estos resultados presentan indicios que, para enfrentar la problemática, los códigos generados pueden gestar una herramienta de análisis complementario para el diseño de sistemas de trasmisión y distribución. Según contactos establecidos con la empresa de transmisión eléctrica (UTE), las torres de alta y media tensión suelen encargarse a empresas privadas que obtienen la obra por licitación y entregan las instalaciones con llave en mano. Estos proyectos suelen importar soluciones del extranjero, que pueden ser no aplicables a las condiciones nacionales. Esto se explica por la carencia de las normas internacionales en materia de fenómenos de viento no sinópticos como CD y ciclones extratropicales. Esto se intensifica en el territorio para sistemas montados hace 30 años en superposición con la asiduidad, intensidad y frecuencia de TC. 

Uniendo resultados de diferentes trabajos internacionales con los resultados del modelo presentado en la Sección \ref{Sec:RN:TransmissionSystem}, es posible teorizar que la mayoría de las incidencias ocurridas en las líneas Palmar-Montevideo de 500kV, pueden deberse al pasaje de tormentas severas sobre la zona. Estas tormentas producen CD, que ejercen cargas desmesuradas sobre el conductor, en el orden de minutos, imponiendo ángulos de balanceo excesivos, acercando los conductores a las torres, a una distancia tal, que inminentes descarga a tierra pueden dejar fuera de servicio a la línea. Además según los estudios de la norma \cite{IEC60826} del Apéndice \ref{Ape2}, esta solo considera vientos tipo CLA. Esto podría estar subestimando las fuerzas ejercidas por el viento, ya que para el territorio uruguayo según \cite{Duranona2018}, los periodos de retorno para velocidades de hasta $100$ km/h es menor para CD en comparación con vientos tipo CLA. 




% Sobre el Ejemplo de FOti
%
% Como primer ejemplo aplicado al modelado de conductores se eligió un problema postulado en la publicación \citep{Foti2016}. Para esto, se investigó la normativa \cite{IEC60826} que detalla propiedades geométricas y constructivas de conductores para alta y media tensión.  Con el fin de cotejar fielmente los resultados obtenidos, se extrajeron, tanto los parámetros del flujo, como las propiedades geométricas y materiales, del trabajo de referencia correspondientes con un conductor DRAKE ASCR 7/26. No obstante, con el objetivo acercar la representación al fenómeno, se incorporaron dos elementos aisladores ilustrativos, que por sus condiciones de borde, no afectan el comportamiento dinámico y cinemático del problema. (Ver Figura \ref{fig:RN:FotiCable:Ilustracion})
% 
% Para este ejemplo de la Sección \ref{Sec:RN:FotiCable}, se aplicó un viento progresivo desde un valor nulo hasta una velocidad de un perfil Capa límite atmosférica en 20 segundos, según la Figura \ref{fig:RN:FotiCable:VelocidadCable}. Este cálculo se realizó considerando las propiedades extraídas de la norma \citep{IEC60826}, explicitadas en la Tabla \ref{table:RN:FotiCable:propiedadesCable}. Al espejar los perfiles de velocidad presentados en las Figuras \ref{fig:RN:FotiCable:DispY} y \ref{fig:RN:FotiCable:DispZ}, con las fuerzas aplicadas de la Ilustración \ref{fig:RN:FotiCable:FuerzaZ} se observa una homología. Esto se fundamenta con un análisis de Foruier donde los desplazamientos ofician de salida y las fuerzas de entrada. 
% 
% Las contribuciones principales del Ejemplo \ref{Sec:RN:FotiCable} se desprenden al contrastar los resultados del ángulo $\Phi$, gratificado en la Figura \ref{fig:RN:FotiCable:Angulo} con los presentados por \citeauthor{Foti2016}. De este análisis se extraen ciertos paralelismos y discordancias. En primer lugar, los perfiles arrojados son semejantes, presentando un relación cuadrática con la velocidad. Esto se atribuye a la función de dependencia cuadrática entre la fuerza y la velocidad media de viento. Sin embargo, para el caso implementado en esta tesis se alcanzan mayores valores de ángulo. Esto puede deberse a múltiples diferencias entre los modelos: la omisión de las componentes turbulentas del flujo, el estado inicial de tensado y la presencia de hielo en las líneas. Los últimos dos factores intuitivamente tienden a disminuir el ángulo máximo alcanzado por la línea, durante el transcurso del movimiento, por su mayor rigidez inicial e inercial. Dado estos resultados, se decidió llevar las simulaciones a un grado mayor de complejidad, e implementar un modelo con múltiples elementos simulando un sistema de trasmisión eléctrica.  

\section{Trabajos a futuro}

Actualmente este trabajo abre claras líneas de investigación y desarrollo para continuar la mejora de los modelos que se aproximen a la realidad con mayor precisión. Como trabajo a futuro para continuar la línea de investigación con un encare general se proponen los siguientes lineamientos:

\begin{enumerate}
	\item Investigar exhaustivamente sobre el origen de las oscilaciones de alta frecuencia observadas en los resultados numéricos del ejemplo de la Sección \ref{Sec:RN:TransmissionSystem}.
	\item Incluir en el análisis teórico de la formulación corrotacional condiciones de Dirichlet no homogéneas en desplazamientos, que sean capaces de representar el tensado del conductor durante la instalación. La hipótesis reduccionista sobre la tensión inicial, aparenta ser imprecisa respecto a la rigidez del sistema y tiende a reducir la exactitud en la representación del fenómeno. Según el punto de vista del autor, esta implementación en \href{https://github.com/ONSAS/ONSAS.m/}{ONSAS} es el punto de partida en la continuación de este trabajo. 
	\item Implementar un módulo modal dentro del \href{https://github.com/ONSAS/ONSAS.m/}{ONSAS} capaz de calcular los modos estructurales, insumo fundamental para realizar un análisis en frecuencia de posibles resonancias viento-conductor.
	\item Agregar al desarrollo analítico de la formulación corrotacional la posibilidad de incluir relaciones de fuerza viscosas, no lineales con diferentes coeficientes de \textit{drag} y \textit{lift} de acuerdo al perfil geométrico de la sección e implementarlo en el Software \href{https://github.com/ONSAS/ONSAS.m/}{ONSAS}.
	\item Agregar al modelo del Ejemplo \ref{Sec:RN:TransmissionSystem} los elementos separadores con más de un conductor por aislador. En las instalaciones visitadas de forma presencial, se observaron una serie de separadores que mantienen distanciados los conductores evitando el cortocircuito. Además, al unir cuatro cables generan una mayor rigidez e inercia en los tendidos. Este análisis deberá incluir diferentes valores de coeficientes de drag dada la proximidad entre conductores y sus efectos sobre las líneas de flujo.  
	\item Verificar el no deslizamiento interno entre las lingas que conforman el conductor, según los estudios propuestos por \cite{Foti2016}. Esto permitiría verificar la hipótesis asumida respecto al comportamiento de unión que mantiene el conductor durante sus trayectorias. Asu vez generar un aporte original estudiando como las TC afectan al fenómeno de deslizamiento interno de \cite{Papailiou1997}. 
	\item Generar un análisis de malla en el número de elementos por unidad de largo del conductor y sensibilidad respecto a las condiciones de borde establecidas. Esto permitiría estudiar el grado de discretización óptimo, para minimizar el error numérico sin incurrir en un tiempo excesivo de simulación. 
	\item Integrar la herramienta \href{https://github.com/ONSAS/ONSAS.m/}{ONSAS} con un solver de fluidos como por ejemplo el caffa.3d.MBRi basado en volúmenes finitos con paralelización multiforntal \cite{mendina2014general}. Esta ardua integración permitiría generar una herramienta sumamente potente para atacar problemas de interacción fluido-estructura.	
\end{enumerate}

Con el objetivo de generar una herramienta de diseño complementario para UTE se proponen los siguientes trabajos a futuro:

\begin{enumerate}
	\item Incorporar diferentes geometrías de torres presentes en los distintos tendidos de distribución del país. Según los intercambios con el personal de trasmisión de UTE, las líneas de distribución, a partir de la década del 2000, respecto a los que se representaron el Ejemplo \ref{Sec:RN:TransmissionSystem} cambiaron las geometrías de torres. Es importante este análisis para lograr emular la influencia de la arquitectura de las torres, en la aproximación excesiva del conductor a las barras. De igual manera, adquirir datos reales aportados por UTE podría aportar un valor significativo a esta investigación.
	\item Incorporar al modelo el agarre doble que, en determinadas ocasiones, se dispone en las líneas centrales de la torre. Esta es una solución ante la aproximación inminente del aislador, consiste en instalar una cadena aisladora extra que oficia de sujetador adicional para los conductores. Rigidizando y evitando de este modo el balanceo desmesurado. Otro tipo de soluciones implantadas, consiste en agregar pesos sobre puntos estratégicos en las líneas, aumentando la inercia del sistema. En este caso, la elección del peso consiste en un compromiso entre los esfuerzos generados en el cable sin alcanzar la fluencia y la masa que atenúa el balanceo. Este tipo de soluciones paliativas resultan interesantes como objeto de simulación.	
\end{enumerate}


\section{Conclusiones de formación}
El desarrollo de este trabajo constituyó una instancia de formación fundamental y enriquecedora para el autor enmarcada dentro del programa de Maestría en Ingeniería Estructural. Este documento es la síntesis y aplicación de un conjunto de conocimientos profundizados durante la actividad programada, aplicada al modelado numérico de estructuras. Desde la óptica del autor, la creación de herramientas endógenas con foco en atacar problemáticas a nivel nacional constituye un pilar fundamental en el desarrollo autónomo y original de la ingeniería uruguaya. Este trabajo es una muestra de la convicción y determinación, que el conocimiento académico, debe desarrollarse de forma transparente, comunitaria y democrática. Es por esto, que todos los códigos utilizados en esta investigación se implementaron en la herramienta de software libre \href{https://github.com/ONSAS/ONSAS.m/}{ONSAS}. Esto abre la posibilidad a cualquier tercero, ya sea una organización o persona, de estudiar, modificar y difundir los códigos creados como también aplicarlos a sus propias necesidades. 
\subsection{Reflexión personal}
Antes que nada, es necesario realizar una arqueología de las palabras sujeto y fenómeno en castellano. Sujeto en latín \emph{sub}-{iectum} significa lo que está debajo, según una interpretación posmoderna. Desde esta perspectiva, es el sujeto el sustrato de cualquier ente, que lo dota de sustancia, colores, palabras y formas. Por otra parte, fenómeno tiene una raíz etimológica en la palabra \emph{phainomenon} al igual que la palabra fantasía. Esto alude a lo que se muestra, lo que se deja ver, lo que brilla. Ahora bien, en el acto de percibir cognitivamente existe una dirección previa (inconsciente o consciente) de apuntar el foco hacia algo, entonces ¿Quién y cómo se dirige ese foco?

Toda disciplina e investigación debería conocer sus propias fugas, fronteras y puntos ciegos. De lo contrario, cualquier pretensión hermética podría ser un síntoma de arrogancia y altanería.  A lo largo de este trabajo, he canonizado una redacción en tercera persona, como si existiese una determinada imparcialidad y transparencia en dicho escritor. O quizás una búsqueda con necedad de la verdad absoluta. Este sujeto, apuntado y enfocado en los párrafos siguientes, merece ensimismarse y cuestionarse a sí mismo, según el proverbio en templo de Apolo del Oráculo de Delfos, \emph{gnóthi sautón} o en castellano \emph{Conócete a ti mismo}.

Durante el transcurso de este trabajó me surgieron las siguientes inquietudes ¿Es la realidad un conjunto de fenómenos externos o es siempre un acto de interpretación inmanente al sujeto? Además, ¿Ese sujeto accede la realidad (el objeto) a través de la razón para conocer y explicarla, o simplemente la experiencia es quien valida ese conjunto de fenómenos? A partir de esta pregunta, emana una interrogante natural, ¿Es posible entonces, desligar al sujeto del objeto, o más bien este ente (ex-siste) en el mundo, y está siempre arrojado, lanzado y en relación con el? Y de ser así, ¿No se encuentra entonces \textbf{ya} sugestionado por el paradigma actual, su cultura nativa y sus experiencias personales cuando describe?

Esas preguntas han sido abordadas por eminencias de la filosofía y la ciencia, desde la modernidad hasta hoy. Por un lado, el realismo científico concibe que es posible constatar la realidad a través de la experiencia o a través del pensamiento. Para Descartes ese sujeto duda, piensa y por tanto \textbf{ya} en ese acto analítico, existe (\emph{Cogito ergo sum}) \cite{descartes2004discurso}, ósea el ente en tanto ente. El padre del racionalismo nos plantea que es el yo del sujeto, quien a través de la duda metódica puede acceder la verdad. Contrapuesto a este, el empirismo valida cualquier conocimiento sólo por la experiencia. Esta se define por lo que es captado por nuestros sentidos, es decir que la experiencia es sensorial. Estas dos posturas, la del racionalismo de Descartes y la del empirismo de Hume, pueden ser pensadas como una forma de abordaje a la relación realidad - conocimiento. Para Descartes: conozco en tanto analizo y pienso, y los objetos existen cuando yo realizo la abstracción. Para el empirismo: conozco en la medida en que incorporo la realidad ``objetiva", la de los objetos que puedo percibir a través de los sentidos. 

A mediados del sg XIX nació un pensador disruptivo que viró absolutamente a la cuestión. Frederick Niezstche plantea en su libro Voluntad de Poder \cite{nietzsche2018voluntad} `` El pensar no es para nosotros un medio para ``conocer" sino para designar el acontecer, para ordenarlo, para volverlo manejable para nuestro uso: así pensamos hoy acerca del pensar: mañana quizá de otro modo ". Esta frase alude, desde mi voz de hoy, a un nihilismo que niega la posibilidad de conocer algo absoluto verdadero pues no es más que un desarrollo pragmático de poder. Es una cuestión de voluntad de voluntad, un dispositivo ordenatorio de la realidad según categorías y características en nuestro acto de querer/poder conocer. Antípoda a esta teoría nihilista aparece el relativismo. Este se estriba en el principio de incertidumbre Heisenberg, si existe ese conocimiento, es entonces indisoluble de cierta estructura. Thomas Khun en su libro \emph{La estructuras de las revoluciones científicas} \cite{kuhn2019estructura} plantea que el método científico revoluciona, cuando se produce un cambio de paradigma, no a partir de la observación de nuevos hechos o fenómenos. Junto con otros destacados sociólogos, acuñan la idea del concepto de ``cargado de teoría", un cierto conjunto de preconceptos anteriores a la observación, descripción y desarrollo de la cualquier investigación, que llevarán al científico demostrar lo que realmente quiere demostrar... de nuevo demostración de poder.

¿Como se demuestran los resultados de esta investigación?, construyendo un conjunto de artefactos experimentales/computacionales que constatan una supuesta realidad casi como un espejo, por correspondencia. En ese proceso de creación o utilización de instrumentos como ser: un programa, un anemómetro o un código computacional existe una omnipresente intervención humana. ¿Vale entonces seguir redactando en tercera persona desde un objetivismo positivista heredado de hace dos siglos? ¿Es coherente no ser impersonal la descripción de un resultado, cuando \textbf{ya} todo el dispositivo ordenatorio que subyace es una construcción humana? ¿Debemos seguir defendiendo un cadáver \textbf{ya} asesinado por las ciencias humanas, desde un \textbf{sujeto que no es más que un efecto} cultural, histórico y económico?. ¡Por una ciencia que tenga con-ciencia de sus puntos ciegos, Por una ciencia con con-ciencia de que la verdad absoluta ha muerto, Por una ciencia construida por personas en primera persona!  


