\chapter{Conclusiones y trabajos a futuro}\label{Cap:Conlcusiones}
\linenumbers

El presente capítulo se separó en dos secciones que se relacionan con diferentes aristas o perspectivas del trabajo llevado a cabo. En primera instancia, se detallan las consideraciones finales y de síntesis, desde un punto de vista técnico sobre los resultados obtenidos. Luego de esto, se analizan limitaciones de esta investigación que deberían de mejorarse en posibles trabajos a futuro.    

\section{Conclusiones técnicas}
Inicialmente se consultó el estado del arte en el área de Ingeniería del viento e Ingeniería Estructural. Se analizaron bibliografías en materia de simulaciones numéricas aplicadas a conductores eléctricos, con abordajes semi-analíticos y computacionales. También, se estudiaron trabajos nacionales e internacionales, desde un punto de vista cualitativo y experimental de CD y sus posibles perjuicios sobre líneas de transmisión eléctrica. Existe vasta evidencia en la literatura de que el fenómeno de \gls{TC} ha afectado severamente la continuidad e integridad de los sistemas de transmisión eléctrica y por tanto la calidad de vida humana. Induciendo inevitablemente, en costos millonarios de reparación sobre las instalaciones, y pérdidas durante la interrupción del suministro. Esta investigación construye una herramienta de simulación computacional, capaz de emular los desmedidos desplazamientos y esfuerzos que estos eventos producen sobre los sistemas de transmisión eléctrica. 

Habiendo finalizado esta tesis, se enumeran las principales conclusiones técnicas respecto de los objetivos planteados en un comienzo y postulados en la Sección \ref{Sec:Intro:Objetivos}.


\begin{description}
	\item[Conclusión T1:] \hfill \\ Se implementó y validó dentro del código abierto \href{https://github.com/ONSAS/ONSAS.m/}{ONSAS} una formulación corrotacional de vigas 3D para la simulación de problemas dinámicos no lineales de estructuras tridimensionales formadas por vigas.
	\item[Conclusión T2:] \hfill \\ Se estudió la literatura y extendió analíticamente la formulación corrotacional para elementos de cables incorporando términos de amortiguamiento aerodinámicos.
	\item[Conclusión T3:] \hfill \\ Se generó un modelo que representa el acoplamiento entre torres y conductores sometido a la acción de un perfil tipo CD. Según los resultados del modelo, las tormentas convectivas afectan a las líneas generando desplazamientos de casi 7 metros y ángulos de hasta 30º en la cadena aisladora. 
\end{description}

\paragraph*{Conclusión T1:}
Debido a numerosas ventajas se eligió la formulación corrotacional de vigas 3D para grandes desplazamientos y rotaciones. Una vez ahondado en la temática, se implementó y validó un modelo corrotacional consistente, robusto y eficaz, capaz de captar y reproducir desplazamientos de gran amplitud con número reducido de elementos. Esta formulación se validó con el Ejemplo \ref{Sec:RN:RightAngle} \emph{benchmark} del folclore corrotacional presentado por \cite{simo1988dynamics}. Este es cargado con una fuerza abrupta en relación a la rigidez de la estructura alcanzando su valor máximo en apenas 2 segundos de simulación, tal y como se muestra en la Figura \ref{fig:RN:RA:Force}. Esta fuerza posee una esencia análoga al fenómeno de TC per se. Esa semejanza radica en el aumento súbito de su magnitud, en un corto lapso de tiempo, por ende, la capacidad del modelo de reproducir este tipo de impactos es fundamental para poder representar el fenómeno central de este trabajo.

En la Figura \ref{fig:RN:RA:DispzA} se observan amplitudes que alcanzan los $8$ m cuando la estructura mide $10$ m. Esto evidencia, la fuerte presencia de grandes desplazamientos y rotaciones.  En relación a los desplazamientos en el sentido de $y$ del nodo A, presentados en la Figura \ref{fig:RN:RA:DispyA}, se observa el signo negativo de este, concordando con lo esperado intuitivamente según la fuerza aplicada. Por último, el resultado más importante de este ejemplo, se obtiene al cotejar las respuestas de las Figuras \ref{fig:RN:RA:DispyA}, \ref{fig:RN:RA:DispzA} y \ref{fig:RN:RA:DispzB} con lo publicado por el artículo de referencia de \cite{Le2014}. Al comparar estas figuras se concluye que el modelo implementado es capaz de representar cabalmente movimientos de gran amplitud, con apenas 10 elementos por miembro y un paso temporal de 0.25 $s$. Esto permitió validar la formulación para este caso y aplicarla a dominios más complejos específicamente con el foco en conductores eléctricos. 

\paragraph*{Conclusión T2:}
En la Sección \ref{Sec:PRE:Modeloviento} se desarrolló un estudio general sobre los campos de velocidades absolutos y relativos, vinculados al efecto relativo del movimiento del conductor respecto al viento. Este enfoque no se encontró en la bibliografía consultada, esclareciéndose la dinámica del fenómeno. A su vez, según la Figura \ref{fig:MET:Viento:VelRel}, se develó que despreciar la velocidad perpendicular frente a la componente media, en el sentido transversal $z$,  es equivalente a que el ángulo de ataque sea nulo y también así, la componente del \textit{drag} según el sentido de $y$. Por otra parte, se concluyó que al considerar los campos relativos aparece un término aeroelástico, que emerge de la diferencia de velocidades, vista desde un referencial solidario al conductor. A este término se lo identifica en la materia con el nombre de amortiguamiento aerodinámico y, según lo estudiado, no había sido incluido en la metodología corrotacional. 

Una vez descritas las hipótesis en este mismo capítulo, en la Sección \ref{Sec:MET:HHT} se generó un análisis analítico no explicado en la bibliografía de referencia \citep{Le2014}. En esta misma sección se aplicó el método de resolución para problemas dinámicos de HHT, incondicionalmente estable en dinámica lineal, explicando con detenimiento la deducción y premisas utilizadas. Complementario al desarrollo teórico, se establecieron los principales pseudocódigos subyacentes a la implementación numérica en el Software \href{https://github.com/ONSAS/ONSAS.m/}{ONSAS}.

En función de los avances originales de esta investigación mencionados en los párrafos anteriores. Esta tesis constituye un desarrollo complementario a la formulación propuesta, por \cite{Le2014}, incluyendo fuerzas aerodinámicas linealizadas provenientes de la acción de vientos sobre conductores. Esto puede aplicarse a un espectro enorme de estructuras representables por elementos de viga, con grandes desplazamientos y rotaciones, atacadas por el viento. Dado este diverso abanico de aplicaciones, el interés de la comunidad científica puede ser un impulso catalizador para ciertas publicaciones a futuro.

\paragraph*{Conclusión T3:}
Se acoplaron diferentes componentes de un sistema de alta tensión: conductores, aisladores y torres. Con el objetivo de estudiar esta interacción entre elementos se implementó el ejemplo de la Sección \ref{Sec:RN:cantileverPendulum}. Este integra elementos de biela tipo Green y de viga corrotacional con resultados lineales y dinámicos aceptables según la respuesta esperada. Luego se avanzó en complejidad y se modeló un sistema de transmisión eléctrica en la Sección \ref{Sec:RN:TransmissionSystem}. Las geometrías y propiedades que integraron el modelo son extraídas de bibliografías experimentales y normativas buscando representar y emular el fenómeno de forma realista.  Con el mismo cometido, el perfil de viento se extrajo de estudios experimentales en el Norte de Alemania durante el transcurso de una tormenta convectiva, tipo corriente descendente, publicado por \cite{stengel2017measurements}. Esta es de una magnitud intensa, aunque no en comparación con los resultados capturados en diferentes estudios de campo nacionales, en \citep{duranona2009analysis} y \citep{duranona2019first}.
% En estos artículos se presentan medidas que alcanzan umbrales de 88.2 a 162 km/h a 45 m de altura. Otra diferencia al respecto, refiere al gradiente de velocidad, el flujo introducido numéricamente del autor \citeauthor{stengel2017measurements} posee una menor aceleración en comparación con tormentas en el territorio uruguayo. 

La carga del viento se distribuyó en el primer vano, provocando un perfil que ataque diferente a la línea en su coordenada axial. Esto genera un efecto de desfasaje entre los conductores, de los vanos entre la torres 1-2 y 2-3 de la Figura \ref{fig:RN:Transmission:Deformadas}. Esta variabilidad del flujo, busca representar un fenómeno de oscilación axial, relacionado con la presencia de vórtices a lo largo del espacio. Las diferencias en desplazamientos de los puntos A, B, C, y D de la cadena aisladora, se evidencia en las Figuras \ref{fig:RN:Transmission:DispsCB} y \ref{fig:RN:Transmission:DispsAD}. Por más que los movimientos posean diferentes amplitudes de banda, los perfiles obtenidos se encuentran gráficamente emparentados con el perfil de la tormenta en la Figura \ref{fig:RN:Transmission:VelocidadTormentaX}. Vale destacar que en estos resultados se evidencia una cierta oscilación de alta frecuencia que puede deberse a inestabilidades numéricas.  

Se desarrolló un análisis de contraste con un modelo ampliamente utilizado en el área de Ingeniería del Viento. Este se utiliza para calcular de forma cuasiestática, utilizando una fórmula de arctoangente. Esta se basa en un péndulo cuasiestático plano, omitiendo términos inerciales. Los trabajos de \cite{stengel2017measurements}, \cite{duranona2009analysis} y \cite{yan2009numerical} aplican esta aproximación simplificadora. Si bien en los resultados del Ejemplo \ref{Sec:RN:TransmissionSystem} no son comprables, la aproximación plana no funciona. Para este caso en particular, la curva numérica parece reflejar una linealidad, evaluar el ángulo de la cadena mediante el modelo estático, arrojaría un resultado sobrestimado. Esto se detalla en la Figura \ref{fig:RN:Trnamission:CurvaCargaDisp}.

Estos resultados presentan indicios que, para enfrentar la problemática, los códigos generados pueden gestar una herramienta de análisis complementario para el diseño de sistemas de transmisión y distribución. Según contactos establecidos con la empresa de transmisión eléctrica (UTE), la contrstucción de torres de alta y media tensión suelen encargarse a empresas privadas que obtienen la obra por licitación y entregan las instalaciones con llave en mano. Estos proyectos suelen importar soluciones del extranjero, que pueden ser no aplicables a las condiciones nacionales. Esto se explica por la carencia de las normas internacionales en materia de fenómenos de viento no sinópticos como CD y ciclones extratropicales. Esto se intensifica en el territorio para sistemas montados hace 30 años en superposición con la asiduidad, intensidad y frecuencia de TC. 

Uniendo resultados de diferentes trabajos internacionales con los resultados del modelo presentado en la Sección \ref{Sec:RN:TransmissionSystem}, es posible teorizar que la mayoría de las incidencias ocurridas en las líneas Palmar-Montevideo de 500kV, pueden deberse al pasaje de tormentas severas sobre la zona. Estas tormentas producen CD, que ejercen cargas desmesuradas sobre el conductor, en el orden de minutos, imponiendo ángulos de balanceo excesivos, acercando los conductores a las torres, a una distancia tal, que inminentes descarga a tierra pueden dejar fuera de servicio a la línea. Además según los estudios de la norma \cite{IEC60826} del Apéndice \ref{Ape2}, esta solo considera vientos tipo CLA. Esto podría estar subestimando las fuerzas ejercidas por el viento, ya que para el territorio uruguayo según \cite{Duranona2018}, los periodos de retorno para velocidades de hasta $100$ km/h es menor para CD en comparación con vientos tipo CLA. 




% Sobre el Ejemplo de FOti
%
% Como primer ejemplo aplicado al modelado de conductores se eligió un problema postulado en la publicación \citep{Foti2016}. Para esto, se investigó la normativa \cite{IEC60826} que detalla propiedades geométricas y constructivas de conductores para alta y media tensión.  Con el fin de cotejar fielmente los resultados obtenidos, se extrajeron, tanto los parámetros del flujo, como las propiedades geométricas y materiales, del trabajo de referencia correspondientes con un conductor DRAKE ASCR 7/26. No obstante, con el objetivo acercar la representación al fenómeno, se incorporaron dos elementos aisladores ilustrativos, que por sus condiciones de borde, no afectan el comportamiento dinámico y cinemático del problema. (Ver Figura \ref{fig:RN:FotiCable:Ilustracion})
% 
% Para este ejemplo de la Sección \ref{Sec:RN:FotiCable}, se aplicó un viento progresivo desde un valor nulo hasta una velocidad de un perfil Capa límite atmosférica en 20 segundos, según la Figura \ref{fig:RN:FotiCable:VelocidadCable}. Este cálculo se realizó considerando las propiedades extraídas de la norma \citep{IEC60826}, explicitadas en la Tabla \ref{table:RN:FotiCable:propiedadesCable}. Al espejar los perfiles de velocidad presentados en las Figuras \ref{fig:RN:FotiCable:DispY} y \ref{fig:RN:FotiCable:DispZ}, con las fuerzas aplicadas de la Ilustración \ref{fig:RN:FotiCable:FuerzaZ} se observa una homología. Esto se fundamenta con un análisis de Foruier donde los desplazamientos ofician de salida y las fuerzas de entrada. 
% 
% Las contribuciones principales del Ejemplo \ref{Sec:RN:FotiCable} se desprenden al contrastar los resultados del ángulo $\Phi$, gratificado en la Figura \ref{fig:RN:FotiCable:Angulo} con los presentados por \citeauthor{Foti2016}. De este análisis se extraen ciertos paralelismos y discordancias. En primer lugar, los perfiles arrojados son semejantes, presentando una relación cuadrática con la velocidad. Esto se atribuye a la función de dependencia cuadrática entre la fuerza y la velocidad media de viento. Sin embargo, para el caso implementado en esta tesis se alcanzan mayores valores de ángulo. Esto puede deberse a múltiples diferencias entre los modelos: la omisión de las componentes turbulentas del flujo, el estado inicial de tensado y la presencia de hielo en las líneas. Los últimos dos factores intuitivamente tienden a disminuir el ángulo máximo alcanzado por la línea, durante el transcurso del movimiento, por su mayor rigidez inicial e inercial. Dado estos resultados, se decidió llevar las simulaciones a un grado mayor de complejidad, e implementar un modelo con múltiples elementos simulando un sistema de trasmisión eléctrica.  

\section{Trabajos a futuro}

Actualmente este trabajo abre claras líneas de investigación y desarrollo para continuar la mejora de los modelos que se aproximen a la realidad con mayor precisión. Como trabajo a futuro para continuar la línea de investigación con un encare general se proponen los siguientes lineamientos:

\begin{enumerate}
	\item Investigar exhaustivamente sobre el origen de las oscilaciones de alta frecuencia observadas en los resultados numéricos del ejemplo de la Sección \ref{Sec:RN:TransmissionSystem}.
	\item Incluir en el análisis teórico de la formulación corrotacional condiciones de Dirichlet no homogéneas en desplazamientos, que sean capaces de representar el tensado del conductor durante la instalación. La hipótesis reduccionista sobre la tensión inicial, aparenta ser imprecisa respecto a la rigidez del sistema y tiende a reducir la exactitud en la representación del fenómeno. Según el punto de vista del autor, esta implementación en \href{https://github.com/ONSAS/ONSAS.m/}{ONSAS} es el punto de partida en la continuación de este trabajo. 
	\item Implementar un módulo modal dentro del \href{https://github.com/ONSAS/ONSAS.m/}{ONSAS} capaz de calcular los modos de vibración de la estructura, insumo fundamental para realizar un análisis en frecuencia de posibles resonancias viento-conductor.
	\item Agregar al desarrollo analítico de la formulación corrotacional la posibilidad de incluir relaciones de fuerza viscosas, no lineales con diferentes coeficientes de \textit{drag} y \textit{lift} de acuerdo al perfil geométrico de la sección e implementarlo en el Software \href{https://github.com/ONSAS/ONSAS.m/}{ONSAS}.
	\item Agregar al modelo del Ejemplo \ref{Sec:RN:TransmissionSystem} los elementos separadores con más de un conductor por aislador. En las instalaciones visitadas de forma presencial, se observaron una serie de separadores que mantienen distanciados los conductores evitando el cortocircuito. Además, al unir cuatro cables generan una mayor rigidez e inercia en los tendidos. Este análisis deberá incluir diferentes valores de coeficientes de \textit{drag} dada la proximidad entre conductores y sus efectos sobre las líneas de flujo.  
	\item Verificar el no deslizamiento interno entre las lingas que conforman el conductor, según los estudios propuestos por \cite{Foti2016}. Esto permitiría verificar la hipótesis asumida respecto al comportamiento de unión que mantiene el conductor durante sus trayectorias. A su vez generar un aporte original estudiando cómo las TC afectan al fenómeno de deslizamiento interno de \cite{Papailiou1997}. 
	\item Generar un análisis de malla en el número de elementos por unidad de largo del conductor y sensibilidad respecto a las condiciones de borde establecidas. Esto permitiría estudiar el grado de discretización óptimo, para minimizar el error numérico sin incurrir en un tiempo excesivo de simulación. 
	\item Integrar la herramienta \href{https://github.com/ONSAS/ONSAS.m/}{ONSAS} con un solver de fluidos como por ejemplo el caffa.3d.MBRi basado en volúmenes finitos con paralelización multifrontal \cite{mendina2014general}. Esta ardua integración permitiría generar una herramienta sumamente potente para atacar problemas de interacción fluido-estructura.	
\end{enumerate}

Con el objetivo de generar una herramienta de diseño complementario para UTE se proponen los siguientes trabajos a futuro:

\begin{enumerate}
	\item Incorporar diferentes geometrías de torres presentes en los distintos tendidos de distribución del país. Según los intercambios con el personal de transmisión de UTE, las líneas de distribución, a partir de la década del 2000, respecto a los que se representaron el Ejemplo \ref{Sec:RN:TransmissionSystem} cambiaron las geometrías de torres. Es importante este análisis para lograr emular la influencia de la arquitectura de las torres, en la aproximación excesiva del conductor a las barras. De igual manera, adquirir datos reales aportados por UTE podría aportar un valor significativo a esta investigación.
	\item Incorporar al modelo el agarre doble que, en determinadas ocasiones, se dispone en las líneas centrales de la torre. Esta es una solución ante la aproximación inminente del aislador, consiste en instalar una cadena aisladora extra que oficia de sujetador adicional para los conductores. Rigidizando y evitando de este modo el balanceo desmesurado. Otro tipo de soluciones implantadas, consiste en agregar pesos sobre puntos estratégicos en las líneas, aumentando la inercia del sistema. En este caso, la elección del peso consiste en un compromiso entre los esfuerzos generados en el cable sin alcanzar la fluencia y la masa que atenúa el balanceo. Este tipo de soluciones paliativas resultan interesantes como objeto de simulación.	
\end{enumerate}



