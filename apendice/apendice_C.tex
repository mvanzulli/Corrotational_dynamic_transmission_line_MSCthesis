\chapter{Reflexiones del autor}\label{Ape3}
\subsection*{Síntesis de formación}
El desarrollo de este trabajo constituyó una instancia de formación fundamental y enriquecedora enmarcada dentro del programa de Maestría en Ingeniería Estructural. Este documento es la síntesis y aplicación de un conjunto de conocimientos profundizados durante la actividad programada, aplicada al modelado numérico de estructuras. La creación de herramientas endógenas con foco en atacar problemáticas a nivel nacional constituye un pilar fundamental en el desarrollo autónomo y original de la ingeniería uruguaya. Este trabajo es una muestra de la convicción y determinación, que el conocimiento académico, debe desarrollarse de forma transparente, comunitaria y democrática. Es por esto, que todos los códigos utilizados en esta investigación se implementaron en la herramienta de software libre \href{https://github.com/ONSAS/ONSAS.m/}{ONSAS}. Esto abre la posibilidad a cualquier tercero, ya sea una organización o persona, de estudiar, modificar y difundir los códigos creados como también aplicarlos a sus propias necesidades. 
\subsection*{Episteme del sujeto}
Antes que nada, es necesario realizar una arqueología de las palabras sujeto y fenómeno en castellano. Sujeto en latín \emph{sub}-{iectum} significa lo que está debajo, según una interpretación posmoderna. Desde esta perspectiva, es el sujeto el sustrato de cualquier ente, que lo dota de sustancia, colores, palabras y formas. Por otra parte, fenómeno tiene una raíz etimológica en la palabra \emph{phainomenon} al igual que la palabra fantasía. Esto alude a lo que se muestra, lo que se deja ver, lo que brilla. Ahora bien, en el acto de percibir cognitivamente existe una dirección previa (inconsciente o consciente) de apuntar el foco hacia algo, entonces ¿Quién y cómo se dirige ese foco?

Toda disciplina e investigación debería conocer sus propias fugas, fronteras y puntos ciegos. De lo contrario, cualquier pretensión hermética podría ser un síntoma de arrogancia y altanería.  A lo largo de este trabajo, he canonizado una redacción en tercera persona, como si existiese una determinada imparcialidad y transparencia en dicho escritor. O quizás una búsqueda con necedad de la verdad absoluta. Este sujeto, apuntado y enfocado en los párrafos siguientes, merece ensimismarse y cuestionarse a sí mismo, según el proverbio en templo de Apolo del Oráculo de Delfos, \emph{gnóthi sautón} o en castellano \emph{Conócete a ti mismo}.

Durante el transcurso de este trabajó me surgieron las siguientes inquietudes ¿Es la realidad un conjunto de fenómenos externos o es siempre un acto de interpretación inmanente al sujeto? Además, ¿Ese sujeto accede la realidad (el objeto) a través de la razón para conocer y explicarla, o simplemente la experiencia es quien valida ese conjunto de fenómenos? A partir de esta pregunta, emana una interrogante natural, ¿Es posible entonces, desligar al sujeto del objeto, o más bien este ente (ex-siste) en el mundo, y está siempre arrojado, lanzado y en relación con el? Y de ser así, ¿No se encuentra entonces \textbf{ya} sugestionado por el paradigma actual, su cultura nativa y sus experiencias personales cuando describe?

Esas preguntas han sido abordadas por eminencias de la filosofía y la ciencia, desde la modernidad hasta hoy. Por un lado, el realismo científico concibe que es posible constatar la realidad a través de la experiencia o a través del pensamiento. Para Descartes ese sujeto duda, piensa y por tanto \textbf{ya} en ese acto analítico, existe (\emph{Cogito ergo sum}) \cite{descartes2004discurso}, ósea el ente en tanto ente. El padre del racionalismo nos plantea que es el yo del sujeto, quien a través de la duda metódica puede acceder la verdad. Contrapuesto a este, el empirismo valida cualquier conocimiento sólo por la experiencia. Esta se define por lo que es captado por nuestros sentidos, es decir que la experiencia es sensorial. Estas dos posturas, la del racionalismo de Descartes y la del empirismo de Hume, pueden ser pensadas como una forma de abordaje a la relación realidad - conocimiento. Para Descartes: conozco en tanto analizo y pienso, y los objetos existen cuando yo realizo la abstracción. Para el empirismo: conozco en la medida en que incorporo la realidad ``objetiva", la de los objetos que puedo percibir a través de los sentidos. 

A mediados del sg XIX nació un pensador disruptivo que viró absolutamente a la cuestión. Frederick Niezstche plantea en su libro Voluntad de Poder \cite{nietzsche2018voluntad} `` El pensar no es para nosotros un medio para ``conocer" sino para designar el acontecer, para ordenarlo, para volverlo manejable para nuestro uso: así pensamos hoy acerca del pensar: mañana quizá de otro modo ". Esta frase alude, desde mi voz de hoy, a un nihilismo que niega la posibilidad de conocer algo absoluto verdadero pues no es más que un desarrollo pragmático de poder. Es una cuestión de voluntad de voluntad, un dispositivo ordenatorio de la realidad según categorías y características en nuestro acto de querer/poder conocer. Antípoda a esta teoría nihilista aparece el relativismo. Este se estriba en el principio de incertidumbre Heisenberg, si existe ese conocimiento, es entonces indisoluble de cierta estructura. Thomas Khun en su libro \emph{La estructuras de las revoluciones científicas} \cite{kuhn2019estructura} plantea que el método científico revoluciona, cuando se produce un cambio de paradigma, no a partir de la observación de nuevos hechos o fenómenos. Junto con otros destacados sociólogos, acuñan la idea del concepto de ``cargado de teoría", un cierto conjunto de preconceptos anteriores a la observación, descripción y desarrollo de la cualquier investigación, que llevarán al científico demostrar lo que realmente quiere demostrar... de nuevo demostración de poder.

¿Como se demuestran los resultados de esta investigación?, construyendo un conjunto de artefactos experimentales/computacionales que constatan una supuesta realidad casi como un espejo, por correspondencia. En ese proceso de creación o utilización de instrumentos como ser: un programa, un anemómetro o un código computacional existe una omnipresente intervención humana. ¿Vale entonces seguir redactando en tercera persona desde un objetivismo positivista heredado de hace dos siglos? ¿Es coherente no ser impersonal la descripción de un resultado, cuando \textbf{ya} todo el dispositivo ordenatorio que subyace es una construcción humana? ¿Debemos seguir defendiendo un cadáver \textbf{ya} asesinado por las ciencias humanas, desde un \textbf{sujeto que no es más que un efecto} cultural, histórico y económico?. ¡Por una ciencia que tenga con-ciencia de sus puntos ciegos, por una ciencia con con-ciencia de que la verdad absoluta ha muerto, por una ciencia construida por personas en primera persona!  