\chapter*{Agradecimientos}

Agradezco al universo por haberme dado hálito de vida a través de ese rió inefable que fluye entre la casualidad y la causalidad. Por haberme maravillado con la lagrima, la risa y el atrapante mundo del conocimiento. Las raices de ese universo son principalmente mi familia, que me nutrieron de valores y vivencias envueltas de un afecto inconmensurable. A mi padre, por haberme enseñado a remar por mis objetivos, pelear por mis proyectos con determinación, sacrificio y sobre todo, por haberme inculcado que no hay que ganarle a nadie, unicamente aprender a levantarse. A mi madre por su incodionalidad eterna, por transferirme la vocación de la enseñanza. Por enseñarme la diversidad de las inteligencias múltiples y sobre todo, la semilla del amor inmenso. A Quique por su sabiduría, su visión biocentrica y su flecha existencial que atraviesa cualquier tormenta. 

También agradezco a mis tutores; A Jorge por ser primero un gran ser humano con una visión fascinante, por enseñarme no solo conocimientos técnicos, sino para la vida. Además por su paciencia, constancia y persistencia para guiarme hacia las en salidas en los laberintos. A Gabriel por darme la oportunidad de dedicarme a la investigación e instruirme desde su experiencia insoslayable en aspectos estratégicos profesionales.   

A Flor por convidarme de sus dulces pétalos y por perfumar cada parte de mi ser con el mas sincero y sano amor. Por ser un alero cuando llueve y dos alas cuando hay sol. Que este camino hubiese sido árido y desolado sin ella. A Máximiliano por estar siempre latente en mi pensamiento, convertir las palabras en aves y despertarme un sin fin de ideas. Por enseñarme la senda de la filosofía, e iluminar el portal donde un punto es la inmensidad, y un segundo la eternidad.

Agradezco enormemente a mis compañeros del IIMPI por guiarme, apoyarme y cuestionarme en este camino de aprendizaje. Por el ambiente relajado de fraternidad y sororidad que hacen del trabajo una instancia de disfrute.

Finalmente, quiero agradecer a la Comisión Académica de Posgrados (CAP) de la Universidad de la República por viabilizar económicamente esta investigación. También a la Agencia Nacional de Investigación (ANII) por financiar el proyecto VioLETa "Modelado del efecto del viento sobre líneas eléctricas de  trasmisión y su mitigación" que fue el pilar indispensable en este trabajo.