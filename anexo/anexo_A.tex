

\subsection{Tensión en el conductor}\label{TensionConductor}
La tensión que debe ser aplicada sobre los conductores se determina a partir del método de deflexión, considérese el caso donde las cadenas aisladoras se encuentra a la misma cota, el conductor tiene un largo L y un peso W por unidad de longitud en N/m, se ilustra un esquema en la siguiente figura:


%
%\begin{figure}[h!]
%	\centering
%	\includegraphics[width=0.6\textwidth]{cuerda.png}
%	\caption{Esquema de un elemento tipo cuerda sobre su propio peso}
%\end{figure}


Considerando el cable como un elemento extensible que no posee rigidez a flexión,  entonces la tensión interna a para cualquier punto de este debe ser tangente a la curva. Sea P un punto cualquiera con coordenadas (x,y) en el cable, tomando equilibrio estático sobre la mitad del conductor y planteado la segunda cardinal o el principio de los trabajos virtuales para un giro arbitrario, desde P, se obtiene la catenaria, y de esta la deflexión máxima en función de la tensión:


\begin{equation}
	Ty=W\frac{x^{2}}{2} \rightarrow \delta= \frac{WL^{2}}{8T}
\end{equation}