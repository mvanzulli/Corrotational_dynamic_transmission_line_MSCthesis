\begin{abstract}

Los sistemas de trasmisión eléctrica son frecuentemente afectadas por eventos climáticos severos como corrientes descendentes o tornados. Estos eventos pueden provocar su desconexión, con consecuencias a la integridad de los componentes  potencialmente graves. En el periodo 2000-2007 se registraron más de veinte eventos de salida en servicio. Otro antecedente de este fenómeno se remonta al 10 de marzo de 2002 cuando una tormenta convectiva afecto un área de alderredor 6500 km$^2$ en el sur del país \citep{tormenta2002}. Este evento fue una destrucción masiva que causó el colapso de 19 torres de trasmisión eléctrica de 500 kV y 48 de 150kV, además de unos 700 edificios y 1250 techos de hogares que fueron destruidos \citep{duranona2015significance}. El costo de reparación de las torres es estimo en 2 millones de dolares y en simultaneo se gastaron unos 10 millones de dolares destinados para suplir la red con energía geotérmica proveniente de combustibles fósiles \citep{duranona2019first}. Esta problemática se superpone a la flanecia de las normas internacional como ser \cite{IEC60826} para considerar fuerzas debidas al impacto de vientos extremos. 

Este trabajo apuntala la creación de una herramienta capaz de reproducir el comportamiento de conductores eléctricos, sometidos a perfiles de viento tipo tormenta convectiva. Para esto, se extendió el planteo de la formulación corrotacional de vigas 3D, considerando componentes aerodinámicos y se implementó  en la herramienta de software libre \emph{Open Non-linear Structural Analysis Solver } (\href{https://github.com/ONSAS/ONSAS/}{ONSAS}). Con este cometido se desarrollaron tres modelos: el primero de ellos valida la formulación para un ejemplo clásico en el área corrotacional, la segunda es una modificación de un modelo presentado en un trabajo de refrentes en simulación estructural de conductores eléctricos, donde se observan resultados semejantes.  Por último, se construye un ejemplo compuesto por tres torres y seis conductores, integrando elementos de viga barras, atacados por un perfil de corriente descendente extraído de un estudio experimental en el norte de Alemania. 

Finalmente, se concluye que los resultados generados representan un disparador para seguir profundizando en la temática, generando capacidades del software para emular el fenómeno de manera mas precias y poder así incluirlo como una herramienta complementaria para el diseño de sistemas de trasmisión. Según los resultados se observa como las tormentas convectivas afectan severamente a las instalaciones y que pueden causar potenciales prejucios graves. De esta forma la metodología planteada en esta tesis constituye el puntapié inicial para la publicación de un trabajo donde se extiende la formulación corrotacional de vigas 3D considerando fuerzas aerodinámicas sobre los elementos. 
\end{abstract}

