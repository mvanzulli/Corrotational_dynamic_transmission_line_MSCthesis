\begin{abstract}

Los sistemas de trasmisión eléctrica son frecuentemente afectados por eventos climáticos severos como corrientes descendentes o tornados. Estos eventos pueden provocar su desconexión con consecuencias a la integridad de los componentes potencialmente graves, así como también a la integridad de las personas circundantes. En el periodo 2000-2007 se registraron más de veinte eventos de salida en servicio. Otro antecedente de este tipo fenómenos, se remonta al 10 de marzo de 2002 cuando una tormenta convectiva afectó un área de alrededor 6500 km$^2$ en el sur del país \citep{tormenta2002}. La tormenta causó una destrucción masiva para el país colapsando 19 torres de trasmisión eléctrica de 500 kV y 48 de 150kV pertenecientes a la empresa \gls{UTE}. De igual modo, unos 700 edificios y 1250 techos de hogares fueron destruidos según \citep{duranona2015significance}. El costo de reparación de las torres se estimó en 2 millones de dólares y en simultaneo se gastaron unos 10 millones de dólares destinados a suplir la red con energía geotérmica, proveniente de combustibles fósiles \citep{duranona2019first}. Esta problemática en parte responde a la falencia de las normas internacional como ser \cite{IEC60826} para considerar fuerzas debidas a fenómenos de vientos extremos. 

Este trabajo apuntala la creación de una herramienta capaz de reproducir el comportamiento de conductores eléctricos, sometidos a perfiles de viento tipo tormenta convectiva. Para esto, se extendió un planteo de la formulación corrotacional de vigas 3D, considerando componentes aerodinámicos y se implementó en la herramienta de software libre \emph{Open Non-linear Structural Analysis Solver} (\href{https://github.com/ONSAS/ONSAS/}{ONSAS}). Con este cometido se desarrollaron tres modelos: el primero de ellos valida la formulación para un ejemplo clásico en el área corrotacional, el segundo es una modificación de un modelo presentado en el trabajo de \citep{Foti2016}, referente en simulación estructural de conductores eléctricos, donde se observan resultados semejantes.  Por último, se construye un ejemplo compuesto por tres torres y seis conductores, integrando elementos de viga con barras, atacados por un perfil de corriente descendente, extraído de un estudio experimental en el norte de Alemania publicado por \cite{stengel2017measurements}. 

Finalmente, se concluye que los resultados generados representan un disparador para seguir profundizando en la temática, generando capacidades del software para emular el fenómeno de manera más precia y poder así, incluirlo como una herramienta complementaria durante el diseño de sistemas de trasmisión. Según los resultados obtenidos, se observa como las tormentas convectivas afectan severamente a las instalaciones, pudiendo causar potenciales prejuicios graves. De esta forma la metodología planteada en esta tesis constituye el puntapié inicial para la publicación de un trabajo donde se extiende la formulación corrotacional de vigas 3D considerando fuerzas aerodinámicas sobre los elementos.

\end{abstract}

