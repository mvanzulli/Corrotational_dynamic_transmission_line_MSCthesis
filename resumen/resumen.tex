\begin{abstract}

En Uruguay los sistemas de transmisión eléctrica son frecuentemente afectados por eventos climáticos severos como corrientes descendentes o tornados. Estos eventos pueden provocar su desconexión con consecuencias a la integridad de los componentes potencialmente graves, así como también a la integridad de las personas circundantes. En el periodo 2000-2007 se registraron más de veinte eventos de salida en servicio y 10 de marzo de 2002 una tormenta convectiva afectó un área de alrededor 6500 km$^2$ en el sur del país. La tormenta causó una destrucción masiva para el país colapsando 19 torres de transmisión eléctrica de 500 kV y 48 de 150kV pertenecientes a la empresa \gls{UTE}. El costo de reparación de las torres fue millonario. Esta problemática representa un desafío, dadas las limitaciones de aplicabilidad de las normativas internacionales ante este tipo de vientos extremos.

Este trabajo está enfocado en desarrollar herramientas capaces de reproducir el comportamiento de conductores eléctricos, sometidos a perfiles de viento tipo tormenta convectiva. Para esto, se integró una formulación corrotacional de vigas 3D, considerando componentes aerodinámicas debido a la acción del viento. Esta formulación fue implementada en la herramienta de software libre \emph{Open Non-linear Structural Analysis Solver} (\href{https://github.com/ONSAS/ONSAS/}{ONSAS}). Se resolvieron tres problemas numéricos aplicando las herramientas desarrolladas: en el primero de ellos sirve se valida la formulación para un ejemplo clásico en la literatura, el segundo es una modificación de un modelo de un conductor propuesto por investigadores referentes en simulación estructural líneas eléctricas, donde se observan resultados semejantes.  Por último, se construye un ejemplo compuesto por tres torres y seis conductores, integrando elementos de viga con barras, sometidos por un perfil de corriente descendente, extraído de un estudio experimental en el norte de Alemania.

Finalmente, se concluye que los resultados generados representan un disparador para seguir profundizando en la temática, generando capacidades nativas para emular el fenómeno de manera más precisa y poder así, incluirlo como una herramienta complementaria durante el diseño de sistemas de trasmisión. Respecto a la metodología se realizó un aporte original incorporando términos aerodinámicos a una formulación corrotacional.  Según los resultados obtenidos, se observa como las tormentas convectivas afectan severamente a las instalaciones, pudiendo causar potenciales prejuicios graves.

\end{abstract}

