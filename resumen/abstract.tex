\begin{foreignabstract}
Overhead transmission lines are frequently affected by severe climate events such as thunderstorms or heavy snowfalls. Such events might cause the disconnection of the line, with potentially severe consequences. Between 2000 and 2007, more than twenty events of disconnection were registered in one of the main transmission lines in Uruguay. On 10th March 2002 a thunderstorm affected an area of 6500 km$^2$ in the south of Montevideo. The storm caused massive destruction collapsing 19 500 kV and 48 150 kV transmission towers. The repair costs were in the millions of dollars. Given the particular features of local winds and temperatures, solutions applied in other countries might not be applicable. This demonstrates the necessity to develop numerical models to enhance the prediction capabilities of these events, guaranteeing in that manner a continuous supply of energy.

The aim of this work is to develop tools capable to reproduce the response of electrical conductors loaded by  thunderstorm wind profiles. A corotational formulation of 3D beams was implemented, considering aerodynamic components caused by the action of the wind. In this work the reference formulation for consistent non-linear dynamic analysis is implemented in the open source software \emph{Open Non-linear Structural Analysis Solver} (\href{https://github.com/ONSAS/ONSAS.m/}{ONSAS}). Four numerical examples are presented: the first one validates the formulation for a classical example in the literature, the second one takes into account the interaction dynamics occurring between the insulator chain and the conductor. The third is a modification of a problem presented in structural simulation of power lines literature.  Finally, a realistic example model is implemented, consisting of three towers and six conductors, integrating beam and truss elements. This example is loaded by a downburst wind profile,  extracted from an experimental study in Northern Germany. The results of this model reveal how convective storms are severely affected and may cause potentially serious damage.

The results obtained in this work represent a trigger for future research on the subject, developing native and endogenous capabilities to emulate the phenomenon more accurately. Moreover, the codes developed in this thesis could be included as a complementary tool during the design of power transmission systems. The methodology of this thesis incorporates aerodynamic terms in a corotational formulation, which is an original contribution of this work.  

\end{foreignabstract}