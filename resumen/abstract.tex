\begin{foreignabstract}
The overhead transmission lines are frequently affected by severe climate events such as thunderstorms
10 or heavy snowfalls. Such events might cause the disconnection of the line, with potentially severe consequences. In
11 the period of 2000- 2007, more than twenty events of disconnection were registered in one of the main transmission
12 lines in Uruguay. Given the particular features of local winds and temperatures, solutions applied in other countries
13 might not be applicable. This demonstrates the necessity to develop numerical models to enhance the prediction
14 capabilities of these events, guaranteeing in that manner a continuous supply of energy.
15 The Universidad de la Repu´blica (UdelaR) counts with research groups working on this problem. The Com16
putational Fluid Mechanics Group (GMFC) is working, since 2004, in the development of computational models
17 of tridimensional fluxes for various applications. The main code developed is called caffa.3d.MBRi and it’s based
18 on the Finite Volume Method, using MPI parallelization. The group called Modelling and Identification in Solids
19 and Structures (MISES) is committed, since its creation in 2018, to the development of numerical codes for struc20
tural analysis. The main code developed is called Open Non-linear Structural Analysis Solver (ONSAS) and it is
21 publicly available.
22 In this work a reference formulation for consistent non-linear dynamic analysis of beam structures using a
23 co-rotational approach is implemented in the ONSAS code. The authors are not aware of any other open imple24
mentation of this formulation available. The implementation is validated using reference problems and also applied
25 to the modelling of high voltage transmission lines considering realistic geometries and loadings.

\end{foreignabstract}